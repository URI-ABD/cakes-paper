% SIAM Supplemental File Template
\documentclass[review,supplement,onefignum,onetabnum]{siamonline220329}

% SIAM Shared Information Template
% This is information that is shared between the main document and any
% supplement. If no supplement is required, then this information can
% be included directly in the main document.


% Packages and macros go here
\usepackage{lipsum}
\usepackage{amsfonts}
\usepackage{graphicx}
\usepackage{epstopdf}
\usepackage{algpseudocode}
\ifpdf
  \DeclareGraphicsExtensions{.eps,.pdf,.png,.jpg}
\else
  \DeclareGraphicsExtensions{.eps}
\fi

% Prevent itemized lists from running into the left margin inside theorems and proofs
\usepackage{enumitem}
\setlist[enumerate]{leftmargin=.5in}
\setlist[itemize]{leftmargin=.5in}

% Add a serial/Oxford comma by default.
\newcommand{\creflastconjunction}{, and~}

% Used for creating new theorem and remark environments
\newsiamremark{remark}{Remark}
\newsiamremark{hypothesis}{Hypothesis}
\crefname{hypothesis}{Hypothesis}{Hypotheses}
\newsiamthm{claim}{Claim}

% Sets running headers as well as PDF title and authors
\headers{CAKES: Scalable, Exact Search on Big Data}{M. E. Prior, T. J. Howard III, O. McLaughlin, T. Ferguson, N. Ishaq, N. M. Daniels}

% Title. If the supplement option is on, then "Supplementary Material"
% is automatically inserted before the title.
\title{Let them have CAKES: A Cutting-Edge Algorithm for Scalable, Efficient, and Exact Search on Big Data\thanks{Submitted to the editors DATE.
}}

\author{
    Morgan E. Prior\thanks{
    Department of Mathematics,
    University of Rhode Island,
    Kingston, RI
    (\email{meprior424@gmail.com})}
    \and
    Thomas J. Howard III\thanks{
    Department of Computer Science and Statistics,
    University of Rhode Island,
    Kingston, RI
    (\email{thoward27@uri.edu}, \email{olwmc@gmail.com}, \email{fergusontr@gmail.com}, \email{najib\_ishaq@zoho.com}, \email{noah\_daniels@uri.edu})}
    \and
    Oliver McLaughlin\footnotemark[3]
    \and
    Terrence Ferguson\footnotemark[3]
    \and
    Najib Ishaq\footnotemark[3]
    \and
    Noah M. Daniels\footnotemark[3]
}


\usepackage{amsopn}
\DeclareMathOperator{\diag}{diag}


%%% Local Variables: 
%%% mode:latex
%%% TeX-master: "ex_article"
%%% End: 

\begin{document}
% Optional PDF information
\ifpdf
\hypersetup{
  pdftitle={Supplementary Materials: An Example Article},
  pdfauthor={D. Doe, P. T. Frank, and J. E. Smith}
}
\fi


\maketitle


\section{Supplementary Results}

\subsection{Indexing and Tuning}

The plots in Figure~\ref{fig:supplement:indexing} show the results of these benchmarks.
The horizontal axis in each subplot shows the cardinality of the dataset augmented with synthetic points.
The left-most point on each line is at the cardinality of the original dataset without any synthetic augmentation.
The vertical axis denotes the sum of indexing and tuning time in seconds.
Both axes are on a logarithmic scale.
Hereafter, when we refer to the ``indexing time'' of an algorithm, we are implicitly referring to the sum of indexing and tuning time for said algorithm.

On all datasets, we observe that the indexing time for CAKES increases roughly linearly as cardinality increases.
HNSW and ANNOY have the slowest indexing times across all the algorithms we benchmarked for each of the four datasets, at each cardinality. 
On some datasets, HNSW and ANNOY exhibit indexing times which are orders of magnitude slower than that of CAKES.
FAISS-Flat exhibits the fastest indexing time on each dataset. 
This is not surprising, however, given that FAISS-Flat is a na\"{\i}ve linear search algorithm and is not building an index.

We also highlight some differences in indexing time between different datasets.
With Fashion-Mnist, as shown in Figure~\ref{fig:supplement:fashion-mnist-indexing}, we observe that the indexing time for CAKES is faster than that of FAISS-IVF for all cardinalities.
With Glove-25 (see Figure~\ref{fig:supplement:glove-25-indexing}), however, at cardinalities greater than $10^7$, FAISS-IVF has faster indexing time than CAKES.
With Sift, CAKES's indexing time is faster than that of FAISS-IVF until a cardinality of nearly $10^8$, and with the Random dataset, we observe that CAKES has faster indexing time than FAISS-IVF until a cardinality of nearly $10^7$.

\begin{figure}
    \begin{subfigure}[b]{0.47\textwidth}
        \includegraphics[width=0.95\textwidth]{plots/fashion-mnist-indexing.png}\\
        \subcaption{Fashion-mnist}
        \label{fig:supplement:fashion-mnist-indexing}
    \end{subfigure}%
    \begin{subfigure}[b]{0.47\textwidth}
        \includegraphics[width=0.95\textwidth]{plots/glove-25-indexing.png}\\
        \subcaption{Glove-25}
        \label{fig:supplement:glove-25-indexing}
    \end{subfigure}
    \vspace{1em}
    \\
    \begin{subfigure}[b]{0.47\textwidth}
        \includegraphics[width=0.95\textwidth]{plots/sift-indexing.png}\\
        \subcaption{Sift}
        \label{fig:supplement:sift-indexing}
    \end{subfigure}%
    \begin{subfigure}[b]{0.47\textwidth}
        \includegraphics[width=0.95\textwidth]{plots/random-indexing.png}\\
        \subcaption{Random}
        \label{fig:supplement:random-indexing}
    \end{subfigure}%
    \\
    \vspace{1em}
    \caption{Indexing and tuning time for each algorithm with each of the ANN benchmark datasets and the Random dataset.}
    \label{fig:supplement:indexing}
\end{figure}

\begin{figure}[ht!]
    \centering
    \includegraphics[width=3.4in]{plots/fashion-mnist_PermutedBall_100_throughput.png}
    \caption{
        Scaling behavior of algorithms on fashion-mnist with $k=100$. 
    }
    \label{fig:supplement:fashion-mnist-k-100}
\end{figure}

\begin{figure}[ht!]
    \centering
    \includegraphics[width=3.4in]{plots/sift_PermutedBall_100_throughput.png}
    \caption{
        Scaling behavior of algorithms on sift with $k=100$. 
    }
    \label{fig:supplement:sift-k-100}
\end{figure}

\begin{figure}[ht!]
    \centering
    \includegraphics[width=3.4in]{plots/glove-25_PermutedBall_100_throughput.png}
    \caption{
        Scaling behavior of algorithms on glove-25 with $k=100$. 
    }
    \label{fig:supplement:glove-25-k-100}
\end{figure}

\begin{figure}[ht!]
    \centering
    \includegraphics[width=3.4in]{plots/sift_PermutedBall_100_throughput.png}
    \caption{
        Scaling behavior of algorithms on random with $k=100$. 
    }
    \label{fig:supplement:random-k-100}
\end{figure}

\subsection{Clustering}

\begin{figure}[ht!]
    \begin{subfigure}[b]{0.47\textwidth}
    \includegraphics[width=0.95\textwidth]{images/radius/fashion-mnist-60000.png}\\
    \subcaption{Fashion-mnist}
    \label{fig:results:fashion-mnist-radius}
    \end{subfigure}%
    \begin{subfigure}[b]{0.47\textwidth}
    \includegraphics[width=0.95\textwidth]{images/radius/glove-25-1183514.png}\\
    \subcaption{Glove-25}
    \label{fig:results:glove-25-radius}
    \end{subfigure}
    \vspace{1em}
    \\
    \begin{subfigure}[b]{0.47\textwidth}
    \includegraphics[width=0.95\textwidth]{images/radius/sift-1000000.png}\\
    \subcaption{Sift}
    \label{fig:results:sift-radius}
    \end{subfigure}%
    \begin{subfigure}[b]{0.47\textwidth}
    \includegraphics[width=0.95\textwidth]{images/radius/radio-ml-97920.png}\\
    \subcaption{RadioML}
    \label{fig:results:radioml-radius}
    \end{subfigure}%
    \\
    \begin{subfigure}[b]{0.47\textwidth}
    \includegraphics[width=0.95\textwidth]{images/radius/silva-2224640.png}\\
    \subcaption{Silva 18S}
    \label{fig:results:silva-radius}
    \end{subfigure}%  
    \begin{subfigure}[b]{0.47\textwidth}
    \includegraphics[width=0.95\textwidth]{images/radius/random-1000000.png}\\
    \subcaption{A random dataset}
    \label{fig:results:random-radius}
    \end{subfigure}
    \vspace{1em}
    \caption{Radius vs. cluster depth across six datasets, grouped by decile of radius and weighted by the cardinalities of the clusters.
    The last dataset is randomly generated.}
    \label{fig:results:radius-plots}
\end{figure}

\begin{figure}[ht!]
    \begin{subfigure}[b]{0.47\textwidth}
    \includegraphics[width=0.95\textwidth]{images/fractal_density/fashion-mnist-60000.png}\\
    \subcaption{Fashion-mnist}
    \label{fig:results:fashion-mnist-fractal_density}
    \end{subfigure}%
    \begin{subfigure}[b]{0.47\textwidth}
    \includegraphics[width=0.95\textwidth]{images/fractal_density/glove-25-1183514.png}\\
    \subcaption{Glove-25}
    \label{fig:results:glove-25-fractal_density}
    \end{subfigure}
    \vspace{1em}
    \\
    \begin{subfigure}[b]{0.47\textwidth}
    \includegraphics[width=0.95\textwidth]{images/fractal_density/sift-1000000.png}\\
    \subcaption{Sift}
    \label{fig:results:sift-fractal_density}
    \end{subfigure}%
    \begin{subfigure}[b]{0.47\textwidth}
    \includegraphics[width=0.95\textwidth]{images/fractal_density/radio-ml-97920.png}\\
    \subcaption{RadioML}
    \label{fig:results:radioml-fractal_density}
    \end{subfigure}%
    \\
    \begin{subfigure}[b]{0.47\textwidth}
    \includegraphics[width=0.95\textwidth]{images/fractal_density/silva-2224640.png}\\
    \subcaption{Silva 18S}
    \label{fig:results:silva-fractal_density}
    \end{subfigure}%  
    \begin{subfigure}[b]{0.47\textwidth}
    \includegraphics[width=0.95\textwidth]{images/fractal_density/random-1000000.png}\\
    \subcaption{A random dataset}
    \label{fig:results:random-fractal_density}
    \end{subfigure}
    \vspace{1em}
    \caption{Fractal Density vs. cluster depth across six datasets, grouped by decile of fractal density and weighted by the cardinalities of the clusters.
    The last dataset is randomly generated.
    Fractal Density is defined as $\frac{cardinality}{radius^{LFD}}$}
    \label{fig:results:fractal_density-plots}
\end{figure}


\bibliographystyle{siamplain}
\bibliography{references}


\end{document}
