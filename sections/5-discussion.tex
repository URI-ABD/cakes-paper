\section{Discussion and Future Work}
\label{sec:discussion-and-future-work}

We have presented CAKES, a suite of three algorithms for fast $k$-NN search across diverse distance functions.
When the distance is a metric (Sec.~\ref{sec:methods}), CAKES is exact.
Even under a non-metric, namely Cosine distance, CAKES achieves near-perfect recall.
Tables~\ref{tab:results:qps-and-recall-fmn}, \ref{tab:results:qps-and-recall-glove} and \ref{tab:results:qps-and-recall-sift}, and Figure~\ref{fig:results:silva-scaling} show that CAKES is most effective when data satisfy the manifold hypothesis, i.e., they occupy a low-dimensional manifold within a high-dimensional space.
Conversely, performance degrades on randomly generated data, where such structure is absent, as shown in Table~\ref{tab:results:qps-and-recall-random} and Figure~\ref{fig:results:radioml-scaling}.
HNSW and ANNOY do maintain high throughput but their recall degrades significantly as cardinality grows, and is near zero on the Random dataset.
FAISS-IVF maintains high recall but its throughput degrades significantly as cardinality grows.

In Figure~\ref{fig:results:lfd-plots}, we show the extent to which each dataset exhibits a manifold structure, as quantified by LFD.
As expected, CAKES scales sub-linearly with cardinality on real-world datasets with low LFD and linearly on synthetic datasets with high LFD;
Sift versus Random illustrates this contrast, as does Radio-ML.

With HNSW, ANNOY, and FAISS-IVF failing to maintain both high recall and high throughput as cardinality grows, CAKES fills an important gap in the $k$-NN search literature.
It is a scalable solution for exact $k$-NN search on large real-world datasets that obey the manifold hypothesis, and whose cardinalities are expected to grow exponentially over time.
Most datasets collected from constrained physical processes (e.g., neural embeddings, genomic sequences, sensor data, etc.) are expected to satisfy the manifold hypothesis, making CAKES widely applicable.

This study can be extended in several directions.
More specialized distance functions, such as Wasserstein~\cite{vallender1974calculation} for probability distributions and Tanimoto~\cite{bajusz2015tanimoto} for molecular structures, would require only a Rust implementation of the distance function itself to be compatible with CAKES.
An examination of the geometric and topological properties of datasets that favor one CAKES algorithm over another would also be valuable.

\subsection{Availability}

CAKES is implemented in Rust and available under the MIT license at https://github.com/URI-ABD/clam.
