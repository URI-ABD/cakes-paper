\section{Methods}
\label{sec:methods}

Write about methods \dots


\subsection{Proofs}
\label{subserc:methods:proofs}

\subsubsection{Guaranteed decreases in Radii}
\label{subsubsec:methods:proofs:radii-decrease}

We take a dataset to be any set $X$ with $n$ instances in $\mathcal{D}$ embedding dimensions where the data follow a low-dimensional manifold with fractal dimension $d$ where $d \ll \mathcal{D}$.

We take a distance function to be $f: X \times X \mapsto \mathbb{R}$ with the following properties:

\begin{itemize}
    \item $f(x, y) = 0 \ iff \ x = y$, identity
    \item $f(x, y) = f(y, x) \ \forall x, y \in X$, symmetry
    \item $f(x - z, y - z) = f(x, y) \ \forall x, y, z \in X$, invariance under translation 
\end{itemize}

We show that the radii of clusters are guaranteed to decrease after every $d$ partitions (at most).

Assume that the data follow a $d$-dimensional distribution embedded in $\mathcal{D}$-dimensional space.
We can describe this distribution choosing some set of $d$ mutually orthagonal axes.
Let $2r$ be the maximum distance between any pair of points along any axis from any choice of mutually orthagonal axes for our $d$-dimensional distribution.
The cluster containing all these points will have a radius of $r$.
In the worst case, the distribution is a $d$-sphere and $2r$ is the maximum distance along every axis from every choice of axes.

The partition algorithm will choose the maximally distant pair of points and, thus, the axis defined by those points as the axis for partitioning the points into two child clusters.
For points in either child cluster, the maximum distance between any pair of points along that axis will have been reduced to $r$ or less.
If $d > 1$ then the child clusters will have a radius $r_{child}$ such that $r_{child} \leq r \sqrt{2}$.
By partitioning the cluster in this way, we have taken one axis and reduced the maximum distance along that axis from $2r$ to $r$.
However, as per the distribution of data as discussed earlier, there may be other axes along which the maximum pairwise distance is still $2r$.
We, thus, recursively partition each child cluster.

With each recursive application of Partition, we consume one axis along which the maximum distance was $2r$ and reduce the maximum distance along that axis to $r$ in the child clusters.
After $d$ recursive applications of partition, we will have exhausted the $d$ axes with the maximum distance of $2r$.
After these $d$ partitions, the points in each child cluster will have a maximum pairwise distance of $r$ along any axis.
Thus the radii of those child clusters will be at most $\frac{r}{2}$.
