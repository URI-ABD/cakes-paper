\documentclass[conference,compsoc]{IEEEtran}

\ifCLASSOPTIONcompsoc
  % IEEE Computer Society needs nocompress option
  % requires cite.sty v4.0 or later (November 2003)
  \usepackage[nocompress]{cite}
\else
  % normal IEEE
  \usepackage{cite}
\fi

\usepackage[utf8]{inputenc} % allow utf-8 input
\usepackage[T1]{fontenc}    % use 8-bit T1 fonts
% \usepackage{hyperref}       % hyperlinks
\usepackage{url}            % simple URL typesetting
\usepackage{booktabs}       % professional-quality tables
\usepackage{amsfonts}       % blackboard math symbols
\usepackage{nicefrac}       % compact symbols for 1/2, etc.
\usepackage{microtype}      % microtypography
\usepackage{lipsum}
\usepackage{graphicx}
% \usepackage{subfigure}
\usepackage{amsmath}
\usepackage{amssymb}
\usepackage{amsthm}
\usepackage{mathtools}
\usepackage{algorithm}
\usepackage{algorithmic}
\usepackage{hyperref}
\usepackage{xr}
\usepackage{subfig}
\usepackage{placeins}

\DeclareMathOperator*{\argmax}{arg\,max}
\DeclareMathOperator*{\argmin}{arg\,min}
\DeclareMathOperator{\score}{score}

% \pdfoutput=1
% below is for arxiv
% \documentclass{article}
% \pdfoutput=1
% \usepackage{arxiv}
%%
%% \BibTeX command to typeset BibTeX logo in the docs
% \ifarxiv
% \else
% \ifpdf
% \hypersetup{
%   pdftitle={Let them have CAKES: A Cutting-Edge Algorithm for Scalable, Efficient, and Exact Search on Big Data},
%   pdfauthor={M. E. Prior, T. J. Howard III, O. McLaughlin, T. Ferguson, N. Ishaq, N. M. Daniels}
% }
% \fi
% \fi

\newcommand{\linebreakand}{%
  \end{@IEEEauthorhalign}
  \hfill\mbox{}\par
  \mbox{}\hfill\begin{@IEEEauthorhalign}
}
\makeatother




% % SIAM Shared Information Template
% This is information that is shared between the main document and any
% supplement. If no supplement is required, then this information can
% be included directly in the main document.


% Packages and macros go here
\usepackage{lipsum}
\usepackage{amsfonts}
\usepackage{graphicx}
\usepackage{epstopdf}
\usepackage{algpseudocode}
\ifpdf
  \DeclareGraphicsExtensions{.eps,.pdf,.png,.jpg}
\else
  \DeclareGraphicsExtensions{.eps}
\fi

% Prevent itemized lists from running into the left margin inside theorems and proofs
\usepackage{enumitem}
\setlist[enumerate]{leftmargin=.5in}
\setlist[itemize]{leftmargin=.5in}

% Add a serial/Oxford comma by default.
\newcommand{\creflastconjunction}{, and~}

% Used for creating new theorem and remark environments
\newsiamremark{remark}{Remark}
\newsiamremark{hypothesis}{Hypothesis}
\crefname{hypothesis}{Hypothesis}{Hypotheses}
\newsiamthm{claim}{Claim}

% Sets running headers as well as PDF title and authors
\headers{CAKES: Scalable, Exact Search on Big Data}{M. E. Prior, T. J. Howard III, O. McLaughlin, T. Ferguson, N. Ishaq, N. M. Daniels}

% Title. If the supplement option is on, then "Supplementary Material"
% is automatically inserted before the title.
\title{Let them have CAKES: A Cutting-Edge Algorithm for Scalable, Efficient, and Exact Search on Big Data\thanks{Submitted to the editors DATE.
}}

\author{
    Morgan E. Prior\thanks{
    Department of Mathematics,
    University of Rhode Island,
    Kingston, RI
    (\email{meprior424@gmail.com})}
    \and
    Thomas J. Howard III\thanks{
    Department of Computer Science and Statistics,
    University of Rhode Island,
    Kingston, RI
    (\email{thoward27@uri.edu}, \email{olwmc@gmail.com}, \email{fergusontr@gmail.com}, \email{najib\_ishaq@zoho.com}, \email{noah\_daniels@uri.edu})}
    \and
    Oliver McLaughlin\footnotemark[3]
    \and
    Terrence Ferguson\footnotemark[3]
    \and
    Najib Ishaq\footnotemark[3]
    \and
    Noah M. Daniels\footnotemark[3]
}


\usepackage{amsopn}
\DeclareMathOperator{\diag}{diag}


%%% Local Variables: 
%%% mode:latex
%%% TeX-master: "ex_article"
%%% End: 

\begin{document}
\title{Let them have CAKES: A Cutting-Edge Algorithm for Scalable, Efficient, and Exact Search on Big Data}

\author{
\IEEEauthorblockN{Morgan E. Prior}
\IEEEauthorblockA{Mathematics\\
University of Rhode Island\\
Kingston, RI 02881\\
meprior424@gmail.com}
\and
\IEEEauthorblockN{Thomas J. Howard III}
\IEEEauthorblockA{Computer Science and Statistics\\
University of Rhode Island\\
Kingston, RI 02881\\
thoward27@uri.edu}
\and
\IEEEauthorblockN{Oliver McLaughlin}
\IEEEauthorblockA{Computer Science and Statistics\\
University of Rhode Island\\
Kingston, RI 02881\\
olwmcjp@gmail.com}
\and
\IEEEauthorblockN{Terrence Ferguson}
\IEEEauthorblockA{Computer Science and Statistics\\
University of Rhode Island\\
Kingston, RI 02881\\
fergusontr@gmail.com}
\and
\IEEEauthorblockN{Najib Ishaq}
\IEEEauthorblockA{Computer Science and Statistics\\
University of Rhode Island\\
Kingston, RI 02881\\
najib\_ishaq@zoho.com}
\and
\IEEEauthorblockN{Noah M. Daniels}
\IEEEauthorblockA{Computer Science and Statistics\\
University of Rhode Island\\
Kingston, RI 02881\\
noah\_daniels@uri.edu}}


\IEEEoverridecommandlockouts
\IEEEpubid{\makebox[\columnwidth]{978-1-6654-3902-2/21/\$31.00~\copyright2021 IEEE \hfill} \hspace{\columnsep}\makebox[\columnwidth]{ }}
% make the title area
\maketitle
\IEEEpubidadjcol
% this must go after the closing bracket ] following \twocolumn[ ...


\begin{abstract}
  The Big Data explosion has created a demand for efficient and scalable algorithms for similarity search.
  While much recent work has focused on \textit{approximate} $k$-NN search, \textit{exact} $k$-NN search has not kept up.
  We present CAKES, a set of three novel algorithms for exact $k$-NN search.
  CAKES's algorithms are generic over \textit{any} distance function, and do not scale with the cardinality or embedding dimension of the dataset. Instead, they scale with geometric properties of the dataset--namely, metric entropy and fractal dimension--thus providing immense speed improvements over existing exact $k$-NN search algorithms when the dataset conforms to the manifold hypothesis.
  We demonstrate these claims by contrasting performance on a randomly-generated dataset against that on some datasets from the ANN-Benchmarks suite under commonly-used distance functions, a genomic dataset under Levenshtein distance, and a radio-frequency dataset under Dynamic Time Warping distance. CAKES exhibits near-constant running time on data conforming to the manifold hypothesis as cardinality grows, and has perfect recall on data in metric spaces. CAKES also has significantly higher recall than state-of-the-art $k$-NN search algorithms even when the distance function is not a metric.
  We conclude that CAKES is a highly efficient and scalable algorithm for exact $k$-NN search on Big Data.
  We provide a Rust implementation of CAKES under an MIT license at https://github.com/URI-ABD/clam.

\end{abstract}
%
% \keywords{K-NN Search, Manifold Hypothesis, Sub-Linear  Algorithms, Big Data}

% REQUIRED
% \ifarxiv
% \else
% \begin{MSCcodes}
% 68P05, 68P10
% \end{MSCcodes}
% \fi
%%
%% This command processes the author and affiliation and title
%% information and builds the first part of the formatted document.
% \maketitle
\IEEEpeerreviewmaketitle
    \section{Introduction}
\label{sec:introduction}

Researchers are collecting data at an unprecedented rate, with datasets in many fields growing exponentially, and outpacing improvements in computing performance predicted by Moore's Law~\cite{kahn2011future}.
Examples include genomic databases~\cite{10.1093/nar/gks1219}, time-series signals~\cite{oshea2018radioml}, and neural embeddings~\cite{2020arXiv200514165B, OpenAI2023GPT4TR, Touvron2023Llama2O, radford2021learning, dosovitskiy2020image}.
This ``Big Data explosion'' has created a need for algorithms that scale efficiently to large datasets.
One such class of algorithms is similarity search;
yet as datasets grow, fast and accurate similarity search becomes increasingly challenging.
Even state-of-the-art algorithms exhibit a steep tradeoff between recall and throughput~\cite{malkov2016hnsw,johnson2019billion,annoy,aumuller2020ann}.

Given some distance function to measure similarity between data points, $k$-nearest neighbors ($k$-NN) search is the task of finding the $k$ points in a dataset that are closest to a given query point.
We use the term \textit{approximate} to refer to algorithms that may not return all true nearest neighbors, while \textit{exact} refers to algorithms that guarantee perfect recall as measured against a brute-force search.
Existing fast algorithms are often approximate~\cite{gao2023high}, which may suffice for some applications, but the need for fast \textit{and} exact search remains~\cite{ukey2023survey}.

This paper introduces CAKES (CLAM-Accelerated $K$-NN Entropy-Scaling Search), a set of three novel algorithms for exact $k$-NN search.
We benchmark these against FAISS~\cite{johnson2019billion}, HNSW~\cite{malkov2016hnsw}, and ANNOY~\cite{annoy}, on datasets from the ANN-benchmarks suite~\cite{aumuller2020ann}.
We also evaluate performance on SILVA 18S~\cite{10.1093/nar/gks1219} using Levenshtein~\cite{levenshtein1966binary} distance on unaligned genomic sequences, and RadioML~\cite{oshea2018radioml} using Dynamic Time Warping (DTW)~\cite{gold2018dynamic} distance on complex-valued time-series.
To further contextualize results, we also use a synthetic dataset with simple statistical properties.


\subsection{Related Works}
\label{sec:intoduction:related-works}

Several algorithms have been developed to tackle $k$-NN search on large datasets, including entropy-scaling search~\cite{yu2015entropy, ishaq2019clustered}, Hierarchical Navigable Small World networks (HNSW)~\cite{malkov2016hnsw}, InVerted File indexing (FAISS-IVF)~\cite{faissivf}, and random projection and tree building (ANNOY)~\cite{annoy}.

HNSW~\cite{malkov2016hnsw} relies on navigable small world (NSW) networks and skip lists.
FAISS-IVF~\cite{faissivf, sacks1987multikey, kent1990signature} partitions data into high-dimensional Voronoi cells, and searches only within cells near the query.
ANNOY~\cite{annoy} uses random projection and tree building.
However, many of these algorithms do not support \emph{exact} search (as defined in Section \ref{sec:introduction}).


\subsection{Entropy-Scaling Search}
\label{sec:intoduction:entropy-scaling-search}

Entropy-scaling algorithms exploit the inherent manifold structure of large datasets, achieving complexity that scales with topological properties, such as metric entropy and fractal dimension, rather than cardinality.
This makes them especially suitable for manifold-constrained data.

In this paper, we introduce CAKES, a set of three entropy-scaling algorithms for $k$-NN search, implemented in Rust.
These algorithms build upon previous work in CHESS~\cite{ishaq2019clustered}, improving the clustering method and extending to $k$-NN search.
We provide a theoretical analysis of CAKES in Sections~\ref{sec:methods:knn-search:repeated-rnn-complexity} and~\ref{sec:methods:knn-search:complexity-of-sieve-methods}.
These analyses are not worst-case analyses in the traditional sense, as they do not assume the worst possible dataset, namely, a random uniform distribution.
Given that CAKES's algorithms are intended for datasets with a manifold structure, an analysis assuming a uniform distribution would be uninformative.
As a result, our analysis assumes datasets with manifold structure, reflecting the conditions for which CAKES is designed.

    \section{Methods}
\label{sec:methods}

In this manuscript, we are primarily concerned with $k$-nearest neighbors in a finite-dimensional 
vector space. Given a dataset $\textbf{X} = \{x_1 \dots x_n\}$, we define a \emph{point} or \emph{datum} $x_i \in \textbf{X}$ as a singular observation (e.g., the genome of 
an organism, the vector representation of a single image, or any entity on which we can define a \emph{distance function}).

We define a \emph{distance function} $f : (\textbf{X}, \textbf{X}) \mapsto \mathbb{R}^+ \cup \{0\}$ as a function which, 
given two points $x, y \in \textbf{X}$, deterministically returns a non-negative real number. We require that the distance function 
be symmetric (i.e., $f(x, y) = f(y, x)$ for all $x, y \in \textbf{X}$) and that the distance between two points $x$ and $y$ be zero if and only if $x = y$. 
When, in addition to these constraints, the distance function obeys
the triangle inequality, it is a \emph{distance metric}, and in such cases we can guarantee that our search algorithms have perfect recall. 
Choice of distance function varies by type of data. For example, with electromagnetic spectra, we use both 
Euclidean (L2) and Cosine distances. With biological or string data, Hamming and Levenshtein distances are more appropriate.


Some of our algorithms for $\rho$- and $k$-nearest neighbors search rely on the \emph{local fractal dimension} at some point in the dataset, 
which we define as: 
\begin{equation} \frac{\text{log}(\frac{|B_X(q, r_1)|}{|B_X(q, r_2)|})}{\text{log}(\frac{r_1}{r_2}) } \label{1} \end{equation}
where $B_X(q, r)$ is the set of points contained in a ball of radius $r$ 
centered at a point $q$ in the dataset $\textbf{X}$; in this example, we compute fractal dimension for some radius $r_1$ and a smaller radius $r_2 = \frac{r_1}{2}$.
We stress that this concept differs from the \emph{embedding dimension} of a dataset. To illustrate the difference,
consider SDSS's APOGEE dataset, wherein each datum is a nonnegative real-valued vector of length 8,575. Hence, the \emph{embedding dimension} of this dataset is 8,575. 
However, due to physical constraints (namely, the laws that govern stellar fusion and spectral emission lines), the data are constrained to a lower-dimensional 
manifold within the 8,575-dimensional embedding space. The \emph{local fractal dimension} is an approximation of the dimensionality of that lower-dimensional manifold at a given point, for some length scale.
The notion that high-dimensional data collected from constrained generating phenomena typically only occupy a low-dimensional manifold is known as the \emph{manifold hypothesis}~\cite{fefferman2016testing}.

\begin{figure}[ht!]
    \centering
    \includegraphics[width=2.5in]{images/lfd_plots/lfd_plot_apogee.png}
    \caption{Mean fractal dimension of APOGEE clusters as a function of
    depth when clustered based on L2 norm. Each plot line represents a distinct
    decile of fractal dimension. Beyond a depth of 56, no clusters are further
    divided due to the minimum-cluster-cardinality stopping criteria.}
    \label{fig:methods:lfd_apogee}
\end{figure}

\begin{figure}[ht!]
    \centering
    \includegraphics[width=2.5in]{images/lfd_plots/lfd_plot_greengenes.png}
    \caption{Mean fractal dimension of GreenGenes clusters as a function of
    depth when clustered based on Hamming distance. Each plot line represents
    a distinct decile of fractal dimension.}
    \label{fig:methods:lfd_greengenes}
\end{figure}

Figure~\ref{fig:methods:lfd_apogee} shows the mean local fractal dimension of the APOGEE dataset at each level of hierarchical clustering, 
under the L2 (Euclidean) distance metric. Each plot line represents a decile of fractal dimension; note that, other than the most extreme 10\% of clusters,
virtually all clusters have a local fractal dimension of less than $2 \ll 8,575$. This suggests that APOGEE is a good candidate for use with CAKES. 


Figure~\ref{fig:methods:lfd_greengenes} shows the mean local fractal dimension of the GreenGenes dataset at each level of hierarchical clustering,
under the Hamming distance metric. Each plot line represents a decile of fractal dimension; once again, other than the most extreme 10\% of clusters,
virtually all clusters have a local fractal dimension of less than $2 \ll 2,250$. This suggests that GreenGenes is a good candidate for use with CAKES.



We define a \emph{cluster} as a set of points with a \emph{center}, a \emph{radius}, and an approximation of the \emph{local fractal dimension}.
The \emph{center} is the geometric median of a sample of points in the \emph{cluster}, and so it is a real data point. The \emph{radius} is the
maximum distance from a point in the cluster to its \emph{center}. We estimate \emph{local fractal dimension} at the cluster radius and half
the cluster radius. Each cluster (unless it is a leaf cluster) has two child clusters in much the same way that a node in
a binary tree has two child nodes. We define the \emph{metric entropy} of a data set under a hierarchical clustering scheme as a refinement of [7], where
metric entropy for a given cluster radius $r_c$ was the number of clusters of radius $r_c$ needed to cover all data. Here, we use a hierarchical, divisive clustering 
approach, but with early stopping criteria; clusters which satisfy a user-specified stopping criterion (e.g. a specified cardinality or radius) are not further split. 
Since we frame the asymptotic complexity of $\rho$-NN search in terms of the number of leaf clusters, the \emph{metric entropy} is best thought of in terms of the number of
leaf clusters. 


With these concepts defined, we can now pose the $k$- and $\rho$- nearest neighbors search problems.
Given a query $q$, along with a distance function $f$ defined on a dataset $\textbf{X}$, $k$-nearest neighbors search aims to find 
the set $S_q$ such that  $|S_q| = k$ and $\forall x \in X \setminus S_q$, $f(q, x) \leq \max\{f(y, q): y \in S_q \}$; that is,
for a given $k$, find the $k$ closest points to $q$ in $ \textbf{X}$.
We also have the $\rho$-nearest neighbors search problem, which aims to find the set $\{x \in \textbf{X}: f(q, x) \leq \rho \}$; that is, 
find all points in $\textbf{X}$ that are at most a distance $\rho$ from $q$.

Given a Cluster $C$, let $c$ be its center and $r$ be its radius. Our $\rho$- and $k$-NN algorithms make use of the following cluster 
properties:
\begin{itemize}
    \item $\delta = f(q, c)$ is the distance from the query to the cluster center $c$.
    \item $\delta_{max} = \delta + r$ is the distance from the query to the theoretically farthest possible instance in $C$.
    \item $\delta_{min} = \text{max}(0, \delta - r)$ is the distance from the query to the theoretically closest possible instance in $C$.
\end{itemize}


We define \emph{singletons} as clusters which either contain a single point (i.e., have cardinality 1) or which contain 
many instances of the same point (i.e., have cardinality greater than 1 but contain only one unique point). Singletons clearly 
have zero radius, and so $\delta_{max} = \delta_{min} = \delta$ for these clusters. Hence, we sometimes overload the above 
notation to refer to distance from a query to a point; in these cases, we also have that $\delta_{max} = \delta = \delta_{min}$.


\subsection{Clustering}
\label{subsec:methods:clustering}

We start by building a divisive hierarchical clustering of the dataset with CLAM, using a 
similar recursive procedure as outlined in CHESS, but with the following 
improvements: better selection of poles for partitioning and depth-first dataset reordering 
(see Section ~\ref{subsubsec:methods:dataset-depth-first-reordering}). 


CLAM assumes the manifold hypothesis. 
In other words, we assume that the dataset is embedded in a $D$-dimensional space, but that the data only occupy 
a $d$-dimensional manifold, where $d \ll D$. 
While we sometimes use Euclidean notions, such as voids and volumes, to talk about geometric and topological 
properties of clusters and of the manifold, CLAM does not rely on such notions; 
they serve merely as a convenient, more intuitive way to talk about the underlying mathematics.

\subsubsection {Cluster Partitioning}

For a cluster $C$ with $|C|$ points, we begin by computing a 
random sample of $\sqrt{|C|}$ of its points, and computing pairwise distances 
between all points in the sample. Based on those pairwise distances, we compute the \emph{geometric median} of this sample; 
that is, the point which minimizes the sum of distances to all other points in the sample. This geometric median 
is $C$'s \emph{center}. The \emph{radius} of $C$ is the maximum distance from any point in $C$ to its center.
The point $l$ which is responsible for that radius (i.e., the furthest point from the center) is designated the \emph{left pole}, and the point $r$ which is furthest
from $l$ is designated the \emph{right pole}. We then partition the cluster into a \emph{left child} and \emph{right child}, where the 
left child contains all points in the cluster which are closer to $l$ than to $r$, and the right child contains all 
points in the cluster which are closer to $r$ than to $l$. Without loss of generality, we assign to the left child 
those points which are equidistant from $l$ and $r$. Starting from a root-cluster containing the entire dataset, we 
repeat this procedure until each leaf contains only one datum or until some other user-specified stopping criterion 
is met.


\begin{algorithm} % enter the algorithm environment
\caption{Cluster Partition} % give the algorithm a caption
\label{alg:partition} % and a label for \ref{} commands later in the document
\begin{algorithmic}[1] % enter the algorithmic environment
    \REQUIRE $cluster$
    \STATE $m \Leftarrow \lfloor \sqrt{|cluster.points|} \rfloor$
    \STATE $seeds \Leftarrow m$ random points from $cluster.points$
    \STATE $c \Leftarrow$ geometric median of $seeds$
    \STATE $l \Leftarrow \argmax d(c,x) \ \forall \ x \in cluster.points$
    \STATE $r \Leftarrow \argmax d(l,x) \ \forall \ x \in cluster.points$
    \STATE $left \Leftarrow \{x | x \in cluster.points \land d(l,x) \le d(r,x)\}$
    \STATE $right \Leftarrow \{x | x \in cluster.points \land d(r,x) < d(l,x)\}$
    \IF{$|left| > 1$}
        \STATE Partition($left$)
    \ENDIF
    \IF{$|right| > 1$}
        \STATE Partition($right$)
    \ENDIF
\end{algorithmic}
\end{algorithm}

\subsubsection {Dataset Depth-First Reordering}
\label{subsubsec:methods:dataset-depth-first-reordering}

CAKES also improves upon CHESS by reordering the dataset. In CHESS, each cluster stored a list of indices into the dataset, 
which we used during search to retrieve a cluster's points from the dataset. 
Though this approach allowed us to retrieve the points in constant 
time, its memory cost was prohibitively high; since each cluster stores indices for each of its
points, for a dataset with cardinality $n$, we store a total of $n$ indices at each depth in the tree.
With $\log{}n$ levels in the tree, this approach had $\mathcal{O}(n\log{}n)$ memory cost. 


One may think to improve this memory cost to $\mathcal{O}(n)$ by storing only the indices of points at 
the leaf cluster level.
This approach, however, introduces an $\mathcal{O}(n)$ time cost whenever we need to find the indices for 
a non-leaf cluster, since it a traversal of the subtree rooted at that cluster.


In this work, we introduce a new approach, wherein after building the cluster tree, we reorder the dataset 
so that points are stored in a depth-first order. Then, within each cluster, we need only 
store the cluster's cardinality and an \emph{offset} to access the cluster's points from the dataset. Parent clusters 
have the same offset as their left child, and right child clusters have an offset equal to their sibling's offset
plus their siblings's cardinality. With no additional memory cost nor time cost for retrieving a points during search, 
dataset depth-first reordering offers significant improvements over the approach used in CHESS.


\subsubsection {Scaling Behavior of Cluster Radii}
\label{subsubsec:methods:guaranteed-decrease-in-cluster-radii}
While it may be tempting to assume that cluster radii decrease with each application of Cluster Partition (refer to Algorithm \ref{alg:partition}), unfortunately, this assumption is incorrect. 
Fortunately, we \emph{can} make some guarantees about the scaling behavior of cluster radii; in particular, we can guarantee that cluster radii will stop increasing after at most 
$d$ partitions, where $d$ is the dimensionality of the dataset.  In the remainder of this subsection, we prove this guarantee. 


We can describe a $d$-dimensional distribution by choosing some set of $d$ mutually orthogonal axes.
Let $2R$ be the maximum pairwise distance among the instances in the dataset. 
We choose the axes such that the two points that are $2R$ apart lie along one of the axes. 
Thus, a $d$-dimensional hyper-sphere of radius $R$ would bound the dataset. 
In the worst case, (i.e., with a uniform-density distribution that fills the $d$-sphere), our axes will be such that $2R$ is the maximum pairwise distance along every axis. 
Such a distribution would also produce a balanced clustering.


Partition will select a maximally distant pair of points to use as poles, i.e., it will choose one of the $d$ axes along
which to split the cluster into two children. 
After one application of Algorithm \ref{alg:partition}, the maximum pairwise distance along that axis will be
bounded above by $R$. 
The next recursive Partition will select another of the $d$ axes. 
Thus, after at most $d$ applications of Algorithm \ref{alg:partition}, the
maximum pairwise distance along each axis will be bounded above by $R$. 
The overall (i.e., not restricted to be along one axis) maximum pairwise distance 
will be bounded above by $R\sqrt{2}$ by, for example, two instances that lie at the extrema of different axes. 

Thus, starting with a cluster $C$ of radius $R$, after at most $d$ Partitions, the descendants of $C$ will each have radius
bounded above by $\frac{R}{\sqrt{2}}$. In other words, cluster radii are guaranteed to decrease by a multiplicative factor of $\frac{\sqrt{2}}{2}$ after at 
most $d$ applications of Algorithm \ref{alg:partition}. 


Note that, in practice, we never see a balanced clustering. Cluster Partition  produces unbalanced trees due to the varying density of the sampling 
in different regions of the manifold and the low dimensional "shape" of the manifold. Further, the cluster radii decrease by a factor much larger than 
$\frac{\sqrt{2}}{2}$ in practice, and the upper bound of $d$ applications of Algorithm \ref{alg:partition} is seldom realized. 


\begin{figure}[ht!]
    \centering
    \includegraphics[width=2.5in]{images/geometry/geometry.pdf}
    \caption{Scaling Behavior of Radii in a Two-Dimensional Uniform-Density Disk}
    \label{fig:methods:scaling_behavior}
\end{figure}

\subsubsection {Complexity}
\label{subsubsec:methods:clustering:complexity}
%A lot of this is lifted from the CHESS paper and could probably get cut out. % 

To analyze the computational complexity of hierarchical
clustering, we start by considering the top level of the hierarchy. 
For a data set with cardinality $n$, we have $n$ data points at this 
level, from which we randomly choose a sample of $\sqrt{n}$ points. 
We compute distances between every pair of seeds in this sample, 
thus making $n$ distance comparisons. Based on these distances 
we compute the center and left and right poles. Once the poles 
have been chosen, every data point except for the two poles is 
compared to both poles, for an additional 2$n$ comparisons. 


In general, at depth $d$ of
the hierarchical tree, we have $\frac{n}{2^d}$ points per cluster (assuming a
perfectly balanced clustering), $\sqrt{\frac{n}{2^d}}$ seeds, $\frac{n}{2^d}$
pairwise seed comparisons, and $\frac{2n}{2^{d}}$ additional comparisons when 
each other data point is compared to the two poles, for a total of $\frac{3n}{2^d}$
distance comparisons. At each level, there are $2^d$ clusters,
so the total number of distance comparisons per level is
3$n$. Thus, the asymptotic complexity in terms of distance
comparisons for a cluster tree of depth $d$ is $\mathcal{O}(dn)$

By using an approximate partitioning with a $\sqrt{n}$ sample, we achieve $\mathcal{O}(n)$ cost of 
partitioning and $\mathcal{O}(n \log n)$ cost of building the tree. This is a significant improvement over 
exact partitioning and tree-building, which have $\mathcal{O}(n^2)$ and $\mathcal{O}(n^2 \log n)$ costs respectively.


\subsection {Index-Building and Sharded Search}
In this work, we also introduce the method of \emph{sharding} for $k$-nearest neighbors search.
With this approach, we aim to leverage the fact that $\rho$-nearest neighbors search is typically 
much faster than $k$-nearest neighbors search. 


We begin by randomly partitioning the dataset into $s$ shards, where $s$ is determined by [TODO: how did 
we decide we're going to select $s$ for this version of the paper?]. On the first shard, we perform 
$k$-nearest neighbors search using one of the four algorithms described in Section ~\ref{subsec:methods:knn-search}.
The result of this search will give us the $k$ closest neighbors to the query in the first shard. We then take the furthest of 
those $k$ neighbors and use its distance from the query as the radius for $\rho$-nearest neighbors search (see Section ~\ref{subsec:methods:rnn-search}) 
on each remaining shard. We then take the union of the results of these $\rho$-nearest neighbors searches, along with the $k$ neighbors
from the first shard, and then linear search over those points to find the actual $k$ nearest neighbors.


Note that the size of the shards is not necessarily consistent; it is often optimal to have the first shard be smaller than the others,
since the first shard is the only one for which we perform $k$-nearest neighbors search. However, making this first shard too small
creates the risk of significantly overestimating the radius, resulting in much slower $\rho$-nearest neighbors search on the remaining shards.
Finding the precise optimal ratio between the size of the first shard and the size of the remaining shards, as well 
as computing the expected level of error between the distance from the $k$-th closest neighbor in the first shard and the 
actual $k$-th closest neighbor, are topics for future work.

\subsection{\texorpdfstring{$\rho$}{p}-Nearest Neighbors Search}
\label{subsec:methods:rnn-search}

We conduct $\rho$-nearest neighbors search as described in \cite{ishaq2019clustered}, but 
with the following improvement: when a cluster overlaps with the query ball, instead of  
always proceeding with both of its children, we only proceed with those children which 
also overlap with the query ball. 

\begin{algorithm} 
    \caption{$\rho$-NN(\emph{clusters, query, r})} 
    \label{alg:rnn} 
    \begin{algorithmic}[2]
        \REQUIRE $r \geq 0$
        \REQUIRE $clusters \neq \emptyset$
        \STATE $results \Leftarrow \emptyset$
        \IF{$clusters.left \land clusters.left \cap B_X(query, r) \neq \emptyset$}
            \IF{$clusters.left.\delta$ $\leq$ $r$ + $clusters.left.radius$}
                \STATE $\rho$-NN($clusters.left, query, r$)
            \ENDIF
        \ENDIF
        \IF{$clusters.right \land clusters.right \cap B_X(query, r) \neq \emptyset$}
            \IF{$clusters.right.\delta$ $\leq$ $r$ + $clusters.right.radius$} 
                \STATE $\rho$-NN($clusters.right, query, r$)
            \ENDIF
        \ENDIF
        \IF{$\neg$$clusters.left$ $\land \neg$$clusters.right$}
            \FOR{$p \in clusters.points$}
                \IF{$p.\delta \leq r$}
                    \STATE $results.push((p, r))$
                \ENDIF
            \ENDFOR
        \ENDIF
        \STATE Return $results$
    \end{algorithmic}
    \end{algorithm}


The asymptotic complexity of $\rho$-nearest neighbors is the same as in ~\cite{ishaq2019clustered}.

\subsection{\texorpdfstring{$k$}{k}-Nearest Neighbors Search}
\label{subsec:methods:knn-search}

In this section, we present four novel algorithms for $k$-nearest neighbors search: $k$-NN by Repeated $\rho$-NN, Sieve Search, and Sieve Search with Separate Centers, 
and  Greedy Sieve. 
We also have an implementation of linear search. We use a preliminary auto-tuning function to predict the variant which will perform 
best for a given query, dataset, and value of $k$. We then proceed with search using that variant.  

In these algorithms, we typically use $H$, for \emph{hits}, to refer to the data structure which stores the closest points to the query found so far and
$Q$ to refer to the data structure which stores the clusters and points which are still in contention for being one of the nearest neighbors.


\subsubsection{$k$-NN by Repeated $\rho$-NN}
\label{subsubsec:methods:knn-search:repeated-rnn}


In this algorithm, we perform $\rho$-nearest neighbors search at a radius equal to the radius of the root cluster divided by
the cardinality of the dataset, repeatedly increasing the search radius until $k$ neighbors
have been found.

Let $H$ be the set of nearest neighbors found thus far.
Search starts with a radius $m$ equal to the radius of the root cluster divided by
the cardinality of the dataset. We perform $\rho$-NN search with radius $m$. 
If no points are within a distance $m$ of the query, we increase the radius by a factor of 
2 and perform $\rho$-NN search again, repeating until at least one point is found, i.e., 
until $|H| > 0$.


Once $|H| > 0$, we continue to perform $\rho$-NN search, but instead of 
increasing the radius by a factor of 2 on each iteration, we increase it by a factor determined 
by the local fractal dimension of the clusters containing the points found so far. In particular, 
we increase the radius by a factor of 
\begin{equation} \text{arg min}\left(2, \left({\frac{k}{|H|}}\right)^{\frac{1}{\mu}}\right) \label{2} \end{equation}
where $\mu$ is the local fractal dimension (abbreviated lfd in algorithm below) in the region around the query point.
Intuitively, the factor by which we increase the radius should be \emph{inversely} related to the number of points found so far. 
Additionally, when the local fractal dimension at the radius scale from the previous iteration is low, this suggests that the data 
are relatively concentrated in that region; thus, the factor of radius increase should be \emph{directly} related to the 
local fractal dimension. Once $|H| >= k$, we return the $k$ closest points.

\begin{algorithm} % enter the algorithm environment
    \caption{Repeated$\rho$-NN(\emph{root, query, k})} % give the algorithm a caption
    \label{alg:knn-by-rnn} % and a label for \ref{} commands later in the document
    \begin{algorithmic}[4] % enter the algorithmic environment
        \STATE $H \Leftarrow$ $\emptyset$
        \STATE $m \Leftarrow$ $\frac{root.radius}{|root|}$
        \WHILE {$|H| = 0$}
            \STATE $H.push(\rho$-NN$(root, query, m)$)
            \STATE $m \Leftarrow 2m$
        \ENDWHILE
        \WHILE {$|H| < k$}
            \STATE $Q \Leftarrow \{ C: \exists p \in H \land p \in C \}$
            \STATE $\mu \Leftarrow \frac{1}{|Q|} \sum_{C \in Q} C.lfd$
            \STATE $m \Leftarrow \text{arg min}\left(2, \left({\frac{k}{|H|}}\right)^{\frac{1}{\mu}}\right)$
            \STATE $H.push(\rho$-NN$(root, query, m)$)
        \ENDWHILE
        \STATE $H.sort()$
        \STATE Return $H[0], H[1], \cdots H[k-1]$
    \end{algorithmic}
    \end{algorithm}


\subsubsection{Sieve Search}
\label{subsubsec:methods:knn-search:sieve}
With Sieve Search, we begin by letting $Q$ be a list initialized with the root cluster. This list will maintain the 
clusters and points which are still in contention for being one of the nearest neighbors.
While $Q$ contains at least one cluster (i.e., $Q$ is not a list of only points), we repeat the process described in 
the following paragraphs. 

First, 
we aim to find the element $q_{\tau} \in Q$ with the smallest $\delta_{max}$ such that 
$q_{\tau}$ and all elements in $Q$ with smaller $\delta_{max}$ collectively contain at least $k$ points. 

To find this element, 
we use a process similar to the Partition algorithm used in Quicksort~\cite{10.1093/comjnl/5.1.10}, adjusted to account for the varying cardinalities of elements in $Q$. Our modified algorithm finds the smallest index $i$ such 
that all elements in $Q$ with $\delta_{max}$ closer to or equal to $Q[i]$'s $\delta_{max}$ collectively have cardinality greater
than or equal to $k$. In the process of finding this index, we alter the order of $Q$ so that all elements to left
of index $i$ have $\delta_{max}$ less than or equal to $Q[i]$'s $\delta_{max}$ and all elements to right of index $i$
have $\delta_{max}$ greater than $Q[i]$'s $\delta_{max}$. We consider $Q[i]$ to be our $q_{\tau}$, and our threshold 
$\tau$ to be $Q[i]$'s $\delta_{max}$.

We then remove from $Q$ any element whose $\delta_{min}$ is greater than $\tau$. Then, for each element left 
in $Q$, if it is a leaf or contains $k$ or fewer points, we remove it and all of its points to $Q$. Finally, we
replace all remaining non-leaf clusters in $Q$ with their children. 

We complete this process until $Q$ contains only points. At this point, we 
just use [INSERT ref to quicksort partition in paper] to find the $k$th 
nearest neighbor. Since this algorithm will also elements on the correct side of 
the index of the $k$th nearest neighbor, the first $k$ elements in $Q$ are the $k$-nearest neighbors.


\begin{algorithm} % enter the algorithm environment
    \caption{QuicksortPartition(\emph{Q, k, l, r})} % give the algorithm a caption
    \label{alg:quicksort-partition} % and a label for \ref{} commands later in the document
    \begin{algorithmic}[5] % enter the algorithmic environment
        \IF{$l \geq r$}
            \STATE Return $min(l, r)$
        \ELSE 
            \STATE $pivot = l + (r - l) / 2$
            \STATE $Q.swap(pivot, r)$
            \STATE $a \Leftarrow l$
            \STATE $b \Leftarrow l$
            \WHILE {$b < r$}
                \IF {$Q[b].\delta_{max} < Q[r].\delta_{max}$}
                    \STATE $Q.swap(a, b)$
                    \STATE $a \Leftarrow a + 1$
                \ENDIF
                \STATE $b \Leftarrow b + 1$
            \ENDWHILE
            \STATE $Q.swap(a, r)$
            \STATE $p \Leftarrow r$
            \STATE $g \Leftarrow \sum_{i=0}^{p} |Q[i]|$
            \IF {$g = k$}
                \STATE Return $p$
            \ELSIF {$g < k$}
                \STATE $QuicksortPartition(Q, k, p + 1, r)$
            \ELSE 
                \IF {$p > 0 \land g > k + |Q[p - 1]|$}
                    \STATE $QuicksortPartition(Q, k, l, p - 1)$
                \ELSIF {$p > 0 \land g = k + |Q[p - 1]|$}
                    \STATE Return $p - 1$
                \ELSE
                    \STATE Return $p$
                \ENDIF
            \ENDIF
        \ENDIF
    \end{algorithmic}
    \end{algorithm}

\begin{algorithm} % enter the algorithm environment
    \caption{Sieve(\emph{tree, query, k})} % give the algorithm a caption
    \label{alg:sieve} % and a label for \ref{} commands later in the document
    \begin{algorithmic}[6] % enter the algorithmic environment
        \REQUIRE $tree$, $query$, $k$
        \STATE $Q \Leftarrow$ [$tree.root$]
        \WHILE{$|Q| > k$}
            \STATE $i \Leftarrow QuicksortPartition(Q, k, 0, |Q| - 1)$
            \STATE $\tau \Leftarrow Q[i].\delta_{max}$
            \FOR {$q \in Q$}
                \IF {$q.\delta_{min} > \tau$}
                    \STATE $Q.pop(q)$
                \ENDIF
            \ENDFOR
            \FOR {$q \in Q$}
                \IF {$q.isLeaf \lor |q| \leq k$}
                    \STATE $Q.push(q.points)$
                \ELSE
                    \STATE $[l, r] \Leftarrow q.children$
                    \STATE $Q.push([l, r])$   
                \ENDIF
                \STATE $Q.pop(q)$
            \ENDFOR 
        \ENDWHILE
        \STATE Return $Q$
    \end{algorithmic}
    \end{algorithm}

\subsubsection{Sieve Search with Separate Centers}
\label{subsubsec:methods:knn-search:sieve2}
Sieve Search with Separate Centers is the same as Sieve Search, but with the following modification: clusters 
in $Q$ are represented twice-- once as their center and once as the rest of their points. 
Because we have that for any cluster $C$, $C.\delta \leq C.\delta_{max}$, representing the cluster 
center separately from the rest of the points causes the threshold $\tau$ to decrease more quickly, 
thus allowing us to eliminate some clusters from $Q$ earlier in the process.


\subsubsection{Greedy Sieve}
\label{subsubsec:methods:knn-search:greedy-search}

With Greedy Sieve, we first let $Q$ be a min queue of clusters by non-increasing $\delta_{min}$. We initialize $Q$ with the root cluster.
Let $H$ be a fixed-size ($k$) max queue of points by $\delta$. This queue starts empty.

While $H$ has fewer than $k$ clusters or $\delta$ of the top priority element in $H$ is greater 
than $\delta_{min}$ of the top priority element in $Q$, do the following:
\begin{enumerate}
\item While the top priority element $q$ in $Q$ is not a leaf, pop $q$ from the queue and push its children.
\item For each point $p \in q$, push $p$ to $H$. 
\item Pop points from $H$ until $H$ has $k$ points. 
\end{enumerate}
After this process, the points left in $H$ are the $k$ nearest neighbors of $q$.

\begin{algorithm} 
\caption{GreedySearch(\emph{root, query, k})} 
\label{alg:greedy_search} 
\begin{algorithmic}[3]
    \STATE $Q \Leftarrow$ priority queue
    \STATE $Q.push(root)$
    \STATE $H \Leftarrow$ priority queue
    \WHILE{$|H| < k \lor H.pop.\delta > Q.pop.\delta_{min}$}
        \WHILE{$\neg Q.pop.isLeaf$}
            \STATE $[l, r] \Leftarrow Q.pop().children$
            \STATE $Q.push([l, r])$
        \ENDWHILE
        \STATE $leaf \Leftarrow Q.pop()$
        \STATE $H.push(leaf)$
        \WHILE{$|H| > k$}
            \STATE $H.pop()$
        \ENDWHILE
    \ENDWHILE
    \STATE Return $H$
\end{algorithmic}
\end{algorithm}

\subsubsection{Complexity of $k$-NN Search}
\label{paragraph:methods:knn-complexity}

To bound the complexity of $k$-NN search, we group all of the "Sieve" algorithms (Sieve, 
Sieve with Separate Centers, and Greedy Sieve) together, and handle $k$-NN by Repeated $\rho$-NN separately.

We start by considering the complexity of $k$-NN by Repeated $\rho$-NN, as this  
is a natural extension of the complexity bounds for $\rho$-NN search described in 
\cite{ishaq2019clustered}. For this algorithm, we adopt the terminology used in 
\cite{ishaq2019clustered} and \cite{yu2015entropy} and address \emph{coarse search} and \emph{fine search} separately. 
Coarse search refers to the process of identifying clusters
which have overlap with the query ball (i.e., which are likely to contain one of the $k$ nearest neighbors). 
Fine search refers to the process
of identifying the $k$ nearest neighbors among the points in those clusters identified by coarse search.




% Our $k$-NN algorithms do not use two completely distinct phases of coarse and fine search, 
% as is the case in $\rho$-NN search. Still, this distinction is useful in analyzing 
% the asymptotic complexity of $k$-NN search. In particular, we can think of the complexity of 
% coarse search as an estimate of how many clusters we examine during search, and 
% the complexity of fine search as an estimate of how many clusters we must examine all the points 
% from during search, even if we do not actually perform these as two distinct phases. 


% Our complexity bounds depend on the same technical assumption about the self-similarity of the dataset made in 
% \cite{yu2015entropy}: if call the number of points within a given 
% radius of a point the \emph{density} around that point, we assume that for this radius, the densities around all points
% are bounded within a constant multiplicative factor $\gamma$ of each other. Formally, 
% we assume that for any point $p$, query $q$, and radius 
% $\rho$, \begin{equation} \frac{1}{\gamma}|B(p, \rho)| \leq \mathbb{E}[|B(q, \rho)|] \leq \gamma |B(p, \rho)|. \label{3} \end{equation}

% For tree search, in the worst case, we must consider [lift ideas here from chess for this justification and ask 
% if this still holds for knn]. 

% To determine the asymptotic complexity of fine search, we must estimate $|F|$, the 
% number of clusters we must examine all the points from during search. With $\rho$-NN 
% search, $F$ is the union of clusters which are within a distance of $r + r_c$ from the query,
% where $r$ is the search radius and $r_c$ is the radius of leaf clusters,
% as in \cite{yu2015entropy}. With $k$-NN search, however, 
% the value of $r$ is dependent on the distance to the $k$th nearest neighbor, which 
% is query-dependent. Thus, for $k$-NN search, we  adjust the definition of 
% $F$ to be the union of clusters which are within a distance of $r' + r_c$ from the query, 
% where $r'$ is the distance to the $k$th nearest neighbor and $r_c$ is defined as before.

% By the triangle inequality, we have that $F \subseteq B(q, r' + 2r_c)$, 
% where $B(q, r' + r_c)$ is the ball of radius $r' + r_c$ centered at $q$.
% Hence, $|F| \leq |B(q, r' + 2r_c)|$.

% Now we have that \begin{align*} \mathbb{E}_q[|B(q, r' + 2r_c)|] &\leq \gamma|B(p, r' + 2r_c)| \\ %because of the assumption that density at different locations on the manifold differ by only a constant factor%
%     &\leq \gamma|B(p, r')|\left(\frac{r' + 2r_c}{r'}\right)^d \label{5}\\ %because d = lfd tells us how the number of additional hits reached scales with a change in radius, and our ratio between radii is (r' + 2r_c)/r'%
%     &\leq \gamma^2 \mathbb{E}[|B(q, r')|]\left(\frac{r'+2r_c}{r'}\right)^d \\ %using density bound again to go from a statement about an arbitrary point back to a statement about the query%
%     &\leq \gamma^2 k\left(\frac{r'+ 2r_c}{r'}\right)^d  \\   %using the fact that for k-nn, the number points in the query ball is k%
% \end{align*}

% It remains to provide an estimate for $r'$. As described in \cite{yu2015entropy} bounded density assumption suggests that increase in number of
% points reached by doubling the radius has to be roughly uniform across all regions of the manifold. Thus, the global 
% average local fractal dimension over all regions of the dataset is not too different from the local fractal dimension around any particular point.
% We rely on this fact in our estimation of $r'$. 

% We let $d'$ be the average of the local fractal dimension across all clusters in the dataset and based on the bounded density assumption, we have that
% $d' \approx d_{0}$, where $d_0$ is the local fractal dimension of the root cluster (i.e., $d_0$ is the logarithm of the ratio between the cardinality of the root cluster 
% and the cardinality of the ball centered at the root clustIn this algorithm, we perform $\rho$-nearest neighbors search at equal to the radius of the cluster tree divided by
% er's center with half its radius). While, ordinarily we compute local fractal dimension 
% by comparing cardinalities of balls centered at some point at two different radii, in order to estimate $r'$, instead compare the cardinality of 
% a ball \emph{around the query} of radius $r'$ to the cardinality \emph{of the root cluster} at its radius.

% If $C_0$ is the root cluster, $r_0$ is its radius, then we have 
% \begin{equation} d_0 = \frac{\log{}\frac{|C_0|}{k}}{\log{}\frac{r_0}{r'}}. \label{4} \end{equation}

% Since $d_0$, $|C_0|$, $r_0$, and $k$ are all known values, we can solve equation 4 for $r'$ to get
% \begin{equation} r' = r_0\left(\frac{k}{|C_0|}\right)^{\frac{1}{d_0}}. \label{6} \end{equation}

    \section{Datasets And Benchmarks}
\label{sec:datasets-and-benchmarks}

\subsection{ANN-Benchmark Datasets}
\label{sec:datasets-and-benchmarks:ann-benchmark-datasets}

We benchmark on a variety of datasets from the ANN-benchmarks suite~\cite{aumuller2020ann}.
Table~\ref{tab:datasets:summary} summarizes these datasets.
All benchmarks were conducted on an Intel Xeon E5-2690 v4 CPU @ 2.60GHz with 512GB RAM.
The OS kernel was Manjaro Linux 5.15.164-1-MANJARO.
The Rust compiler was Rust 1.83.0, and the Python interpreter version was 3.9.18.

\begin{table}
    \caption{Datasets used in benchmarks.}
    \label{tab:datasets:summary}
    \begin{center}
        \begin{sc}
            \begin{tabular}{|l|l|l|l|}
                \hline
                \textbf{Dataset} & \textbf{Dist. Function}  &\textbf{Card}  & \textbf{Dim}  \\
                \hline
                Fashion-Mnist    & Euclidean                   & 60,000             & 784                    \\
                \hline
                Glove-25         & Cosine                      & 1,183,514          & 25                     \\
                \hline
                Sift             & Euclidean                   & 1,000,000          & 128                    \\
                \hline
                Random           & Euclidean                   & 1,000,000          & 128                    \\
                \hline
                SILVA            & Levenshtein                 & 2,224,640          & 3,712         \\
                \hline
                RadioML          & Dynamic Time Warping        & 97,920             & 1,024                  \\
                \hline
            \end{tabular}
        \end{sc}
    \end{center}
    \vskip -0.1in
\end{table}

\subsection{Random Datasets and Synthetic Augmentation}
\label{sec:datasets-and-benchmarks:random-datasets}

In addition to benchmarks on datasets from the ANN-Benchmarks suite, we also benchmarked on synthetic augmentations of these real datasets, using the process described in \ref{sec:methods:synthetic-data}.
In particular, we use a noise tolerance $\epsilon = 0.01$ and explore the scaling behavior as the cardinality multiplier (referred to as ``Mult.'' in Tables~\ref{tab:results:qps-and-recall-fmn},~\ref{tab:results:qps-and-recall-glove},~\ref{tab:results:qps-and-recall-sift} and~\ref{tab:results:qps-and-recall-random}) increases.
We also benchmarked on purely randomly-generated datasets of various cardinalities.
For this, we used a base cardinality of 1,000,000 and a dimensionality of 128 to match the Sift dataset; hereafter, we refer to the random dataset with cardinality 1,000,000 and dimensionality 128 as ``Random.''
This benchmark allows us to isolate the effect of a manifold structure (which we expect to be absent in a purely random dataset) on the performance of the CAKES' algorithms.

In order to calculate recall on the augmented datasets, we perform linear search in Rust and save the results to disk.
We verified that linear search on the original dataset produced neighbors with perfect recall compared to those provided by the ANN-Benchmarks suite.
We then use the results of linear search on the augmented datasets as ground truth for calculating recall of CAKES's algorithms in Rust and read them in Python for calculating recall of HNSW, ANNOY, and FAISS-IVF.


\subsection{SILVA 18S}
\label{sec:datasets-and-benchmarks:silva-18s}

To demonstrate CAKES with a more exotic distance function, we also benchmarked on the SILVA 18S ribosomal RNA dataset~\cite{10.1093/nar/gks1219}.
This dataset contains ribosomal RNA sequences of 2,224,640 genomes, with the longest sequence having 3,712 letters.
We held out a set of 1,000 random sequences from the dataset to use as queries for benchmarking.
We use Levenshtein~\cite{levenshtein1966binary} distance on the unaligned sequences to build the tree and to perform $k$-NN search.
We note that the sequences in this dataset were provided in a multiple sequence alignment with a width of 50,000 characters.
We could have used Hamming distance on the aligned sequences, but we chose to use Levenshtein distance on the unaligned sequences to help demonstrate the flexibility of CAKES in handling exotic distance functions.
Under Hamming distance, we could consider this dataset to have an embedding dimension of 50,000, but for Levenshtein distance, we state the dimensionality as the length of the longest sequence, i.e.\, 3,712.


\subsection{Radio ML}
\label{sec:datasets-and-benchmarks:radio-ml}

As another example of an exotic distance function, we benchmarked CAKES on the Radio-ML dataset~\cite{oshea2018radioml} under Dynamic Time Warping~\cite{muller2007dynamic} as the distance function.
This dataset contains samples of synthetically generated signal captures of different modulation modes over a range of signal-to-noise ratio (SNR) levels.
Specifically, it comprises 24 modulation modes at 26 different SNRs ranging from -20 dB to 30 dB, with 4,096 samples at each combination of modulation mode and SNR level.
Thus, it contains $24 \cdot 26 \cdot 4096 = 2,555,504$ samples in total.
Each sample is a 1,024-dimensional complex-valued vector, representing a signal capture (a time-series of complex-valued numbers).
We used a subset of this dataset, containing $97,304$ samples at 10dB SNR and used another $1,000$ samples at the same SNR as a hold-out set of queries.


\subsection{Other Algorithms}
\label{sec:datasets-and-benchmarks:other-algorithms}

We benchmarked the three CAKES algorithms against a na\"ive linear search implementation in Rust.
We also benchmarked against three state-of-the-art similarity search algorithms: HNSW, ANNOY, and FAISS-IVF in Python.
We verified that our implementation of linear search produces the same neighbors as provided by the ANN-Benchmarks suite for Fashion-MNIST, Glove-25 and Sift datasets.
We then used this linear search implementation to find and store the ground-truth for the augmented versions of the datasets.
We used this ground-truth to calculate recall for CAKES's algorithms in Rust and for HNSW, ANNOY, and FAISS-IVF in Python.
We plot the results of these benchmarks in Figure~\ref{fig:results:scaling-plots}.

    \section{Results}
\label{sec:results}

For every dataset and distance funciton listed in Table~\ref{tab:datasets:summary}, we benchmark $k$-NN search using $k=10$.
Smaller values of $k$, as described in Section~\ref{sec:introduction}, would be too sensitive to local perturbations, while larger values can capture too much of the global structure.

In Figure~\ref{fig:results:scaling-plots}, we show how the throughput of each algorithm scales with cardinality for each of the six datasets we examined.
To do this, we synthetically augmented the Fashion-MNIST, Glove-25, Sift, and Random datasets to create larger datasets using the process described in Section~\ref{sec:methods:synthetic-data}.
For Silva and RadioML, due to the massive sizes of these datasets and challenges in generating plausible augmentations, we took random sub-samples ranging up to the entirety of the dataset to examine how performance scales with cardinality.
These plots also illustrate how the CAKES algorithms compare to na\"{i}ve linear search. 

Tables~\ref{tab:results:qps-and-recall-fmn},~\ref{tab:results:qps-and-recall-glove},~\ref{tab:results:qps-and-recall-sift} and~\ref{tab:results:qps-and-recall-random} compare the performance (throughput and recall) of the CAKES algorithms against state-of-the-art algorithms on the Fashion-MNIST, Glove-25, Sift and Random datasets.
In particular, we examine performance of HNSW, ANNOY and FAISS-IVF on each of those datasets as well as synthetically augmented versions of the datasets to isolate the effect of dataset size on performance.
We did not perform similar benchmarks on the Silva and RadioML datasets because HNSW, ANNOY and FAISS support neither the required distance functions nor, in the case of RadioML, complex-valued data.
For HNSW, ANNOY, and FAISS-IVF, we allow for a hyper-parameter search to tune their index for maximum recall.
For CAKES, we build the tree and use our auto-tuning approach (see Section~\ref{sec:methods:auto-tuning}) to select the fastest algorithm for each dataset and cardinality.

Though the plots in Figure~\ref{fig:results:scaling-plots} present results for each of CAKES's three algorithms separately, the results in the CAKES column in these tables represent the fastest CAKES algorithm at that dataset and cardinality only.

Throughput is measured in queries per second.
A recall value of $1.000*$ denotes imperfect recall that rounds to $1.000$.
In each of Tables \ref{tab:results:qps-and-recall-fmn},\ref{tab:results:qps-and-recall-glove}, \ref{tab:results:qps-and-recall-sift}, we observe that while recall for CAKES does \emph{not} degrade with cardinality, recall for
HNSW and ANNOY does degrade with cardinality. CAKES exhibits perfect recall on the Fashion-MNIST and Sift datasets, and near-perfect recall on the Glove-25 dataset (which uses cosine distance).


% Finally, we test our intuitions about unbalanced clustering being better for search
% than balanced clustering (see Section~\ref{sec:results:clustering-strategies-and-number-of-distance-computations} and Figure~\ref{fig:results:distance-counts}).


% \subsection{Local Fractal Dimension of Datasets}
% \label{sec:results:lfd-of-datasets}

% Since the time complexity of CAKES algorithms scales with the LFD of the dataset, we examine the LFD of each dataset we used for benchmarks.
% Figure~\ref{fig:results:lfd-plots} illustrates the trends in LFD for Fashion-MNIST, Glove-25, Sift, Random, Silva 18S, and Radio-ML.
% In this section, when we discuss trends in LFD, unless otherwise noted, we are referring to the 95$^{th}$ percentile of LFD.
% This is because our algorithms scale exponentially in the LFD, a

% The Fashion-MNIST dataset has an embedding dimension of 784 and uses the Euclidean distance metric.
% In Figure~\ref{fig:results:fashion-mnist-lfd} we observe that until approximately depth 5, Fashion-MNIST's LFD is low (i.e., less than 4).
% It then starts increasing, reaching a peak of about 6 near depth 20, before decreasing to 1 at the maximum depth.

% The Glove-25 dataset has an embedding dimension of 25 and uses the cosine distance function which, notably, is not a metric.
% Relative to Fashion-MNIST, Glove-25 has low LFD, as shown in Figure~\ref{fig:results:glove-25-lfd}.
% All percentile lines for Glove-25 are flatter and lower, indicating that the LFD is lower across the entire dataset, and that the LFD does not vary as much by depth.
% In particular, Glove-25's LFD is less than 3 for all depths.

% The Sift dataset has an embedding dimension of 128 and uses the Euclidean distance metric.
% Figure~\ref{fig:results:sift-lfd} shows the LFD by depth for Sift, which has higher LFD relative to Fashion-MNIST and Glove-25.
% It increases sharply to a peak of 9 around a depth of 10.
% It then decreases smoothly until reaching the deepest leaves in the tree.

% We generated the Random dataset to have the same cardinality and dimensionality as Sift.
% We used a uniform distribution in a 128-dimensional unit-hypercube to generate the points and the Euclidean metric to measure distances among them.
% Figure~\ref{fig:results:random-lfd} shows that the character of this dataset is significantly different from the others.
% The LFD starts at 20 at depth 0 and all percentile lines decrease linearly with depth until reaching the leaves of the tree.
% The spread in LFD starts very small for the first few clusters and increases as depth increases.
% The LFD of approximately 20 for the root cluster $\mathcal{R}$ is what we expect for this random dataset.
% To elaborate, the distribution of points in such a dataset should reflect the curse of dimensionality, i.e.,\,the fact that in high dimensional spaces, the minimum and maximum pairwise distances between any two points are approximately equal.
% As a result, $\mathcal{R}$'s radius $r$, which reflects the maximum distance between the center $c$ and any other point, should not differ significantly from the distance between the center and its closest point.
% A consequence of this is that, with high probability, for every point in $\mathcal{R}$, its distance from $c$ is greater than $\tfrac{r}{2}$;
% in other words, $B(c, \tfrac{r}{2})$ contains only $c$ while $B(c, r)$ contains the entire dataset.
% Given our definition of LFD in Equation~\ref{eq:methods:lfd-half}, this means that the LFD of $\mathcal{R}$ is approximately $\log_2(\frac{|X|}{1}) = \log_2(1,000,000) \approx 20$, which is what we observe in Figure~\ref{fig:results:random-lfd}.
% Theoretically, the LFD of this dataset should be 128, i.e.\, it should be the same as the embedding dimension.
% This reflects the difference between how we empirically measure the LFD and the value we would expect.
% With sample sizes larger than 1,000,000, we would expect the LFD to approach 128 until we have sampled $2^{128}$ points, at which point the LFD would be 128 and would stay at 128 for even larger sample sizes.
% Unfortunately, such a large sample is practically impossible to generate.

% The Silva-18S dataset consists of genomic sequences whose unaligned lengths are at-most 3,712.
% As such the embedding dimension is 3,712, though as discussed previously, the embedding dimension would be 50,000 in a multiple sequence alignment.
% We use the Levenshtein edit distance (a metric) to measure distances between sequences.
% This dataset, as shown in Figure~\ref{fig:results:silva-lfd}, exhibits consistently low LFD.
% In particular, LFD is less than 3 for all depths, hovering near 1 for clusters at depth 40 and deeper.

% The Radio-ML dataset consists of measurements of radio-frequency signals using 1,024 dimensional complex-valued vectors.
% We use the Dynamic Time Warping distance metric on this dataset.
% This dataset is synthetic~\cite{oshea2018radioml} but uses a far more elaborate generation process than our Random dataset.
% The LFD values show three distinct peaks around an LFD of 12 at or near depths of 8, 25 and 50.
% Each peak is followed by a linear decrease until encountering a sharp spike for the next peak.
% Within each of the tree portions, this dataset has a character very similar to that of the Random dataset.
% This suggests that the dataset obeys the manifold hypothesis at some scales, but that it is not ``scale free,'' as the LFD varies significantly by depth.
% This is likely the result of a piecewise uniform sampling strategy used to generate the different modulation modes present in the dataset.

% \begin{figure}
%     \captionsetup[subfigure]{aboveskip=-15pt,belowskip=-3pt}
%     \begin{subfigure}[b]{0.5\textwidth}
%         \includegraphics[width=0.99\textwidth]{images/lfd/fashion-mnist.png}\\
%         \subcaption{Fashion-MNIST}
%         \label{fig:results:fashion-mnist-lfd}
%     \end{subfigure}%
%     \begin{subfigure}[b]{0.5\textwidth}
%         \includegraphics[width=0.99\textwidth]{images/lfd/glove-25.png}\\
%         \subcaption{Glove-25}
%         \label{fig:results:glove-25-lfd}
%     \end{subfigure}
%     \\
%     \begin{subfigure}[b]{0.5\textwidth}
%         \includegraphics[width=0.99\textwidth]{images/lfd/sift.png}\\
%         \subcaption{Sift}
%         \label{fig:results:sift-lfd}
%     \end{subfigure}%
%     \begin{subfigure}[b]{0.5\textwidth}
%         \includegraphics[width=0.99\textwidth]{images/lfd/random.png}\\
%         \subcaption{A random dataset}
%         \label{fig:results:random-lfd}
%     \end{subfigure}
%     \\
%     \begin{subfigure}[b]{0.5\textwidth}
%         \includegraphics[width=0.99\textwidth]{images/lfd/silva-SSU-Ref.png}\\
%         \subcaption{Silva 18S}
%         \label{fig:results:silva-lfd}
%     \end{subfigure}%
%     \begin{subfigure}[b]{0.5\textwidth}
%         \includegraphics[width=0.99\textwidth]{images/lfd/radio-ml.png}\\
%         \subcaption{RadioML}
%         \label{fig:results:radioml-lfd}
%     \end{subfigure}%
%     \\
%     \vskip -0.1in
%     \begin{subfigure}[b]{0.94\textwidth}
%         \centering
%         \includegraphics[width=0.7\textwidth]{images/lfd/legend.png}
%         \label{fig:results:lfd-legend}
%     \end{subfigure}%
%     \vskip -0.1in
%     \caption{Local fractal dimension vs. cluster depth across six datasets. The `random' dataset is randomly generated according to the procedure in Section~\ref{sec:datasets-and-benchmarks:random-datasets}; note that the y-axis is different for this dataset. In each plot, the horizontal axis denotes depth in the cluster tree, and the vertical axis denotes the LFD of clusters at that depth. We show lines for the 5$^{th}$, 25$^{th}$, 50th, 75$^{th}$ and 95$^{th}$ percentiles of LFD, as well as the minimum and maximum LFD at each depth. So that plots best reflect the distribution of LFDs across the entire \textit{dataset}, we count each cluster as many times as its cardinality. For example, if, for some dataset, the 95$^{th}$ percentile of LFD at depth 40 is 3, this means that 95\% of the points in clusters at depth 40 belong to a cluster whose LFD is at most 3.}
%     \label{fig:results:lfd-plots}
%     \vskip -0.4in
% \end{figure}


% \subsection{Indexing and Tuning Time}
% \label{sec:results:indexing-and-tuning-time}

% For each of the ANN-benchmark datasets and the Random dataset, we report the time taken for each algorithm to build the index and to tune the hyper-parameters for these indices to achieve the highest possible recall. For the sake of brevity, these results are reported in the supplement.


% We benchmark the CAKES algorithms, na\"{i}ve linear search, HNSW, ANNOY, and FAISS-IVF on the Fashion-MNIST, Glove-25, Sift, and Random datasets.
% We augment these datasets with synthetic points to examine how performance scales with cardinality.
% We also benchmark the CAKES algorithms on the Silva and RadioML datasets, and we subsample these datasets instead of augmenting them to examine how performance scales with cardinality.

% On the Fashion-MNIST, Glove-25, and Sift datasets (in Figures~\ref{fig:results:fashion-mnist-scaling},~\ref{fig:results:glove-25-scaling},~and~\ref{fig:results:sift-scaling} respectively), we observe that as cardinality increases, the Depth-First Sieve algorithm is consistently the fastest CAKES algorithm with a throughput that is constant in the cardinality of the dataset.
% The Breadth-First Sieve algorithm is the usually the second fastest, also with a nearly constant throughput across all cardinalities.
% The Repeated $\rho$-NN algorithm, however, falls off in throughput as cardinality increases.
% All three CAKES algorithms exhibit perfect recall on the Fashion-MNIST and Sift datasets, and near-perfect recall on the Glove-25 dataset (which uses cosine distance).
% In contrast, HNSW and ANNOY are faster than CAKES's algorithms for all cardinalities and have near constant throughput as cardinality increases, but their recall degrades quickly as cardinality increases.
% FAISS-IVF exhibits linearly decreasing throughput as cardinality increases, which is expected from the algorithm given that we tune the hyper-parameters to maximize recall.

% On the Random dataset, HNSW and ANNOY are still the fastest algorithms but exhibit recall values near 0.
% CAKES's algorithms show linearly decreasing throughput as cardinality increases and are also slower than na\"{i}ve linear search.



\begin{figure}[h]
  \centering
  \subfloat[Fashion-MNIST for $k=10$.]{
    \includegraphics[width=0.48\columnwidth]{plots/fashion-mnist_PermutedBall_10_throughput.png}
    \label{fig:results:fashion-mnist-scaling}
  }
  \subfloat[Glove-25 for $k=10$.]{
    \includegraphics[width=0.48\columnwidth]{plots/glove-25_PermutedBall_10_throughput.png}
    \label{fig:results:glove-25-scaling}
  }

  \subfloat[Sift for $k=10$.]{
    \includegraphics[width=0.48\columnwidth]{plots/sift_PermutedBall_10_throughput.png}
    \label{fig:results:sift-scaling}
  }
  \subfloat[Random dataset for $k=10$.]{
    \includegraphics[width=0.48\columnwidth]{plots/random_PermutedBall_10_throughput.png}
    \label{fig:results:random-scaling}
  }

  \subfloat[Silva for $k=10$.]{
    \includegraphics[width=0.48\columnwidth]{plots/silva-SSU-Ref_PermutedBall_10_throughput.png}
    \label{fig:results:silva-scaling}
  }
  \subfloat[RadioML for $k=10$ at SnR = 10dB.]{
    \includegraphics[width=0.48\columnwidth]{plots/radio-ml_Ball_10_throughput.png}
    \label{fig:results:radioml-scaling}
  }

  \vspace{2pt}
  \includegraphics[width=1\columnwidth]{plots/legend.png}

  \caption{Throughput across six datasets, including a randomly-generated dataset.
  Each plot shows throughput (queries per second; higher is better) versus dataset cardinality. For Fashion-MNIST, Glove-25, and Sift, CAKES becomes faster than linear search as cardinality grows, with the crossover point varying by dataset. For Fashion-MNIST and Glove-25, Depth-First Sieve is consistently fastest. For Sift, Repeated $\rho$-NN is fastest at small cardinalities, while Depth-First Sieve is fastest at large cardinalities. For Silva, throughput for all algorithms initially decreases approximately linearly with cardinality and then levels off at higher cardinalities; Depth-First Sieve is consistently fastest. For Radio-ML and Random, all CAKES variants are slower than na\"{i}ve linear search, and their throughput decreases linearly with cardinality. HNSW and ANNOY are the fastest algorithms on all four datasets we benchmarked them on, but their recall degrades quickly as cardinality increases on all datasets; on Random, their recall is near zero.}
  \label{fig:results:scaling-plots}
\end{figure}

% \begin{table}[t]
%     \centering
%     \caption{Fashion-MNIST: throughput and recall. See general notes above.}
%     \label{tab:results:qps-and-recall-fmn}
%     \small
%     \setlength{\tabcolsep}{4pt}
%     \begin{adjustbox}{width=\columnwidth,center}
%     \begin{tabular}{@{} lcccccccc @{}}
%     \toprule
%     \textbf{M} &
%     \multicolumn{2}{c}{\textbf{HNSW}} &
%     \multicolumn{2}{c}{\textbf{ANNOY}} &
%     \multicolumn{2}{c}{\textbf{FAISS-IVF}} &
%     \multicolumn{2}{c}{\textbf{CAKES}} \\
%     \cmidrule(lr){2-3}\cmidrule(lr){4-5}\cmidrule(lr){6-7}\cmidrule(lr){8-9}
%     & QPS & Recall & QPS & Recall & QPS & Recall & QPS & Recall \\
%     \midrule
%     1   & \num{1.33e4} & 0.95 & \num{2.19e3} & 0.95 & \num{2.01e3} & $1.00^{*}$ & \num{3.46e3} & 1.00 \\
%     2   & \num{1.38e4} & 0.80 & \num{2.12e3} & 0.93 & \num{9.39e2} & $1.00^{*}$ & \num{3.68e3} & 1.00 \\
%     4   & \num{1.66e4} & 0.68 & \num{2.04e3} & 0.90 & \num{4.61e2} & $1.00^{*}$ & \num{3.44e3} & 1.00 \\
%     8   & \num{1.68e4} & 0.53 & \num{1.93e3} & 0.86 & \num{2.26e2} & $1.00^{*}$ & \num{3.30e3} & 1.00 \\
%     16  & \num{1.87e4} & 0.49 & \num{1.84e3} & 0.86 & \num{1.17e2} & 0.99       & \num{3.34e3} & 1.00 \\
%     32  & \num{1.56e4} & 0.54 & \num{1.85e3} & 0.78 & \num{5.91e1} & 0.99       & \num{2.96e3} & 1.00 \\
%     64  & \num{1.50e4} & 0.38 & \num{1.78e3} & 0.69 & \num{2.61e1} & 0.97       & \num{3.25e3} & 1.00 \\
%     128 & \num{1.49e4} & 0.36 & \num{1.66e3} & 0.54 & \num{1.33e1} & 0.96       & \num{2.96e3} & 1.00 \\
%     256 & --           & --    & \num{1.60e3} & 0.59 & \num{6.65e0} & 0.96       & \num{2.79e3} & 1.00 \\
%     512 & --           & --    & \num{1.83e3} & 0.58 & \num{3.56e0} & 0.95       & \num{2.84e3} & 1.00 \\
%     \bottomrule
%     \end{tabular}
%     \end{adjustbox}
%     \end{table}

\begin{table}[t]
  \centering
  \begin{tiny}
  \caption{Fashion-MNIST: throughput and recall.}
  \end{tiny}
  \label{tab:results:qps-and-recall-fmn}
  \small
  \setlength{\tabcolsep}{4pt}
  \begin{adjustbox}{width=\columnwidth,center}
  \begin{tabular}{@{} lllllllll @{}}
    \toprule
    \textbf{Mult.} &
    \multicolumn{2}{c}{\textbf{HNSW}} &
    \multicolumn{2}{c}{\textbf{ANNOY}} &
    \multicolumn{2}{c}{\textbf{FAISS-IVF}} &
    \multicolumn{2}{c}{\textbf{CAKES}} \\
    \cmidrule(lr){2-3}\cmidrule(lr){4-5}\cmidrule(lr){6-7}\cmidrule(lr){8-9}
    & QPS & Recall & QPS & Recall & QPS & Recall & QPS & Recall \\
    \midrule
    1   & \num{1.33e4} & 0.954 & \num{2.19e3} & 0.950 & \num{2.01e3} & $1.000^{*}$ & \num{3.46e3} & 1.000 \\
    2   & \num{1.38e4} & 0.803 & \num{2.12e3} & 0.927 & \num{9.39e2} & $1.000^{*}$ & \num{3.68e3} & 1.000 \\
    4   & \num{1.66e4} & 0.681 & \num{2.04e3} & 0.898 & \num{4.61e2} & 0.997       & \num{3.44e3} & 1.000 \\
    8   & \num{1.68e4} & 0.525 & \num{1.93e3} & 0.857 & \num{2.26e2} & 0.995       & \num{3.30e3} & 1.000 \\
    16  & \num{1.87e4} & 0.494 & \num{1.84e3} & 0.862 & \num{1.17e2} & 0.991       & \num{3.34e3} & 1.000 \\
    32  & \num{1.56e4} & 0.542 & \num{1.85e3} & 0.775 & \num{5.91e1} & 0.985       & \num{2.96e3} & 1.000 \\
    64  & \num{1.50e4} & 0.378 & \num{1.78e3} & 0.677 & \num{2.61e1} & 0.968       & \num{3.25e3} & 1.000 \\
    128 & \num{1.49e4} & 0.357 & \num{1.66e3} & 0.538 & \num{1.33e1} & 0.964       & \num{2.96e3} & 1.000 \\
    256 & --           & --    & \num{1.60e3} & 0.592 & \num{6.65e0} & 0.962       & \num{2.79e3} & 1.000 \\
    512 & --           & --    & \num{1.83e3} & 0.581 & \num{3.56e0} & 0.949       & \num{2.84e3} & 1.000 \\
    \bottomrule
  \end{tabular}
  \end{adjustbox}
\end{table}

\begin{table}
  \centering
  \caption{Glove-25: throughput and recall.}
  \label{tab:results:qps-and-recall-glove}
  \small
  \setlength{\tabcolsep}{4pt}
  \begin{adjustbox}{width=\columnwidth,center}
  \begin{tabular}{@{} lllllllll @{}}
    \toprule
    \textbf{Mult.} &
    \multicolumn{2}{c}{\textbf{HNSW}} &
    \multicolumn{2}{c}{\textbf{ANNOY}} &
    \multicolumn{2}{c}{\textbf{FAISS-IVF}} &
    \multicolumn{2}{c}{\textbf{CAKES}} \\
    \cmidrule(lr){2-3}\cmidrule(lr){4-5}\cmidrule(lr){6-7}\cmidrule(lr){8-9}
    & QPS & Recall & QPS & Recall & QPS & Recall & QPS & Recall \\
    \midrule
    1   & \num{2.28e4} & 0.801 & \num{2.83e3} & 0.835 & \num{2.38e3} & 1.000* & \num{1.54e3} & 1.000* \\
    2   & \num{2.38e4} & 0.607 & \num{2.70e3} & 0.832 & \num{1.19e3} & 1.000* & \num{1.49e3} & 1.000* \\
    4   & \num{2.50e4} & 0.443 & \num{2.61e3} & 0.839 & \num{6.19e2} & 1.000* & \num{1.28e3} & 1.000* \\
    8   & \num{2.78e4} & 0.294 & \num{2.51e3} & 0.834 & \num{3.03e2} & 1.000* & \num{1.30e3} & 1.000* \\
    16  & \num{3.11e4} & 0.213 & \num{2.23e3} & 0.885 & \num{1.51e2} & 1.000* & \num{1.14e3} & 1.000* \\
    32  & \num{3.24e4} & 0.178 & \num{2.01e3} & 0.764 & \num{7.40e1} & 0.999  & \num{1.05e3} & 1.000* \\
    64  & --           & --    & \num{1.99e3} & 0.631 & \num{3.77e1} & 0.997  & \num{1.07e3} & 1.000* \\
    128 & --           & --    & --           & --    & \num{1.90e1} & 0.998  & \num{8.92e2} & 1.000* \\
    256 & --           & --    & --           & --    & \num{9.47e0} & 0.998  & \num{8.91e2} & 1.000* \\
    \bottomrule
  \end{tabular}
  \end{adjustbox}
\end{table}


\begin{table}
  \centering
  \caption{Sift: throughput and recall.}
  \label{tab:results:qps-and-recall-sift}
    \small
    \setlength{\tabcolsep}{4pt}
    \begin{adjustbox}{width=\columnwidth,center}
    \begin{tabular}{@{} lllllllll @{}}
    \toprule
    \textbf{Mult.} &
    \multicolumn{2}{c}{\textbf{HNSW}} &
    \multicolumn{2}{c}{\textbf{ANNOY}} &
    \multicolumn{2}{c}{\textbf{FAISS-IVF}} &
    \multicolumn{2}{c}{\textbf{CAKES}} \\
    \cmidrule(lr){2-3}\cmidrule(lr){4-5}\cmidrule(lr){6-7}\cmidrule(lr){8-9}
    & QPS & Recall & QPS & Recall & QPS & Recall & QPS & Recall \\
    \midrule
    1   & \num{1.93e4} & 0.782 & \num{3.98e3} & 0.686 & \num{6.98e2} & 1.000* & \num{6.20e2} & 1.000 \\
    2   & \num{2.03e4} & 0.552 & \num{3.80e3} & 0.614 & \num{3.30e2} & 1.000* & \num{2.95e2} & 1.000 \\
    4   & \num{2.18e4} & 0.394 & \num{3.69e3} & 0.637 & \num{1.65e2} & 1.000* & \num{1.76e2} & 1.000 \\
    8   & \num{2.48e4} & 0.298 & \num{3.58e3} & 0.710 & \num{7.72e1} & 1.000* & \num{1.27e2} & 1.000 \\
    16  & \num{2.68e4} & 0.210 & \num{3.50e3} & 0.690 & \num{3.98e1} & 1.000* & \num{1.47e2} & 1.000 \\
    32  & \num{2.75e4} & 0.193 & \num{3.44e3} & 0.639 & \num{2.09e1} & 0.999  & \num{1.24e2} & 1.000 \\
    64  & --           & --    & \num{3.39e3} & 0.678 & \num{8.87e0} & 0.997  & \num{1.34e2} & 1.000 \\
    128 & --           & --    & \num{3.36e3} & 0.643 & \num{4.78e0} & 0.993  & \num{1.31e2} & 1.000 \\
    \bottomrule
  \end{tabular}
  \end{adjustbox}
\end{table}


\begin{table}
  \caption{Random dataset: throughput and recall.
  In contrast with the results on the ANN Benchmark datasets reported above, with the Random dataset, we observe that CAKES's algorithms perform quite slowly.
  CAKES exhibits perfect recall at all cardinalities, whereas HNSW and ANNOY exhibit \textit{much} lower recall on this random dataset than on any of the ANN benchmark datasets.}
  \label{tab:results:qps-and-recall-random}
  \small
  \setlength{\tabcolsep}{4pt}
  \begin{adjustbox}{width=\columnwidth,center}
  \begin{tabular}{@{} lllllllll @{}}
    \toprule
    \textbf{Mult.} &
    \multicolumn{2}{c}{\textbf{HNSW}} &
    \multicolumn{2}{c}{\textbf{ANNOY}} &
    \multicolumn{2}{c}{\textbf{FAISS-IVF}} &
    \multicolumn{2}{c}{\textbf{CAKES}} \\
    \cmidrule(lr){2-3}\cmidrule(lllllllll){4-5}\cmidrule(lr){6-7}\cmidrule(lr){8-9}
    & QPS & Recall & QPS & Recall & QPS & Recall & QPS & Recall \\
    \midrule
    1  & \num{1.17e4} & 0.060 & \num{4.28e3} & 0.028 & \num{7.342} & 1.000* & \num{6.06e2} & 1.000 \\
    2  & \num{1.01e4} & 0.048 & \num{4.04e3} & 0.021 & \num{3.582} & 1.000* & \num{2.75e2} & 1.000 \\
    4  & \num{9.12e3} & 0.031 & \num{3.64e3} & 0.014 & \num{1.902} & 1.000* & \num{1.35e2} & 1.000 \\
    8  & \num{8.35e3} & 0.022 & \num{3.37e3} & 0.013 & \num{8.841} & 1.000* & \num{6.13e1} & 1.000 \\
    16 & \num{8.25e3} & 0.008 & \num{3.17e3} & 0.006 & \num{4.361} & 1.000* & \num{2.82e1} & 1.000 \\
    32 & --           & --    & \num{3.01e3} & 0.007 & \num{1.721} & 1.000* & \num{1.31e1} & 1.000 \\
    \bottomrule
  \end{tabular}
  \end{adjustbox}
\end{table}





% \begin{figure}
%     \begin{subfigure}[b]{0.5\textwidth}
%         \includegraphics[width=0.99\textwidth]{images/distance_counts/fashion-mnist_KnnRepeatedRnn_10_throughput.png}
%         \subcaption{Repeated $\rho$-NN}
%         \label{fig:results:fashion-mnist-counts-throughput}
%     \end{subfigure}%
%     \begin{subfigure}[b]{0.5\textwidth}
%         \includegraphics[width=0.99\textwidth]{images/distance_counts/fashion-mnist_KnnRepeatedRnn_10_counts.png}
%         \subcaption{Repeated $\rho$-NN}
%         \label{fig:results:glove-25-counts-counts}
%     \end{subfigure}%
%     \\
%     \begin{subfigure}[b]{0.5\textwidth}
%         \includegraphics[width=0.99\textwidth]{images/distance_counts/fashion-mnist_KnnBreadthFirst_10_throughput.png}
%         \subcaption{Breadth First Sieve}
%         \label{fig:results:sift-counts-throughput}
%     \end{subfigure}%
%     \begin{subfigure}[b]{0.5\textwidth}
%         \includegraphics[width=0.99\textwidth]{images/distance_counts/fashion-mnist_KnnBreadthFirst_10_counts.png}
%         \subcaption{Breadth First Sieve}
%         \label{fig:results:random-counts-counts}
%     \end{subfigure}%
%     \\
%     \begin{subfigure}[b]{0.5\textwidth}
%         \includegraphics[width=0.99\textwidth]{images/distance_counts/fashion-mnist_KnnDepthFirst_10_throughput.png}
%         \subcaption{Depth First Sieve}
%         \label{fig:results:silva-counts-throughput}
%     \end{subfigure}%
%     \begin{subfigure}[b]{0.5\textwidth}
%         \includegraphics[width=0.99\textwidth]{images/distance_counts/fashion-mnist_KnnDepthFirst_10_counts.png}
%         \subcaption{Depth First Sieve}
%         \label{fig:results:radioml-counts-counts}
%     \end{subfigure}%
%     \\
%     \begin{subfigure}[b]{0.94\textwidth}
%         \centering
%         \includegraphics[width=0.6\textwidth]{images/distance_counts/legend.png}
%         \label{fig:results:counts-legend}
%     \end{subfigure}%
%     \caption{Number of distance computations across four clustering strategies and three search algorithms on the Fashion-MNIST dataset.
%     Adding the instrumentation to count the number of distance computations had the side-effect of significantly slowing down the search algorithms compared to those reported in Figure~\ref{fig:results:scaling-plots}.
%     The left column shows the throughput in queries per second, while the right column shows the mean number of distance computations per query.
%     The x-axis represents increasing cardinality of the dataset.}
%     \label{fig:results:distance-counts}
% \end{figure}

    \section{Discussion}
\label{sec:discussion}

Discuss results and future work \dots

CLAM-CAKES.
This is most effective when datasets exhibit low metric entropy and low local fractal dimension.

Unlike LSH, this is extensible to any user provided distance function.

Exactness of search and metric vs non-metric.


\subsection{Future Works}
\label{subsec:results:future-works}

GPU-acceleration for the more computationally intensive distance functions like wasserstein.

Live-updates as new points are added to the dataset.
For example, build tree with a subsample of the full data and then add the rest of the instances to simulate live-updates.

compression

More distance functions. For example, Tanimoto Distance using maximal-common-subgraph for molecular structures.



    \section*{Acknowledgments}
    The authors thank the members of the University of Rhode Island's Algorithms for Big Data research group for their helpful comments throughout the development of this work.
    We are especially grateful to Carl Stoker and Rachel F. Daniels for their thorough reviews of the paper and valuable feedback.

    % \afterpage{\clearpage}
    \FloatBarrier
    \bibliographystyle{IEEEtran}
    \typeout{}
    \bibliography{references}
    % \newpage
    % % SIAM Supplemental File Template
\documentclass[review,supplement,onefignum,onetabnum]{siamonline220329}

% SIAM Shared Information Template
% This is information that is shared between the main document and any
% supplement. If no supplement is required, then this information can
% be included directly in the main document.


% Packages and macros go here
\usepackage{lipsum}
\usepackage{amsfonts}
\usepackage{graphicx}
\usepackage{epstopdf}
\usepackage{algpseudocode}
\ifpdf
  \DeclareGraphicsExtensions{.eps,.pdf,.png,.jpg}
\else
  \DeclareGraphicsExtensions{.eps}
\fi

% Prevent itemized lists from running into the left margin inside theorems and proofs
\usepackage{enumitem}
\setlist[enumerate]{leftmargin=.5in}
\setlist[itemize]{leftmargin=.5in}

% Add a serial/Oxford comma by default.
\newcommand{\creflastconjunction}{, and~}

% Used for creating new theorem and remark environments
\newsiamremark{remark}{Remark}
\newsiamremark{hypothesis}{Hypothesis}
\crefname{hypothesis}{Hypothesis}{Hypotheses}
\newsiamthm{claim}{Claim}

% Sets running headers as well as PDF title and authors
\headers{CAKES: Scalable, Exact Search on Big Data}{M. E. Prior, T. J. Howard III, O. McLaughlin, T. Ferguson, N. Ishaq, N. M. Daniels}

% Title. If the supplement option is on, then "Supplementary Material"
% is automatically inserted before the title.
\title{Let them have CAKES: A Cutting-Edge Algorithm for Scalable, Efficient, and Exact Search on Big Data\thanks{Submitted to the editors DATE.
}}

\author{
    Morgan E. Prior\thanks{
    Department of Mathematics,
    University of Rhode Island,
    Kingston, RI
    (\email{meprior424@gmail.com})}
    \and
    Thomas J. Howard III\thanks{
    Department of Computer Science and Statistics,
    University of Rhode Island,
    Kingston, RI
    (\email{thoward27@uri.edu}, \email{olwmc@gmail.com}, \email{fergusontr@gmail.com}, \email{najib\_ishaq@zoho.com}, \email{noah\_daniels@uri.edu})}
    \and
    Oliver McLaughlin\footnotemark[3]
    \and
    Terrence Ferguson\footnotemark[3]
    \and
    Najib Ishaq\footnotemark[3]
    \and
    Noah M. Daniels\footnotemark[3]
}


\usepackage{amsopn}
\DeclareMathOperator{\diag}{diag}


%%% Local Variables: 
%%% mode:latex
%%% TeX-master: "ex_article"
%%% End: 

\begin{document}
% Optional PDF information
\ifpdf
\hypersetup{
  pdftitle={Supplementary Materials: An Example Article},
  pdfauthor={D. Doe, P. T. Frank, and J. E. Smith}
}
\fi



\externaldocument[][nocite]{paper}

\maketitle


\section{Supplementary Results}

\begin{figure}[ht!]
    \centering
    \includegraphics[width=3.4in]{plots/fashion-mnist-knn-100.png}
    \caption{
        Scaling behavior of algorithms on fashion-mnist with $k=100$. 
    }
    \label{fig:supplement:fashion-mnist-k-100}
\end{figure}

\begin{figure}[ht!]
    \centering
    \includegraphics[width=3.4in]{plots/sift-knn-100.png}
    \caption{
        Scaling behavior of algorithms on sift with $k=100$. 
    }
    \label{fig:supplement:sift-k-100}
\end{figure}

\begin{figure}[ht!]
    \centering
    \includegraphics[width=3.4in]{plots/glove-25-knn-100.png}
    \caption{
        Scaling behavior of algorithms on glove-25 with $k=100$. 
    }
    \label{fig:supplement:glove-25-k-100}
\end{figure}

\begin{figure}[ht!]
    \centering
    \includegraphics[width=3.4in]{plots/random-knn-100.png}
    \caption{
        Scaling behavior of algorithms on random with $k=100$. 
    }
    \label{fig:supplement:random-k-100}
\end{figure}

\begin{figure}[ht!]
    \begin{subfigure}[b]{0.47\textwidth}
    \includegraphics[width=0.95\textwidth]{images/radius/fashion-mnist-60000.png}\\
    \subcaption{Fashion-mnist}
    \label{fig:results:fashion-mnist-radius}
    \end{subfigure}%
    \begin{subfigure}[b]{0.47\textwidth}
    \includegraphics[width=0.95\textwidth]{images/radius/glove-25-1183514.png}\\
    \subcaption{Glove-25}
    \label{fig:results:glove-25-radius}
    \end{subfigure}
    \vspace{1em}
    \\
    \begin{subfigure}[b]{0.47\textwidth}
    \includegraphics[width=0.95\textwidth]{images/radius/sift-1000000.png}\\
    \subcaption{Sift}
    \label{fig:results:sift-radius}
    \end{subfigure}%
    \begin{subfigure}[b]{0.47\textwidth}
    \includegraphics[width=0.95\textwidth]{images/radius/radio-ml-97920.png}\\
    \subcaption{RadioML}
    \label{fig:results:radioml-radius}
    \end{subfigure}%
    \\
    \begin{subfigure}[b]{0.47\textwidth}
    \includegraphics[width=0.95\textwidth]{images/radius/silva-2224640.png}\\
    \subcaption{Silva 18S}
    \label{fig:results:silva-radius}
    \end{subfigure}%  
    \begin{subfigure}[b]{0.47\textwidth}
    \includegraphics[width=0.95\textwidth]{images/radius/random-1000000.png}\\
    \subcaption{A random dataset}
    \label{fig:results:random-radius}
    \end{subfigure}
    \vspace{1em}
    \caption{Radius vs. cluster depth across six datasets, grouped by decile of radius and weighted by the cardinalities of the clusters.
    The last dataset is randomly generated.}
    \label{fig:results:radius-plots}
\end{figure}

\begin{figure}[ht!]
    \begin{subfigure}[b]{0.47\textwidth}
    \includegraphics[width=0.95\textwidth]{images/fractal_density/fashion-mnist-60000.png}\\
    \subcaption{Fashion-mnist}
    \label{fig:results:fashion-mnist-fractal_density}
    \end{subfigure}%
    \begin{subfigure}[b]{0.47\textwidth}
    \includegraphics[width=0.95\textwidth]{images/fractal_density/glove-25-1183514.png}\\
    \subcaption{Glove-25}
    \label{fig:results:glove-25-fractal_density}
    \end{subfigure}
    \vspace{1em}
    \\
    \begin{subfigure}[b]{0.47\textwidth}
    \includegraphics[width=0.95\textwidth]{images/fractal_density/sift-1000000.png}\\
    \subcaption{Sift}
    \label{fig:results:sift-fractal_density}
    \end{subfigure}%
    \begin{subfigure}[b]{0.47\textwidth}
    \includegraphics[width=0.95\textwidth]{images/fractal_density/radio-ml-97920.png}\\
    \subcaption{RadioML}
    \label{fig:results:radioml-fractal_density}
    \end{subfigure}%
    \\
    \begin{subfigure}[b]{0.47\textwidth}
    \includegraphics[width=0.95\textwidth]{images/fractal_density/silva-2224640.png}\\
    \subcaption{Silva 18S}
    \label{fig:results:silva-fractal_density}
    \end{subfigure}%  
    \begin{subfigure}[b]{0.47\textwidth}
    \includegraphics[width=0.95\textwidth]{images/fractal_density/random-1000000.png}\\
    \subcaption{A random dataset}
    \label{fig:results:random-fractal_density}
    \end{subfigure}
    \vspace{1em}
    \caption{Fractal Density vs. cluster depth across six datasets, grouped by decile of fractal density and weighted by the cardinalities of the clusters.
    The last dataset is randomly generated.
    Fractal Density is defined as $\frac{cardinality}{radius^{LFD}}$}
    \label{fig:results:fractal_density-plots}
\end{figure}


\bibliographystyle{siamplain}
\bibliography{references}


\end{document}

\end{document}
