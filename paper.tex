\documentclass[conference,compsoc]{IEEEtran}

\ifCLASSOPTIONcompsoc
  % IEEE Computer Society needs nocompress option
  % requires cite.sty v4.0 or later (November 2003)
  \usepackage[nocompress]{cite}
\else
  % normal IEEE
  \usepackage{cite}
\fi

\usepackage[utf8]{inputenc} % allow utf-8 input
\usepackage[T1]{fontenc}    % use 8-bit T1 fonts
% \usepackage{hyperref}       % hyperlinks
\usepackage{url}            % simple URL typesetting
\usepackage{booktabs}       % professional-quality tables
\usepackage{amsfonts}       % blackboard math symbols
\usepackage{nicefrac}       % compact symbols for 1/2, etc.
\usepackage{microtype}      % microtypography
\usepackage{lipsum}
\usepackage{graphicx}
% \usepackage{subfigure}
\usepackage{amsmath}
\usepackage{amssymb}
\usepackage{amsthm}
\usepackage{mathtools}
\usepackage{algorithm}
\usepackage{algorithmic}
\usepackage{hyperref}
\usepackage{xr}
\usepackage[caption=false,font=footnotesize]{subfig}
\usepackage{placeins}
\usepackage{siunitx}
\sisetup{
  scientific-notation = fixed,   % disable auto sci notation
  fixed-exponent = 0,            % always print plain numbers
  round-mode = places,
  round-precision = 0,           % round to integer
  detect-weight=true,
  detect-family=true,
  group-minimum-digits= 4
}
\usepackage{booktabs}
\usepackage{adjustbox}

\DeclareMathOperator*{\argmax}{arg\,max}
\DeclareMathOperator*{\argmin}{arg\,min}
\DeclareMathOperator{\score}{score}

% \pdfoutput=1
% below is for arxiv
% \documentclass{article}
% \pdfoutput=1
% \usepackage{arxiv}
%%
%% \BibTeX command to typeset BibTeX logo in the docs
% \ifarxiv
% \else
% \ifpdf
% \hypersetup{
%   pdftitle={Let them have CAKES: A Cutting-Edge Algorithm for Scalable, Efficient, and Exact Search on Big Data},
%   pdfauthor={M. E. Prior, T. J. Howard III, O. McLaughlin, T. Ferguson, N. Ishaq, N. M. Daniels}
% }
% \fi
% \fi

\makeatletter
\newcommand{\linebreakand}{%
  \end{@IEEEauthorhalign}
  \hfill\mbox{}\par
  \mbox{}\hfill\begin{@IEEEauthorhalign}
}
\makeatother




% \input{cakes_shared}
\begin{document}
\title{Let them have CAKES: A Cutting-Edge Algorithm for Scalable, Efficient, and Exact Search on Big Data}

\author{
\IEEEauthorblockN{Morgan E. Prior}
\IEEEauthorblockA{Computer Science\\
Tufts University\\
Medford, MA 02155\\
morgan.prior@tufts.edu}
\and
\IEEEauthorblockN{Thomas J. Howard III}
\IEEEauthorblockA{Computer Science and Statistics\\
University of Rhode Island\\
Kingston, RI 02881\\
thoward27@uri.edu}
\and
\IEEEauthorblockN{Oliver McLaughlin}
\IEEEauthorblockA{Computer Science and Statistics\\
University of Rhode Island\\
Kingston, RI 02881\\
olwmcjp@gmail.com}
\linebreakand
\IEEEauthorblockN{Terrence Ferguson}
\IEEEauthorblockA{Computer Science and Statistics\\
University of Rhode Island\\
Kingston, RI 02881\\
fergusontr@gmail.com}
\and
\IEEEauthorblockN{Najib Ishaq}
\IEEEauthorblockA{Computer Science and Statistics\\
University of Rhode Island\\
Kingston, RI 02881\\
najib\_ishaq@zoho.com}
\and
\IEEEauthorblockN{Noah M. Daniels}
\IEEEauthorblockA{Computer Science and Statistics\\
University of Rhode Island\\
Kingston, RI 02881\\
noah\_daniels@uri.edu}}


\IEEEoverridecommandlockouts
\IEEEpubid{\makebox[\columnwidth]{978-1-6654-3902-2/21/\$31.00~\copyright2021 IEEE \hfill} \hspace{\columnsep}\makebox[\columnwidth]{ }}
% make the title area
\maketitle
\IEEEpubidadjcol
% this must go after the closing bracket ] following \twocolumn[ ...


\begin{abstract}
  The Big Data explosion has created demand for efficient, scalable similarity search. While recent work emphasizes \textit{approximate} $k$-NN search, \textit{exact} $k$-NN search has lagged. We present CAKES, three novel algorithms for exact $k$-NN search. CAKES is generic over \textit{any} distance function and scales not with dataset cardinality or embedding dimension but with geometric properties—metric entropy and fractal dimension—yielding large speedups over existing exact methods when the dataset conforms to the manifold hypothesis. We demonstrate this by contrasting performance on a randomly generated dataset and on ANN-Benchmarks datasets under common distances, on a genomic dataset under Levenshtein distance, and on a radio-frequency dataset under Dynamic Time Warping. CAKES exhibits near-constant running time on manifold-structured data as cardinality grows and achieves perfect recall in metric spaces; it also attains significantly higher recall than state-of-the-art $k$-NN search methods when the distance is not a metric. We conclude that CAKES is a highly efficient, scalable solution for exact $k$-NN on Big Data. A Rust implementation is available under the MIT license at \url{https://github.com/URI-ABD/clam}.


\end{abstract}
%
% \keywords{K-NN Search, Manifold Hypothesis, Sub-Linear  Algorithms, Big Data}

% REQUIRED
% \ifarxiv
% \else
% \begin{MSCcodes}
% 68P05, 68P10
% \end{MSCcodes}
% \fi
%%
%% This command processes the author and affiliation and title
%% information and builds the first part of the formatted document.
% \maketitle
\IEEEpeerreviewmaketitle
    \section{Introduction}
\label{sec:introduction}

Researchers are collecting data at an unprecedented rate, with datasets in many fields growing exponentially, and outpacing improvements in computing performance predicted by Moore's Law~\cite{kahn2011future}. This ``Big Data explosion'' has created a need for algorithms that scale efficiently to large datasets.

Examples of such datasets include genomic databases, time-series signals, and neural network embeddings such as GPT~\cite{2020arXiv200514165B, OpenAI2023GPT4TR}, LLAMA-2~\cite{Touvron2023Llama2O}, and image embedding models~\cite{radford2021learning, dosovitskiy2020image}.
Among biological datasets,  SILVA 18S~\cite{10.1093/nar/gks1219} includes ribosomal RNA sequences from approximately 2.25 million genomes, reaching 50,000 letters when aligned.
Among time-series datasets, the RadioML dataset~\cite{oshea2018radioml} contains approximately 2.55 million samples of synthetically-generated radio frequency signals.

Similarity search is a fundamental task on such datasets, enabling applications such as recommendation~\cite{annoy} and classification ~\cite{suyanto2022knnclassifier}.
Yet as datasets grow in size and dimensionality, efficient and accurate similarity search becomes increasingly challenging;
even state-of-the-art algorithms exhibit a steep tradeoff between recall and throughput~\cite{malkov2016hnsw, johnson2019billion, annoy, aumuller2020ann}.

Given some measure of similarity between data points, two standard similarity search paradigms are $k$-nearest neighbor search ($k$-NN) and $\rho$-nearest neighbor search ($\rho$-NN).
The former retrieves the $k$ points closest to a query, while the latter retrieves all points within a similarity threshold $\rho$ of a query.
While prior work has used the term \textit{approximate} search to refer to $\rho$-NN, we use it more narrowly: an {approximate} search algorithm is any algorithm that yields imperfect recall, whereas an \textit{exact} algorithm guarantees perfect recall.

$k$-NN search is one of the most widely-used methods in classification and recommendation~\cite{fix1952discriminatory, cover1967nearest}, but na\"{i}ve implementations are prohibitively slow because due to linear time complexity.
Existing fast algorithms are often approximate~\cite{gao2023high}, which may suffice for some applications, but the need for efficient and \textit{exact} search remains~\cite{ukey2023survey}.
In a majority voting classifier, approximate $k$-NN search may agree with exact $k$-NN search for large values of $k$, but be sensitive to local perturbations for smaller values of $k$.

This paper introduces CAKES (CLAM-Accelerated $K$-NN Entropy-Scaling Search), a set of three novel algorithms for \emph{exact} $k$-NN search.
We benchmark CAKES against state-of-the-art methods--FAISS~\cite{johnson2019billion}, HNSW~\cite{malkov2016hnsw}, and ANNOY~\cite{annoy}--on datasets from the ANN-benchmarks suite~\cite{aumuller2020ann}.
We also evaluate performance on a large genomic dataset, the SILVA 18S dataset~\cite{10.1093/nar/gks1219} using Levenshtein~\cite{levenshtein1966binary} distance on unaligned genomic sequences, and on a radio frequency dataset, RadioML~\cite{oshea2018radioml}, using Dynamic Time Warping (DTW)~\cite{gold2018dynamic} distance on complex-valued time-series.
To further contextualize results, we compare against a synthetic dataset with simple statistical properties.


\subsection{Related Works}
\label{sec:intoduction:related-works}

Several algorithms have been developed to scale $k$-nearest neighbor search to large datasets, including Hierarchical Navigable Small World networks (HNSW)~\cite{malkov2016hnsw}, InVerted File indexing (FAISS-IVF)~\cite{faissivf}, random projection and tree building (ANNOY)~\cite{annoy}, and entropy-scaling search~\cite{yu2015entropy, ishaq2019clustered}. However, many of these algorithms do not support \emph{exact} search (as defined in Section \ref{sec:introduction}).

Hierarchical Navigable Small World networks~\cite{malkov2016hnsw} rely on navigable small world (NSW) networks and skip lists. InVerted File indexing (IVF)~\cite{faissivf, sacks1987multikey, kent1990signature} partitions data into high-dimensional Voronoi cells, and searches exhaustively only within cells near the query. ANNOY~\cite{annoy} uses random projection and tree building.
At each intermediate node of the tree, a random hyperplane, defined by two sampled points, splits the space.  Multiple such trees form a forest; increasing the number of trees improves recall but increases search time.


\subsection{Entropy-Scaling Search}
\label{sec:intoduction:entropy-scaling-search}

Entropy-scaling algorithms exploit the inherent structure of large datasets, achieving complexity that scales with topological properties—such as metric entropy and fractal dimension—rather than cardinality. This makes them especially suitable for manifold-structured data.

In 2019, we introduced CHESS (Clustered Hierarchical Entropy-Scaling Search)~\cite{ishaq2019clustered}, which extended the original flat clustering approach to entropy-scaling $\rho$-NN search from a flat clustering approach~\cite{yu2015entropy} to a hierarchical one.
CLAM (Clustering, Learning and Approximation with Manifolds), originally developed to allow ``manifold mapping'' for anomaly detection~\cite{ishaq2021clustered}, refined the clustering algorithm from CHESS.


This paper introduces CAKES, a set of three entropy-scaling algorithms for $k$-NN search, implemented in Rust. We also provide a theoretical analysis of their time complexity in Sections~\ref{sec:methods:knn-search:repeated-rnn-complexity} and~\ref{sec:methods:knn-search:complexity-of-sieve-methods}.
These analyses are not worst-case analyses in the traditional sense, as they do not assume the worst possible dataset, namely, a uniform distribution.
Given that CAKES's algorithms are intended to be used on datasets with a manifold structure, complexity analysis assuming a uniform distribution of data would be uninformative.
As a result, our analysis assumes datasets with manifold structure—characterized by low fractal dimension and metric entropy—reflecting the conditions for which CAKES is designed.

    \section{Methods}
\label{sec:methods}

In this manuscript, we study $k$-NN search in finite-dimensional spaces.
Given a dataset $\textbf{X} = \{x_1, \dots, x_n\}$, we refer to each $x_i \in \textbf{X}$ as a point.
Examples include neural embeddings, genomic sequences, and time-series.

We define a \textit{distance function} $f : \textbf{X} \times \textbf{X} \mapsto \mathbb{R}^+ \ \cup \ \{0\}$ which, given two points, deterministically returns a finite and non-negative real number.
We require $f$ to be symmetric ($f(x, y) = f(y, x)$) and identity-preserving ($f(x, y) = 0 \iff x = y$). 
If $f$ satisfies the triangle inequality (i.e.,\,$f(x, y) \leq f(x, z) + f(z, y)$), then it is also a \textit{distance metric}.
Common examples include Euclidean, Levenshtein~\cite{levenshtein1966binary}, and DTW~\cite{muller2007dynamic};
Cosine distance violates the triangle inequality and is not a metric.
As in~\cite{yu2015entropy}, all CAKES' algorithms are exact under distance metrics.

% cosine distance is not a metric because it violates the triangle inequality (e.g.,\,consider the points $x = (1, 0)$, $y = (0, 1)$ and $z = (1, 1)$ on the Cartesian plane).

The appropriate distance function depends on the data: Euclidean or Cosine are often used for neural embeddings, Levenshtein or Hamming for sequences, and DTW or Wasserstein for time-series.

CAKES assumes the manifold hypothesis~\cite{fefferman2016testing}, the notion that high($D$)-dimensional data collected from constrained generating phenomena often occupy a low($d$)-dimensional manifold within their embedding space ($d \ll D$).
We define the \emph{local fractal dimension} (LFD) to quantify this property.
For a point $q \in \textbf{X}$ and radii $r_1 > r_2$,

\begin{equation}
    \text{LFD}(q, r_1, r_2) = \frac{\text{log} \left( \frac{|B(q, r_1)|}{|B(q, r_2)|} \right) }{\text{log} \left( \frac{r_1}{r_2} \right) }
    \label{eq:methods:lfd-original}
\end{equation}
where $B(q, r)$ is the set of points in the metric ball of radius $r$ centered at $q$.
We typically use a simplified version of Equation~\ref{eq:methods:lfd-original} with $r_1 = 2 \cdot r_2$:
\begin{equation}
    \text{LFD}(q, r) = \text{log}_2 \left( \frac{|B(q, r)|}{|B(q, \frac{r}{2})|} \right).
    \label{eq:methods:lfd-half}
\end{equation}

% In other words, we assume that the dataset is embedded in a $D$-dimensional space, but that the data only occupy a $d$-dimensional manifold, where $d \ll D$.
% While we sometimes use Euclidean notions to describe the geometric and topological properties of the clusters and manifold, CLAM and CAKES do not rely on such notions;
% they serve merely as convenient and intuitive vocabulary to discuss the underlying mathematics.
% CAKES exploits the low LFD of such datasets to accelerate search.

Intuitively, LFD measures the rate of change in the number of points in a ball of radius $r$ around a point $q$ as $r$ increases.
When the vast majority of points in the dataset have low ($d \ll D$) LFD, we simply say that the dataset has low LFD.
We stress that this concept differs from the \textit{embedding dimension} of a dataset.
To illustrate the difference, consider the SILVA 18S rRNA dataset that contains genomic sequences with unaligned lengths of up to 3,712 bases and aligned length of 50,000 bases.
Hence, the \textit{embedding dimension} of this dataset is at least 3,712 and at most 50,000.
However, physical limitations (namely, biological evolution and biochemistry) constrain the data to a low-dimensional manifold within this space.
LFD is an approximation of the dimensionality of that low-dimensional manifold in the ``vicinity'' of a given point.
% Section~\ref{sec:results:lfd-of-datasets} discusses this concept on a variety of datasets, showing how real datasets uphold the manifold hypothesis.
For real-world datasets, we expect the LFD to be locally uniform, i.e.,\,when $r$ is small, but (potentially) globally variable.


\subsection{Clustering}
\label{sec:methods:clustering}

We define a \emph{cluster} $C$ as a set of points with a \emph{center} and \emph{radius}.
The center is the geometric median—the point minimizing the sum of distances to all other points—of the points in $C$;
in practice, for large $|C|$ we find the geometric median of a random subsample of $\sqrt{|C|}$ points from $C$.
The radius is the maximum distance from the center to any point in $C$.
Each non-leaf cluster has two children (as in a binary tree).
Clusters may have overlapping volumes, but each point in an overlap is assigned to exactly one cluster;
consequently $C(c,r)\subset B(c,r)$.
We denote the cluster tree by $\mathcal{T}$ and its root by $\mathcal{R}$ (Sec.~\ref{sec:methods:clustering:building-the-tree}).

Unless stated otherwise, we estimate the LFD of clusters at scales $r$ and $r/2$ via Eq.~\ref{eq:methods:lfd-half}, using only the points in $C(c,r)$ (not all of $B(c,r)$).

For a flat clustering, \emph{metric entropy} $\mathcal{N}_r(X)$ is the minimum number of radius-$r$ clusters needed to cover $X$~\cite{yu2015entropy}.
In our hierarchical setting, we define $\mathcal{N}_{\hat r}(X)$ as the number of leaf clusters, where $\hat r$ is the mean radius of all leaves.


\subsubsection{Building the Tree}
\label{sec:methods:clustering:building-the-tree}

We construct $\mathcal{T}$ using CLAM, following CHESS~\cite{ishaq2019clustered} but with improved pole selection, then apply a depth-first reordering (Sec.~\ref{sec:methods:clustering:depth-first-reordering}) on the dataset.

The recursive step is as follows.
For a cluster $C$ with $|C|$ points, we draw a random subsample $S\subset C$ of $\sqrt{|C|}$ points, compute all pairwise distances in $S$, and take the geometric median of $S$ as the \emph{center} of $C$.
The point in $C$ farthest from the center is the \emph{left pole}.
The distance from the center to the left pole if the \emph{radius}.
The point in $C$ farthest from the left pole is the \emph{right pole}.
We partition $C$ into \emph{left} and \emph{right child} clusters containing points closer to the left and right poles respectively, with ties going to the left child.

Starting from the root cluster $\mathcal{R}$ containing the entire dataset, we recurse until each leaf contains a single point or some user-specified stopping criterion is met (e.g., minimum cluster radius, maximum tree depth, etc.).
During partitioning, we compute and cache the LFD of each cluster using Eq.~\ref{eq:methods:lfd-half}.

% \begin{algorithm} % enter the algorithm environment
%     \caption{Partition($C$, $criteria$)} % give the algorithm a caption
%     \label{alg:methods:partition} % and a label for \ref{} commands later in the document
%     \begin{algorithmic} % enter the algorithmic environment
%     \REQUIRE $f: X \times X \mapsto \mathbb{R}^+ \cup \{0\}$, a distance function
%     \REQUIRE $C$, a cluster
%     \REQUIRE $criteria$, user-specified continuation criteria

%     \STATE $seeds \Leftarrow$ random sample of $\left\lceil \sqrt{|C|} \right\rceil$ points from $C$
%     \STATE $c \Leftarrow$ geometric median of $seeds$
%     \STATE $l \Leftarrow \argmax f(c, x) \ \forall \ x \in C$
%     \STATE $r \Leftarrow \argmax f(l, x) \ \forall \ x \in C$
%     \STATE $L \Leftarrow \{x \ | \ x \in C \land f(l, x) \le f(r, x)\}$
%     \STATE $R \Leftarrow \{x \ | \ x \in C \land f(r, x) < f(l, x)\}$

%     \IF{$|L| > 1$ \textbf{and} $L$ satisfies $criteria$}
%         \STATE Partition($L$, $criteria$)
%     \ENDIF

%     \IF{$|R| > 1$ \textbf{and} $R$ satisfies $criteria$}
%         \STATE Partition($R$, $criteria$)
%     \ENDIF
% \end{algorithmic}
% \end{algorithm}


\subsubsection{Depth-First Reordering}
\label{sec:methods:clustering:depth-first-reordering}

In CHESS~\cite{ishaq2019clustered}, each cluster stored a list of indices into the dataset.
This list was used to retrieve the clusters' points during search.
Although this approach allowed us to retrieve the points in constant time, its memory cost was prohibitively high.
With a dataset of cardinality $n$ and each cluster storing a list of indices for its points, we stored a total of $n$ indices at each depth in the tree $\mathcal{T}$.
Assuming $\mathcal{T}$ is balanced, and thus $\mathcal{O}(\log n)$ depth, this approach had a memory overhead of $\mathcal{O}(n \log n)$.
In this work, we introduce a new approach wherein, after building $\mathcal{T}$, we reorder the dataset so that points are stored in a depth-first order.
Then, within each cluster, we need only store its \textit{cardinality} and an \textit{offset} to access its points from the dataset.
The root cluster $\mathcal{R}$ has an offset of zero and a cardinality equal to the number of points in the dataset.
A left child has the same offset its parent, and the corresponding right child has an offset equal to the left child's offset plus the left child's cardinality.
With no additional memory cost nor time cost for retrieving points during search, depth-first reordering offers the same time complexity as CHESS but with $\mathcal{O}(n)$ memory overhead.


\subsubsection{Time Complexity}
\label{sec:methods:clustering:time-complexity}

The asymptotic complexity of building $\mathcal{T}$ is the same as described in~\cite{ishaq2019clustered}, i.e., $\mathcal{O}(n \log n)$.
This is a significant improvement over the exact approach, which costs $\mathcal{O}(n^2 \log n)$.

% This complexity analysis ignores the cost of checking the continuation criteria.
% While a user could specify criteria of any complexity (our implementation allows for this), for this paper we continue clustering until each cluster is a singleton, i.e.,\,it contains only one point or duplicates of the same point and has a radius of zero.
% This criterion costs $\mathcal{O}(1)$ to check per cluster, and so does it not affect the overall complexity of the algorithm as used in this paper.


The $\mathcal{O}(n\log n)$ build complexity assumes a balanced tree $\mathcal{T}$. In practice, Algorithm~\ref{alg:methods:partition} yields unbalanced trees on real data because of nonuniform sampling densities and the manifold’s low-dimensional structure. A balanced tree is expected only for uniformly distributed data (e.g., in a $d$-dimensional hypercube).

Although real trees may be deeper than $\log n$, which may appear to increase the cost of tree-building, many leaves appear at shallow depths, so later levels contain fewer points and their per-level build costs decline. A general analysis for unbalanced trees is dataset-dependent and is not provided; instead, we report empirical results based on forcing a balanced tree in Section~\ref{sec:results:clustering-strategies-and-number-of-distance-computations}. In any case, the tree is built once, so this cost is amortized over all queries and subsequent applications.

% When building a tree for a real dataset, the depth would be larger than $\log n$, which would seem to increase the cost of tree-building.
% However, because of the large numbers of leaf clusters at shallow depths, each subsequent level of $\mathcal{T}$ would also have fewer points, and so the cost of building each subsequent level would be lower.
% Any analysis of unbalanced trees would be highly dependent on the specific dataset, and so we do not provide one here.
% We do, however, explore empirical results based on forcing a balanced tree in Section~\ref{sec:results:clustering-strategies-and-number-of-distance-computations}.
% In any case, the tree is built only once, so this cost is amortized over all search queries and other applications.

\begin{figure}[t]
    \vskip -0.2in
    \centering
    \includegraphics[scale=0.5]{images/geometry/deltas.pdf}
    \caption{
        {\color{blue}$\delta$}, {\color{red}$\delta^{+}$}, and {\color{green}$\delta^{-}$} for a cluster $C$ and a query $q$.
        ${\color{blue}\delta} = f(q, c)$ is the distance from the query to the cluster center $c$.
        ${\color{red}\delta^{+}} = \delta + r$ is the distance from the query to the theoretically farthest point in $C$.
        ${\color{green}\delta^{-}} = \text{max}(0, \delta - r)$ is the distance from the query to the theoretically closest point in $C$.
    }
    \label{fig:methods:deltas}
    % \vskip -0.5in
\end{figure}
\FloatBarrier

% \begin{minipage}{.425\textwidth}
\FloatBarrier
    \begin{algorithm}[!t]
        \caption{tree-search($C$, $q$, $r$)}
        \label{alg:methods:rnn-search:tree-search}
        \begin{algorithmic}[1]
            \REQUIRE $f$, a distance function
            \REQUIRE $C$, a cluster
            \REQUIRE $q$, a query
            \REQUIRE $r$, a search radius
            \IF{$\delta^+_C \leq r$}
                \STATE \textbf{return} $\{C\}$
            \ELSE
                \STATE $[L, R]$ $\Leftarrow$ \textit{children} of $C$
                \STATE \textbf{return} tree-search($L, q, r$) \\
                 $\cup$ tree-search($R, q, r$)
            \ENDIF
        \end{algorithmic}
    \end{algorithm}
% \end{minipage}

% \begin{minipage}{.475\textwidth}
    \begin{algorithm}[!t]
        \caption{leaf-search($Q$, $q$, $r$)}
        \label{alg:methods:rnn-search:leaf-search}
        \begin{algorithmic}[1]
            \REQUIRE $f$, a distance function
            \REQUIRE $Q$, a set of clusters
            \REQUIRE $q$, a query
            \REQUIRE $r$, a search radius
            \STATE $H \Leftarrow \emptyset$
            \FOR{$C \in Q$}
                \IF{$\delta^+_C \leq r$}
                    \STATE $H$ $\Leftarrow$ $H \cup C$
                \ELSE
                    \FOR{$p \in C$}
                        \IF{$f(p, q) \leq r$}
                            \STATE $H$ $\Leftarrow$ $H \cup \{p\}$
                        \ENDIF
                    \ENDFOR
                \ENDIF
            \ENDFOR
            \STATE \textbf{return} $H$
        \end{algorithmic}
    \end{algorithm}
% \end{minipage}


\subsection{\texorpdfstring{$k$}{k}-Nearest Neighbors Search}
\label{sec:methods:knn-search}

Given a user-specified integer $k$ and query $q$, $k$-NN search aims to find the $k$ closest points to $q$ in $\textbf{X}$.
In other words, $k$-NN search aims to find the set $H$ such that $|H| = k$ and $H = B(q, \rho_k)$ where $\rho_k = \max \left\{ f(q, p) \ \forall \ p \in H \right\}$ is the distance from $q$ to the $k^{th}$ nearest neighbor in $\textbf{X}$.

In this section, we present three novel algorithms for exact $k$-NN search:
Repeated $\rho$-NN, Breadth-First Sieve, and Depth-First Sieve.
In these algorithms, we use $H$, for \textit{hits}, to refer to the data structure that stores the closest points to the query found so far, and $Q$ to refer to the data structure that stores the clusters and points that are still in contention for being among the $k$ nearest neighbors.
These algorithms also use some terminology defined in Figure~\ref{fig:methods:deltas}.


\subsubsection{Repeated \texorpdfstring{$\rho$}{p}-NN}
\label{sec:methods:knn-search:repeated-rnn}

This algorithm relies on the \textit{tree-search} (Algorithm~\ref{alg:methods:rnn-search:tree-search}) and \textit{leaf-search} (Algorithm~\ref{alg:methods:rnn-search:leaf-search}) as described in~\cite{ishaq2019clustered} and reproduced here for completeness.

For Repeated $\rho$-NN search, we begin by performing \textit{tree-search} with a search radius $r$ equal to the radius of the root $\mathcal{R}$ divided by the cardinality of the dataset.
If no clusters are found, we keep doubling $r$ and repeating tree-search, until we find a a non-empty set, $Q$, of clusters.

Now, so long as $\sum_{C \in Q} |C| < k$, we continue to perform tree-search, but instead of doubling $r$, we multiply it by a factor determined by the LFD in the vicinity of the query ball.
In particular, we increase the radius by a factor of
\begin{equation}
    \min \left(2, \left( {\frac{k}{\sum_{C \in Q} |C|}} \right)^{\mu} \right)
    \label{eq:methods:repeated-rnn-factor}
\end{equation}
where $\mu$ is the multiplicative inverse of the harmonic mean of the LFD of the clusters in $Q$, i.e.,\,$\mu = \frac{1}{|Q|} \cdot \sum_{C \in Q} \big( LFD(C)^{-1} \big)$.
We use the harmonic mean to ensure that $\mu$ is not dominated by outlier clusters with very high LFD.
We cap this factor at 2 to ensure that we do not increase the radius too quickly in any single iteration.

Intuitively, the factor by which we increase the radius should be \textit{inversely} related to the number of points found so far.
When the LFD at the radius scale from the previous iteration is high, this suggests that the data are densely populated in that region.
Thus, a small increase in the radius would likely encounter many more points, so a smaller radial increase would suffice to find $k$ neighbors.
Conversely, when the LFD at the radius scale from the previous iteration is low, this suggests that the data are sparsely populated in that region.
In such a region, a small increase in the radius would likely encounter vacant space, so a larger radial increase is needed.
Thus, the factor of radius increase should also be \textit{inversely} related to the LFD.
However, we should not increase the radius too drastically with any one iteration because we assume that the LFD is only \textit{locally} uniform.
A large increase in the radius would likely break out of the local region and potentially encounter too many new clusters, which would make the subsequent step computationally expensive.

Once $\sum_{C \in Q} |C| \geq k$, we are guaranteed to have found at least $k$ neighbors, and so we perform \textit{leaf-search} to find the $k$ nearest neighbors among the points in the clusters found after the last tree-search.


\subsubsection{Complexity of Repeated \texorpdfstring{$\rho$}{p}-NN}
\label{sec:methods:knn-search:repeated-rnn-complexity}

% \begin{theorem} Let $X$ be a dataset and $q$ a query sampled from the same distribution (i.e., arising from the same generative process) as $X$. Then time complexity of performing Repeated $\rho$-NN search on $X$ with query $q$ is \begin{gather}
%         \mathcal{O}
%         \Bigg(
%             \underbrace{
%                 \log~\overbrace{\mathcal{N}_{\hat{r}}(X)}^{\textrm{metric entropy}}
%             }_{\textrm{tree-search}}
%             \ + \
%             \underbrace{
%                 \overbrace{k}^{\textrm{output size}} \cdot
%                 \overbrace{\bigg( 1 + 2 \cdot \Big( \frac{\hat{|C|}}{k} \Big) ^ {d^{-1}} \bigg)^d}^{\textrm{scaling factor}}
%             }_{\textrm{leaf-search}}
%         \Bigg)
%         \label{eq:methods:repeated-rnn-complexity}
%     \end{gather}
%     where $\mathcal{N}_{\hat{r}}(X)$ is the metric entropy of the dataset, $d$ is the LFD of the dataset, and $k$ is the number of nearest neighbors.
%     \label{thm:methods:rnn-complexity}
% \end{theorem}

We consider the \textit{tree-search} and \textit{leaf-search} stages of search separately.
Tree-search refers to the process of identifying clusters that overlap with the query ball, or in other words, clusters that might contain one of the $k$ nearest neighbors.
In ~\cite{ishaq2019clustered}, we showed that the complexity of $\rho$-NN search is
\begin{gather}
    \mathcal{O}
    \Bigg(
        \underbrace{
            \log~\overbrace{\mathcal{N}_{\hat{r}}(X)}^{\textrm{metric entropy}}
        }_{\textrm{tree-search}}
        +
        \underbrace{
            \overbrace{ \big| B(q, \rho) \big|}^{\textrm{output size}}
            \overbrace{ \left( \frac{\rho + 2 \cdot \hat{r}}{ \rho} \right) ^ {\text{LFD}(q, \rho)}}^{\textrm{scaling factor}}
        }_{\textrm{leaf-search}}
    \Bigg).
    \label{eq:methods:rnn-search-complexity}
\end{gather}

To extend Equation~\ref{eq:methods:rnn-search-complexity} to Repeated $\rho$-NN, we must first estimate the number of iterations of tree-search (Algorithm~\ref{alg:methods:rnn-search:tree-search}) needed to find a radius that guarantees at least $k$ neighbors.
Since $q$ is sampled from the same distribution as $X$, the LFD near $q$ should not differ significantly from the LFDs of clusters near $q$ at the scale of the distance from the query to the $k^{th}$ nearest neighbor.
Thus, Equation~\ref{eq:methods:repeated-rnn-factor} suggests that in the expected case, we need only two iterations of tree-search to find $k$ neighbors:
one iteration to find at least one cluster, and one more to find enough clusters to guarantee $k$ neighbors.
Since this is a constant factor, complexity of tree-search for Repeated $\rho$-NN is the same as that for $\rho$-NN search, i.e.,\,$\mathcal{O}\big(\log\mathcal{N}_{\hat{r}}(X)\big)$.

We proceed to determine the complexity of leaf-search.
Let $Q$ be the set of clusters returned by tree-search.
We must estimate $\sum_{C \in Q} |C|$, the total cardinality of those clusters;
since we must examine every point in each such cluster, time complexity of leaf-search is linear in this quantity.
Let $\rho_k$ be the distance from the query to the $k^{th}$ nearest neighbor.
Then, we see that $Q$ is expected to be the set of clusters that overlap with a ball of radius $\rho_k$ around the query.
We can estimate this region as a ball of radius $\rho_k + 2\hat{r}$, where $\hat{r}$ is the mean radius of the clusters in $Q$.

The work in~\cite{yu2015entropy} showed that, for some constant $\gamma$,
\begin{equation*}
    \sum_{C \in S} |C| \leq \gamma \left| B(q, \rho_k) \right| \left(\frac{\rho_k + 2 \cdot \hat{r}}{\rho_k} \right)^d.
\end{equation*}

By definition of $\rho_k$, we have that $|B(q, \rho_k)| = k$.
Thus, $\sum_{C \in S} |C| \leq \gamma k \left( 1 + 2 \cdot \frac{\hat{r}}{\rho_k} \right)^d$.
It remains to estimate $\rho_k$.

Let $\hat{d}$ be the harmonic mean of the LFDs of the clusters in $Q$.
While ordinarily we compute LFD by comparing cardinalities of two balls with two different radii centered at \textit{the same} point, in order to estimate $\rho_k$, we instead compare the cardinality of a ball \textit{around the query} of radius $\rho_k$ to the mean cardinality, $\hat{|C|}$, of clusters in $Q$ at a radius equal to the mean of their radii, $\hat{r}$.
Since $q$ is from the same distribution as $X$, the LFD at $q$ should not be significantly different from that at the center of a cluster in $Q$.
By Equation~\ref{eq:methods:lfd-original}, we have $\hat{d} = \frac{\log{}\frac{\hat{|C|}}{k}}{\log{}\frac{\hat{r}}{\rho_k}}$, and rearranging, $\frac{\hat{r}}{\rho_k} = \left( \frac{\hat{|C|}}{k} \right)^{\hat{d}^{-1}}$.
Using this to simplify the term for leaf-search in Equation~\ref{eq:methods:rnn-search-complexity}, we get:
\begin{equation*}
    k \Bigg( 1 + 2 \cdot \bigg( \frac{\hat{|C|}}{k} \bigg) ^ {\hat{d}^{-1}} \Bigg)^d
\end{equation*}

Again using the assumption that $q$ is from the same distribution as $X$, the LFD at $q$ should not be significantly different from that at a cluster in $Q$, we have that $\hat{d} \approx d$.
By combining the bounds for tree-search and leaf-search, we see that time complexity of performing Repeated $\rho$-NN search on $X$ with query $q$ is
\begin{gather}
    \mathcal{O}
    \Bigg(
        \underbrace{
            \log~\overbrace{\mathcal{N}_{\hat{r}}(X)}^{\textrm{metric entropy}}
        }_{\textrm{tree-search}}
        \ + \
        \underbrace{
            \overbrace{k}^{\textrm{output size}} \cdot
            \overbrace{\bigg( 1 + 2 \cdot \Big( \frac{\hat{|C|}}{k} \Big) ^ {d^{-1}} \bigg)^d}^{\textrm{scaling factor}}
        }_{\textrm{leaf-search}}
    \Bigg).
    \label{eq:methods:repeated-rnn-complexity}
\end{gather}

We remark that the scaling factor in Equation~\ref{eq:methods:repeated-rnn-complexity} should be close to 1 unless LFD is highly variable in the region around the query (i.e.,\,if $\hat{d}$ differs significantly from $d$).


\subsubsection{Breadth-First Sieve}
\label{sec:methods:knn-search:bredth-first-sieve}

This algorithm (BFS) performs a breadth-first traversal of $\mathcal{T}$, pruning clusters by using a modified version of the QuickSelect algorithm~\cite{hoare1961algorithm} (a subroutine in QuickSort), at each level of $\mathcal{T}$.

We begin by letting $Q$ be a set of 3-tuples $(p, \delta^{+}_{p}, m)$, where $p$ is either a cluster or a point, $\delta^{+}_{p}$ is the $\delta^{+}$ of $p$ as illustrated in Figure~\ref{fig:methods:deltas}, and $m$ is the multiplicity of $p$ in $Q$.
During the breadth-first traversal, for every cluster $C$ we encounter, we add $(C, \delta^{+}_{C}, |C| - 1)$ and $(c, \delta_{c}, 1)$ to $Q$, where $c$ is the center of $C$.
Recall that by the definitions of $\delta$ and $\delta^{+}$ given in Section~\ref{sec:methods:knn-search}, since $c$ is a point, $\delta_{C} = \delta_{c} = \delta^{+}_{c} = \delta^{-}_{c}$.

We then use the QuickSelect algorithm, modified to account for multiplicities and to reorder $Q$ in-place, to find the element in $Q$ with the $k^{th}$ smallest $\delta^{+}$; in other words, we find $\tau$, the smallest $\delta^{+}$ in $Q$ such that $\left| B(q, \tau) \right| \geq k$.
Since this step may require a binary search for the correct pivot element to find $\tau$ and reordering with a new pivot takes linear time in the size of the input list, this version of QuickSelect has $\mathcal{O}\left(|Q| \log |Q|\right)$ time complexity.

Next, we go over all items in $Q$, skipping over any for which $\delta^{-} > \tau$ because such elements cannot contain (or be) one of the $k$ nearest neighbors.
If an item corresponds to a point, we keep it.
If it corresponds to a leaf cluster, we add all its points to $Q$ with a multiplicity of 1 each.
Otherwise (since it must correspond to a non-leaf cluster), we to $Q$ the pairs of 3-tuples corresponding to its child clusters.
We continue this process until the sum of multiplicities in $Q$ is exactly $k$.
We then use the QuickSelect algorithm one last time to reorder $Q$ and return the $k$ nearest neighbors.

\subsubsection{Depth-First Sieve}
\label{sec:methods:knn-search:depth-first-sieve}

This algorithm (DFS) is similar to a depth-first traversal of $\mathcal{T}$.
It uses a max-priority queue to track hits, and a min-priority queue to choose the next branch of $\mathcal{T}$ to explore.

Let $Q$ be a min-queue of clusters prioritized by $\delta^{-}$ and $H$ be a max-queue (with capacity $k$) of points prioritized by $\delta$.
$Q$ starts containing only $\mathcal{R}$ while $H$ starts empty.
So long as $H$ is not full or the top-priority elements from $H$ and $Q$ have $\delta_H \geq \delta_Q$, we repeat following:
\begin{itemize}
    \item While the top-priority element is not a leaf, remove it from $Q$ and add its children to $Q$.
    \item Remove the top-priority element (a leaf) from $Q$ and add all its points to $H$.
\end{itemize}
This process terminates when $H$ is full and the top-priority elements in $H$ and $Q$ are such that $\delta_H < \delta_Q$, i.e.,\,the theoretically closest point left is farther from the query than the $k^{th}$ nearest neighbor found so far.
This leaves $H$ containing exactly the $k$ nearest neighbors to the query.

Note that this algorithm is not a depth-first traversal of $\mathcal{T}$ in the classical sense, because we use $Q$ to prioritize which branch we descend into.
Indeed, we expect this algorithm to often switch the branch being explored.


\subsubsection{Complexity of Sieve Methods}
\label{sec:methods:knn-search:complexity-of-sieve-methods}

Due to their similarity, we combine the complexity analyses of both Sieve methods.
For these methods we again use the terminology of tree-search and leaf-search.
Tree-search navigates the cluster tree and finds $Q$.
Leaf-search exhaustively searches \textit{some} of the clusters in $Q$ to find the $k$ nearest neighbors.

% \begin{theorem}
%     Let $X$ be a dataset and $q$ a query sampled from the same distribution (i.e., arising from the same generative process) as $X$. Let $T \coloneqq \mathcal{O} \big( \lceil d \rceil \cdot \log \mathcal{N}_{\hat{r}}(X) \big)$ and   $L \coloneqq \mathcal{O} \left( k \cdot \bigg( 1 + 2 \cdot \Big( \frac{\hat{|C|}}{k} \Big) ^ {d^{-1}} \bigg)^d \right)$, where $\mathcal{N}_{\hat{r}}(X)$ is the metric entropy of $X$, $d$ is the LFD of $X$, $\hat{|C|}$ is the mean cardinality of clusters overlapping the query ball, and $k$ is the number of nearest neighbors. Then, for dataset $X$ and query $q$, the time complexity of performing Breadth-First Sieve search is \begin{equation}
%         \mathcal{O} \Big( (T + L ) \log (T + L ) \Big)
%         \label{eq:methods:breadth-first-sieve-complexity}
%     \end{equation} and the the time complexity of performing Depth-First Sieve search is \begin{equation}
%         \mathcal{O} \Big( T \log T + L \log k \Big).
%         \label{eq:methods:depth-first-sieve-complexity}
%     \end{equation}
%     \label{thm:methods:sieve-complexity}
% \end{theorem}

Since $q$ is sampled from the same distribution as $X$, the LFD near $q$ should not differ significantly from the LFDs of clusters near $q$ at the scale of the distance from the query to the $k^{th}$ nearest neighbor.
Let $d$ be the LFD in this region, and consider leaf clusters with cardinalities near $k$.
Then the number of leaf-clusters in $Q$ is bounded above by $2d$, where the bound is achieved by having a cluster overlap the query ball at each end of each of $\lceil d \rceil$ mutually-orthogonal axes.
In the worst-case scenario for tree-search, these leaf clusters would all come from different branches of the tree, and so tree-search would look at $2 \cdot \lceil d \rceil \cdot \log \mathcal{N}_{\hat{r}}(X)$ clusters.
Thus, the asymptotic complexity is $T \coloneqq \mathcal{O} \big( \lceil d \rceil \cdot \log \mathcal{N}_{\hat{r}}(X) \big)$.
For leaf-search, the output size and scaling factor are the same as in Repeated $\rho$-NN, and so the asymptotic complexity is $L \coloneqq \mathcal{O} \Big( k \cdot \Big( 1 + 2 \cdot \big( \frac{\hat{|C|}}{k} \big) ^ {d^{-1}} \Big)^d \Big)$.

The asymptotic complexity of BFS is dominated by QuickSelect.
Since QuickSelect is log-linear in the length of $Q$, and $Q$ contains the clusters from tree-search and the points from leaf-search, we see that the asymptotic complexity of BFS is $\mathcal{O} \big( (T + L ) \log (T + L ) \big)$ and that of DFS is $\mathcal{O} \big( T \log T + L \log k \big)$.

\subsection{Auto-Tuning}
\label{sec:methods:auto-tuning}

We perform some simple auto-tuning to select the optimal $k$-NN algorithm to use with a given dataset.
We start by taking the center of every cluster at a low depth (e.g.,\,10) in $\mathcal{T}$ as a query.
This gives us a small, representative sample of the dataset.
Using these clusters' centers as queries, and a user-specified value of $k$, we record the time taken for $k$-NN search on the sample using each of the three algorithms described in Section~\ref{sec:methods:knn-search}.
We select the fastest algorithm over all the queries as the optimal algorithm for that dataset and value of $k$.
Note that even though we select the optimal algorithm based on use with some user-specified value of $k$, we still allow search with any value of $k$.


\subsection{Synthetic Data}
\label{sec:methods:synthetic-data}

Based on the asymptotic analysis, we expect CAKES to perform well on datasets with low LFD and scale sublinearly with cardinality.
To test this, we use datasets from the ANN-Benchmarks suite~\cite{aumuller2020ann} and synthetically augment them to have exponentially larger cardinalities.
We perform the same procedure on a dataset of uniformly distributed points in a hypercube, and compare CAKES against other algorithms on both the original and augmented datasets.

Our augmentation procedure is as follows.
Let $X$ be a dataset of dimension $d$, $\epsilon$ a user-specified noise level, and $m$ an integer multiplier.
For each $\mathbf{x}\in X$, generate $m-1$ new points within distance $\epsilon$ of $\mathbf{x}$ by sampling a random vector $\mathbf{r}\in\mathbb{R}^d$ from the $\epsilon$-ball (hypersphere of radius $\epsilon$ centered at the origin) and setting $\mathbf{x}'=\mathbf{x}+\mathbf{r}$.
Since $\lVert\mathbf{r}\rVert\le\epsilon$, we have $\lVert\mathbf{x}-\mathbf{x}'\rVert\le\epsilon$.
The result is $X'$ with $\lvert X'\rvert=m\,\lvert X\rvert$.
This increases cardinality by a factor of $m$ without altering the dataset’s overall topological structure, isolating the effect of cardinality from dimensionality, metric choice, and topology.

    \section{Datasets And Benchmarks}
\label{sec:datasets-and-benchmarks}

\subsection{ANN-Benchmark Datasets}
\label{sec:datasets-and-benchmarks:ann-benchmark-datasets}

We benchmark on a variety of datasets from the ANN-benchmarks suite~\cite{aumuller2020ann}.
Table~\ref{tab:datasets:summary} summarizes the cardinality and dimensionality of each these datasets.
All benchmarks were conducted on an Intel Xeon E5-2690 v4 CPU @ 2.60GHz with 512GB RAM.
The OS kernel was Manjaro Linux 5.15.164-1-MANJARO.
The Rust compiler was Rust 1.83.0, and the Python interpreter version was 3.9.18.

\begin{table}
    \caption{Datasets used in benchmarks.}
    \label{tab:datasets:summary}
    \begin{center}
        \begin{sc}
            \begin{tabular}{|l|l|l|l|}
                \hline
                \textbf{Dataset} & \textbf{Dist. Function}  &\textbf{Card}  & \textbf{Dim}  \\
                \hline
                Fashion-Mnist    & Euclidean                   & 60,000             & 784                    \\
                \hline
                Glove-25         & Cosine                      & 1,183,514          & 25                     \\
                \hline
                Sift             & Euclidean                   & 1,000,000          & 128                    \\
                \hline
                Random           & Euclidean                   & 1,000,000          & 128                    \\
                \hline
                SILVA            & Levenshtein                 & 2,224,640          & 3,712         \\
                \hline
                RadioML          & Dynamic Time Warping        & 97,920             & 1,024                  \\
                \hline
            \end{tabular}
        \end{sc}
    \end{center}
    \vskip -0.1in
\end{table}

\subsection{Random Datasets and Synthetic Augmentation}
\label{sec:datasets-and-benchmarks:random-datasets}


In addition to ANN-Benchmarks datasets, we evaluate synthetic augmentations generated as in Sec.~\ref{sec:methods:synthetic-data}, using a noise tolerance $\epsilon=0.01$, and study scaling as the cardinality multiplier (“Mult.” in Tables~\ref{tab:results:qps-and-recall-fmn}, \ref{tab:results:qps-and-recall-glove}, \ref{tab:results:qps-and-recall-sift}, \ref{tab:results:qps-and-recall-random}) increases. We also benchmark purely random datasets at various cardinalities; to match Sift we use a random dataset with base cardinality of $1{,}000{,}000$ and dimensionality $128$, and refer to this dataset as “Random.” This benchmark isolates the effect of manifold structure—expected to be absent in purely random data—on CAKES algorithms’ performance.

To compute recall on the augmented datasets, we run linear search in Rust and write the results to disk. We verified on the ANN-Benchmark datasets that this linear search attains perfect recall using the test data provided. We then use the linear-search results on the augmentations as ground truth to evaluate CAKES in Rust and to compute recall for HNSW, ANNOY, and FAISS-IVF in Python.



\subsection{SILVA 18S}
\label{sec:datasets-and-benchmarks:silva-18s}

To demonstrate CAKES with an exotic distance, we benchmark the SILVA 18S ribosomal RNA dataset~\cite{10.1093/nar/gks1219}, which contains rRNA sequences from 2,224,640 genomes; the longest sequence has 3,712 letters. We hold out 1,000 random sequences as queries. We build the tree and perform $k$-NN using Levenshtein distance~\cite{levenshtein1966binary} on unaligned sequences. Although the dataset is provided as a multiple sequence alignment of width 50{,}000, we intentionally use Levenshtein rather than Hamming on the aligned strings to demonstrate CAKES’s flexibility. Under Hamming, the embedding dimension would be 50{,}000; under Levenshtein, we report the dimensionality as the length of the longest sequence (3{,}712).

\subsection{Radio ML}
\label{sec:datasets-and-benchmarks:radio-ml}

We also benchmark CAKES on the Radio-ML dataset~\cite{oshea2018radioml} using Dynamic Time Warping~\cite{muller2007dynamic} as the distance. The dataset spans 24 modulation modes across 26 SNR levels from $-20$ to $30$\,dB, with 4{,}096 samples per combination, totaling $24\!\cdot\!26\!\cdot\!4096=2{,}555{,}504$ samples. Each sample is a 1{,}024-dimensional complex-valued vector (a time series). For experiments, we use the 10dB SNR subset with 97{,}304 samples and a 1{,}000-sample hold-out query set at the same SNR.



\subsection{Local Fractal Dimension}
\label{sec:dayasets:lfd-of-datasets}


Since the time complexity of CAKES algorithms scales with the LFD of the dataset, we examine the LFD of each
dataset we used for benchmarks, as shown in Figure~\ref{fig:results:lfd-plots}. In this section, when we discuss trends in LFD, unless otherwise noted, we are referring to
the 95th percentile of LFD.

In Figure~\ref{fig:results:fashion-mnist-lfd}, we observe that until approximately depth 5, Fashion-MNIST’s LFD is low (i.e., less than 4). 
It then starts increasing, reaching a peak of about 6 near depth 20, before decreasing to 1 at the maximum depth.
Relative to Fashion-MNIST, Glove-25 has low LFD, as shown in Figure~\ref{fig:results:glove-25-lfd}. 
In particular, Glove-25’s LFD is less than 3 for all depths.
Figure~\ref{fig:results:sift-lfd} shows the LFD by depth for Sift, which has higher LFD relative to Fashion-MNIST and Glove-25. 
It increases sharply to a peak of 9 around a depth of 10.
For the Random dataset, which has the same cardinality and dimensionality as Sift, LFD starts at 20 at depth
0 and all percentile lines decrease linearly with depth until reaching the leaves of the tree. The spread in LFD starts very
small for the first few clusters and increases as depth increases. For Silva, as
shown in Figure~\ref{fig:results:silva-lfd}, exhibits consistently low LFD. In particular, LFD is less than 3 for all depths, hovering near 1 for
clusters at depth 40 and deeper. For the Radio-ML dataset, Figure~\ref{fig:results:radioml-lfd} LFD values show three distinct peaks around an LFD
of 12 at or near depths of 8, 25 and 50. Each peak is followed by a linear decrease until encountering a sharp spike for
the next peak. Within each of the three portions, this dataset has a character very similar to that of the Random dataset.

\begin{figure}[t]
    \centering
    % Row 1
    \subfloat[Fashion-MNIST\label{fig:results:fashion-mnist-lfd}]{
      \includegraphics[width=0.48\columnwidth]{images/lfd/fashion-mnist.png}}
    \hfill
    \subfloat[Glove-25\label{fig:results:glove-25-lfd}]{
      \includegraphics[width=0.48\columnwidth]{images/lfd/glove-25.png}}
  
    % Row 2
    \par\medskip
    \subfloat[Sift\label{fig:results:sift-lfd}]{
      \includegraphics[width=0.48\columnwidth]{images/lfd/sift.png}}
    \hfill
    \subfloat[Random dataset\label{fig:results:random-lfd}]{
      \includegraphics[width=0.48\columnwidth]{images/lfd/random.png}}
  
    % Row 3
    \par\medskip
    \subfloat[Silva 18S\label{fig:results:silva-lfd}]{
      \includegraphics[width=0.48\columnwidth]{images/lfd/silva-SSU-Ref.png}}
    \hfill
    \subfloat[RadioML\label{fig:results:radioml-lfd}]{
      \includegraphics[width=0.48\columnwidth]{images/lfd/radio-ml.png}}
  
    % Legend
    \par\smallskip
    \includegraphics[width=0.65\columnwidth]{images/lfd/legend.png}
  
    \caption{Local fractal dimension vs. cluster depth across six datasets. The ‘random’ dataset is generated as in Sec.~\ref{sec:datasets-and-benchmarks:random-datasets}; note the different y-axis. The x-axis is tree depth; the y-axis is LFD at that depth. We plot the 5th, 25th, 50th, 75th, and 95th percentiles, plus min/max. Each cluster is counted by its cardinality so the curves reflect dataset-level distributions.}
    \label{fig:results:lfd-plots}
  \end{figure}


% Since the CAKES algorithms were designed to scale in time with the LFD of the dataset, we examine the LFD of each dataset we used for benchmarks.
% Figure~\ref{fig:results:lfd-plots} illustrates the trends in LFD for Fashion-MNIST, Glove-25, Sift, Random, Silva 18S, and Radio-ML.
% In this section, when we discuss trends in LFD, unless otherwise noted, we are referring to the 95$^{th}$ percentile of LFD.

% For the Fashion-MNIST dataset we observe in Figure~\ref{fig:results:fashion-mnist-lfd} that until approximately depth 5, Fashion-MNIST's LFD is low (i.e., less than 4).
% It then starts increasing, reaching a peak of about 6 near depth 20, before decreasing to 1 at the maximum depth.

% Relative to Fashion-MNIST, Glove-25 has low LFD, as shown in Figure~\ref{fig:results:glove-25-lfd}.
% In particular, Glove-25's LFD is less than 3 for all depths.

% Figure~\ref{fig:results:sift-lfd} shows the LFD by depth for Sift, which has higher LFD relative to Fashion-MNIST and Glove-25.
% It increases sharply to a peak of 9 around a depth of 10, after which it decreases smoothly until reaching the deepest leaves in the tree.

% We contrast the low LFD of Sift with that of the Random dataset which has the same cardinality and dimension. As shown in Figure~\ref{fig:results:random-lfd}, the character of this dataset is very different from the others.
% The LFD starts at 20 at depth 0 and all percentile lines decrease linearly with depth until reaching the leaves of the tree.
% The spread in LFD starts very small for the first few clusters and increases as depth increases.

% The Silva dataset, shown in Figure~\ref{fig:results:silva-lfd}, exhibits consistently low LFD.
% In particular, LFD is less than 3 for all depths, hovering near 1 for clusters at depth 40 and deeper.

% For the Radio-ML dataset, the LFD values show three distinct peaks around an LFD of 12 at or near depths of 8, 25 and 50.
% Each peak is followed by a linear decrease until encountering a sharp spike for the next peak.
% Within each of the tree portions, this dataset has a character similar to that of the Random dataset.
% This suggests that the dataset obeys the manifold hypothesis at some scales, but that it is not ``scale free,'' as the LFD varies significantly by depth.
% This is likely the result of a piecewise uniform sampling strategy used to generate the different modulation modes present in the dataset.

\subsection{Other Algorithms}
\label{sec:datasets-and-benchmarks:other-algorithms}

We benchmarked the three CAKES algorithms against a na\"ive linear search implementation in Rust.
We also benchmarked against three state-of-the-art similarity search algorithms: HNSW, ANNOY, and FAISS-IVF in Python.
We verified that our implementation of linear search produces the same neighbors as provided by the ANN-Benchmarks suite for Fashion-MNIST, Glove-25 and Sift datasets.
We then used this linear search implementation to find and store the ground-truth for the augmented versions of the datasets.
We used this ground-truth to calculate recall for CAKES's algorithms in Rust and for HNSW, ANNOY, and FAISS-IVF in Python.
We plot the results of these benchmarks in Figure~\ref{fig:results:scaling-plots}.

    \section{Results}
\label{sec:results}

% The algorithms presented in this paper rely heavily on the local fractal dimension (LFD) being much smaller than than the embedding dimension of the dataset.
% As such, we begin by examining our assumptions about the LFD of real-world datasets (see Section~\ref{sec:results:lfd-of-datasets} and Figure~\ref{fig:results:lfd-plots}).


For each dataset, we benchmark using the distance function listed in Table~\ref{tab:datasets:summary}, and results in this section are based on k-NN with $k=10$.

In Figure~\ref{fig:results:scaling-plots}, we show how the throughput of each of CAKES's algorithms scales with cardinality for each of the six datasets we examined. To do this, we synthetically augmented the Fashion-MNIST, Glove-25, Sift, and Random datasets to create larger datasets using the process described in Section~\ref{sec:methods:synthetic-data}. For Silva and RadioML, due to the massive sizes of these datasets and challenges in generating plausible augmentations, we took random sub-samples ranging up to the entirety of the dataset---rather than augmented versions of the dataset---to examine how performance scales with cardinality.


Tables~\ref{tab:results:qps-and-recall-fmn},~\ref{tab:results:qps-and-recall-glove},~\ref{tab:results:qps-and-recall-sift} and~\ref{tab:results:qps-and-recall-random} compare the performance (throughput and recall) of the CAKES algorithms against state-of-the-art algorithms on the Fashion-MNIST, Glove-25, Sift and Random datasets. In particular, we examine performance of HNSW, ANNOY and FAISS-IVF on each of those datasets as well as synthetically augmented versions of the datasets to isolate the effect of dataset size on performance. We did not perform similar benchmarks on the Silva and RadioML datasets because HNSW, ANNOY and FAISS support neither the required distance functions nor, in the case of RadioML, complex-valued data. For HNSW, ANNOY, and FAISS-IVF, we allow for a hyper-parameter search to tune their index for maximum recall.
For CAKES, we build the tree and use our auto-tuning approach (see Section~\ref{sec:methods:auto-tuning}) to select the fastest algorithm for each dataset and cardinality.


Though the plots in Figure~\ref{fig:results:scaling-plots} present results for each of CAKES's three algorithms separately, the results in the CAKES column in these tables represent the fastest CAKES algorithm at that dataset and cardinality only.


% Finally, we test our intuitions about unbalanced clustering being better for search
% than balanced clustering (see Section~\ref{sec:results:clustering-strategies-and-number-of-distance-computations} and Figure~\ref{fig:results:distance-counts}).


% \subsection{Local Fractal Dimension of Datasets}
% \label{sec:results:lfd-of-datasets}

% Since the time complexity of CAKES algorithms scales with the LFD of the dataset, we examine the LFD of each dataset we used for benchmarks.
% Figure~\ref{fig:results:lfd-plots} illustrates the trends in LFD for Fashion-MNIST, Glove-25, Sift, Random, Silva 18S, and Radio-ML.
% In this section, when we discuss trends in LFD, unless otherwise noted, we are referring to the 95$^{th}$ percentile of LFD.
% This is because our algorithms scale exponentially in the LFD, a

% The Fashion-MNIST dataset has an embedding dimension of 784 and uses the Euclidean distance metric.
% In Figure~\ref{fig:results:fashion-mnist-lfd} we observe that until approximately depth 5, Fashion-MNIST's LFD is low (i.e., less than 4).
% It then starts increasing, reaching a peak of about 6 near depth 20, before decreasing to 1 at the maximum depth.

% The Glove-25 dataset has an embedding dimension of 25 and uses the cosine distance function which, notably, is not a metric.
% Relative to Fashion-MNIST, Glove-25 has low LFD, as shown in Figure~\ref{fig:results:glove-25-lfd}.
% All percentile lines for Glove-25 are flatter and lower, indicating that the LFD is lower across the entire dataset, and that the LFD does not vary as much by depth.
% In particular, Glove-25's LFD is less than 3 for all depths.

% The Sift dataset has an embedding dimension of 128 and uses the Euclidean distance metric.
% Figure~\ref{fig:results:sift-lfd} shows the LFD by depth for Sift, which has higher LFD relative to Fashion-MNIST and Glove-25.
% It increases sharply to a peak of 9 around a depth of 10.
% It then decreases smoothly until reaching the deepest leaves in the tree.

% We generated the Random dataset to have the same cardinality and dimensionality as Sift.
% We used a uniform distribution in a 128-dimensional unit-hypercube to generate the points and the Euclidean metric to measure distances among them.
% Figure~\ref{fig:results:random-lfd} shows that the character of this dataset is significantly different from the others.
% The LFD starts at 20 at depth 0 and all percentile lines decrease linearly with depth until reaching the leaves of the tree.
% The spread in LFD starts very small for the first few clusters and increases as depth increases.
% The LFD of approximately 20 for the root cluster $\mathcal{R}$ is what we expect for this random dataset.
% To elaborate, the distribution of points in such a dataset should reflect the curse of dimensionality, i.e.,\,the fact that in high dimensional spaces, the minimum and maximum pairwise distances between any two points are approximately equal.
% As a result, $\mathcal{R}$'s radius $r$, which reflects the maximum distance between the center $c$ and any other point, should not differ significantly from the distance between the center and its closest point.
% A consequence of this is that, with high probability, for every point in $\mathcal{R}$, its distance from $c$ is greater than $\tfrac{r}{2}$;
% in other words, $B(c, \tfrac{r}{2})$ contains only $c$ while $B(c, r)$ contains the entire dataset.
% Given our definition of LFD in Equation~\ref{eq:methods:lfd-half}, this means that the LFD of $\mathcal{R}$ is approximately $\log_2(\frac{|X|}{1}) = \log_2(1,000,000) \approx 20$, which is what we observe in Figure~\ref{fig:results:random-lfd}.
% Theoretically, the LFD of this dataset should be 128, i.e.\, it should be the same as the embedding dimension.
% This reflects the difference between how we empirically measure the LFD and the value we would expect.
% With sample sizes larger than 1,000,000, we would expect the LFD to approach 128 until we have sampled $2^{128}$ points, at which point the LFD would be 128 and would stay at 128 for even larger sample sizes.
% Unfortunately, such a large sample is practically impossible to generate.

% The Silva-18S dataset consists of genomic sequences whose unaligned lengths are at-most 3,712.
% As such the embedding dimension is 3,712, though as discussed previously, the embedding dimension would be 50,000 in a multiple sequence alignment.
% We use the Levenshtein edit distance (a metric) to measure distances between sequences.
% This dataset, as shown in Figure~\ref{fig:results:silva-lfd}, exhibits consistently low LFD.
% In particular, LFD is less than 3 for all depths, hovering near 1 for clusters at depth 40 and deeper.

% The Radio-ML dataset consists of measurements of radio-frequency signals using 1,024 dimensional complex-valued vectors.
% We use the Dynamic Time Warping distance metric on this dataset.
% This dataset is synthetic~\cite{oshea2018radioml} but uses a far more elaborate generation process than our Random dataset.
% The LFD values show three distinct peaks around an LFD of 12 at or near depths of 8, 25 and 50.
% Each peak is followed by a linear decrease until encountering a sharp spike for the next peak.
% Within each of the tree portions, this dataset has a character very similar to that of the Random dataset.
% This suggests that the dataset obeys the manifold hypothesis at some scales, but that it is not ``scale free,'' as the LFD varies significantly by depth.
% This is likely the result of a piecewise uniform sampling strategy used to generate the different modulation modes present in the dataset.

% \begin{figure}
%     \captionsetup[subfigure]{aboveskip=-15pt,belowskip=-3pt}
%     \begin{subfigure}[b]{0.5\textwidth}
%         \includegraphics[width=0.99\textwidth]{images/lfd/fashion-mnist.png}\\
%         \subcaption{Fashion-MNIST}
%         \label{fig:results:fashion-mnist-lfd}
%     \end{subfigure}%
%     \begin{subfigure}[b]{0.5\textwidth}
%         \includegraphics[width=0.99\textwidth]{images/lfd/glove-25.png}\\
%         \subcaption{Glove-25}
%         \label{fig:results:glove-25-lfd}
%     \end{subfigure}
%     \\
%     \begin{subfigure}[b]{0.5\textwidth}
%         \includegraphics[width=0.99\textwidth]{images/lfd/sift.png}\\
%         \subcaption{Sift}
%         \label{fig:results:sift-lfd}
%     \end{subfigure}%
%     \begin{subfigure}[b]{0.5\textwidth}
%         \includegraphics[width=0.99\textwidth]{images/lfd/random.png}\\
%         \subcaption{A random dataset}
%         \label{fig:results:random-lfd}
%     \end{subfigure}
%     \\
%     \begin{subfigure}[b]{0.5\textwidth}
%         \includegraphics[width=0.99\textwidth]{images/lfd/silva-SSU-Ref.png}\\
%         \subcaption{Silva 18S}
%         \label{fig:results:silva-lfd}
%     \end{subfigure}%
%     \begin{subfigure}[b]{0.5\textwidth}
%         \includegraphics[width=0.99\textwidth]{images/lfd/radio-ml.png}\\
%         \subcaption{RadioML}
%         \label{fig:results:radioml-lfd}
%     \end{subfigure}%
%     \\
%     \vskip -0.1in
%     \begin{subfigure}[b]{0.94\textwidth}
%         \centering
%         \includegraphics[width=0.7\textwidth]{images/lfd/legend.png}
%         \label{fig:results:lfd-legend}
%     \end{subfigure}%
%     \vskip -0.1in
%     \caption{Local fractal dimension vs. cluster depth across six datasets. The `random' dataset is randomly generated according to the procedure in Section~\ref{sec:datasets-and-benchmarks:random-datasets}; note that the y-axis is different for this dataset. In each plot, the horizontal axis denotes depth in the cluster tree, and the vertical axis denotes the LFD of clusters at that depth. We show lines for the 5$^{th}$, 25$^{th}$, 50th, 75$^{th}$ and 95$^{th}$ percentiles of LFD, as well as the minimum and maximum LFD at each depth. So that plots best reflect the distribution of LFDs across the entire \textit{dataset}, we count each cluster as many times as its cardinality. For example, if, for some dataset, the 95$^{th}$ percentile of LFD at depth 40 is 3, this means that 95\% of the points in clusters at depth 40 belong to a cluster whose LFD is at most 3.}
%     \label{fig:results:lfd-plots}
%     \vskip -0.4in
% \end{figure}


% \subsection{Indexing and Tuning Time}
% \label{sec:results:indexing-and-tuning-time}

% For each of the ANN-benchmark datasets and the Random dataset, we report the time taken for each algorithm to build the index and to tune the hyper-parameters for these indices to achieve the highest possible recall. For the sake of brevity, these results are reported in the supplement.


% We benchmark the CAKES algorithms, na\"{i}ve linear search, HNSW, ANNOY, and FAISS-IVF on the Fashion-MNIST, Glove-25, Sift, and Random datasets.
% We augment these datasets with synthetic points to examine how performance scales with cardinality.
% We also benchmark the CAKES algorithms on the Silva and RadioML datasets, and we subsample these datasets instead of augmenting them to examine how performance scales with cardinality.

% On the Fashion-MNIST, Glove-25, and Sift datasets (in Figures~\ref{fig:results:fashion-mnist-scaling},~\ref{fig:results:glove-25-scaling},~and~\ref{fig:results:sift-scaling} respectively), we observe that as cardinality increases, the Depth-First Sieve algorithm is consistently the fastest CAKES algorithm with a throughput that is constant in the cardinality of the dataset.
% The Breadth-First Sieve algorithm is the usually the second fastest, also with a nearly constant throughput across all cardinalities.
% The Repeated $\rho$-NN algorithm, however, falls off in throughput as cardinality increases.
% All three CAKES algorithms exhibit perfect recall on the Fashion-MNIST and Sift datasets, and near-perfect recall on the Glove-25 dataset (which uses cosine distance).
% In contrast, HNSW and ANNOY are faster than CAKES's algorithms for all cardinalities and have near constant throughput as cardinality increases, but their recall degrades quickly as cardinality increases.
% FAISS-IVF exhibits linearly decreasing throughput as cardinality increases, which is expected from the algorithm given that we tune the hyper-parameters to maximize recall.

% On the Random dataset, HNSW and ANNOY are still the fastest algorithms but exhibit recall values near 0.
% CAKES's algorithms show linearly decreasing throughput as cardinality increases and are also slower than na\"{i}ve linear search.



\begin{figure}[h]
    \centering
    \subfloat[Fashion-MNIST for $k=10$.]{
      \includegraphics[width=0.48\columnwidth]{plots/fashion-mnist_PermutedBall_10_throughput.png}
      \label{fig:results:fashion-mnist-scaling}
    }
    \subfloat[Glove-25 for $k=10$.]{
      \includegraphics[width=0.48\columnwidth]{plots/glove-25_PermutedBall_10_throughput.png}
      \label{fig:results:glove-25-scaling}
    }
  
    \subfloat[Sift for $k=10$.]{
      \includegraphics[width=0.48\columnwidth]{plots/sift_PermutedBall_10_throughput.png}
      \label{fig:results:sift-scaling}
    }
    \subfloat[Random dataset for $k=10$.]{
      \includegraphics[width=0.48\columnwidth]{plots/random_PermutedBall_10_throughput.png}
      \label{fig:results:random-scaling}
    }
  
    \subfloat[Silva for $k=10$.]{
      \includegraphics[width=0.48\columnwidth]{plots/silva-SSU-Ref_PermutedBall_10_throughput.png}
      \label{fig:results:silva-scaling}
    }
    \subfloat[RadioML for $k=10$ at SnR = 10dB.]{
      \includegraphics[width=0.48\columnwidth]{plots/radio-ml_Ball_10_throughput.png}
      \label{fig:results:radioml-scaling}
    }
  
    \vspace{2pt}
    \includegraphics[width=1\columnwidth]{plots/legend.png}
  
    \caption{Throughput across six datasets, including a randomly-generated dataset.
    Each plot shows throughput (queries per second; higher is better) versus dataset cardinality. For Fashion-MNIST, Glove-25, and Sift, CAKES becomes faster than linear search as cardinality grows, with the crossover point varying by dataset. For Fashion-MNIST and Glove-25, Depth-First Sieve is consistently fastest. For Sift, Repeated $\rho$-NN is fastest at small cardinalities, while Depth-First Sieve is fastest at large cardinalities. For Silva, throughput for all algorithms initially decreases approximately linearly with cardinality and then levels off at higher cardinalities; Depth-First Sieve is consistently fastest. For Radio-ML and Random, all CAKES variants are slower than na"{i}ve linear search, and their throughput decreases linearly with cardinality. HNSW and ANNOY are the fastest algorithms on all four datasets we benchmarked them on, but their recall degrades quickly as cardinality increases on all datasets; on Random, their recall is near zero.}
    \label{fig:results:scaling-plots}
  \end{figure}


\textbf{General notes (applies to Tables \ref{tab:results:qps-and-recall-fmn},\ref{tab:results:qps-and-recall-glove}, \ref{tab:results:qps-and-recall-sift}).}
  Throughput is measured in queries per second. A recall value of $1.000*$ denotes imperfect recall that rounds to $1.000$. In each table, we observe that while recall for CAKES does \emph{not} degrade with cardinality, recall for
  HNSW and ANNOY degrades with cardinality. CAKES exhibits perfect recall on the Fashion-MNIST and Sift datasets, and near-perfect recall on the Glove-25 dataset (which uses cosine distance). 

\begin{table}[t]
    \centering
    \caption{Fashion-MNIST: throughput and recall. See general notes above.}
    \label{tab:results:qps-and-recall-fmn}
    \small
    \setlength{\tabcolsep}{4pt}
    \begin{adjustbox}{width=\columnwidth,center}
    \begin{tabular}{@{} lcccccccc @{}}
    \toprule
    \textbf{Mult.} &
    \multicolumn{2}{c}{\textbf{HNSW}} &
    \multicolumn{2}{c}{\textbf{ANNOY}} &
    \multicolumn{2}{c}{\textbf{FAISS-IVF}} &
    \multicolumn{2}{c}{\textbf{CAKES}} \\
    \cmidrule(lr){2-3}\cmidrule(lr){4-5}\cmidrule(lr){6-7}\cmidrule(lr){8-9}
    & QPS & Recall & QPS & Recall & QPS & Recall & QPS & Recall \\
    \midrule
    1   & \num{1.33e4} & 0.954 & \num{2.19e3} & 0.950 & \num{2.01e3} & $1.000^{*}$ & \num{3.46e3} & 1.000 \\
    2   & \num{1.38e4} & 0.803 & \num{2.12e3} & 0.927 & \num{9.39e2} & $1.000^{*}$ & \num{3.68e3} & 1.000 \\
    4   & \num{1.66e4} & 0.681 & \num{2.04e3} & 0.898 & \num{4.61e2} & 0.997       & \num{3.44e3} & 1.000 \\
    8   & \num{1.68e4} & 0.525 & \num{1.93e3} & 0.857 & \num{2.26e2} & 0.995       & \num{3.30e3} & 1.000 \\
    16  & \num{1.87e4} & 0.494 & \num{1.84e3} & 0.862 & \num{1.17e2} & 0.991       & \num{3.34e3} & 1.000 \\
    32  & \num{1.56e4} & 0.542 & \num{1.85e3} & 0.775 & \num{5.91e1} & 0.985       & \num{2.96e3} & 1.000 \\
    64  & \num{1.50e4} & 0.378 & \num{1.78e3} & 0.677 & \num{2.61e1} & 0.968       & \num{3.25e3} & 1.000 \\
    128 & \num{1.49e4} & 0.357 & \num{1.66e3} & 0.538 & \num{1.33e1} & 0.964       & \num{2.96e3} & 1.000 \\
    256 & --           & --    & \num{1.60e3} & 0.592 & \num{6.65e0} & 0.962       & \num{2.79e3} & 1.000 \\
    512 & --           & --    & \num{1.83e3} & 0.581 & \num{3.56e0} & 0.949       & \num{2.84e3} & 1.000 \\
    \bottomrule
    \end{tabular}
    \end{adjustbox}
    \end{table}

\begin{table}
    \centering
    \caption{Glove-25: throughput and recall. See general notes above.}
    \label{tab:results:qps-and-recall-glove}
    \small
    \setlength{\tabcolsep}{4pt}
    \begin{adjustbox}{width=\columnwidth,center}
    \begin{tabular}{@{} lcccccccc @{}}
        \toprule
        \textbf{Mult.} &
        \multicolumn{2}{c}{\textbf{HNSW}} &
        \multicolumn{2}{c}{\textbf{ANNOY}} &
        \multicolumn{2}{c}{\textbf{FAISS-IVF}} &
        \multicolumn{2}{c}{\textbf{CAKES}} \\
        \cmidrule(lr){2-3}\cmidrule(lr){4-5}\cmidrule(lr){6-7}\cmidrule(lr){8-9}
        & QPS & Recall & QPS & Recall & QPS & Recall & QPS & Recall \\
        \midrule
        1   & \num{2.28e4} & 0.801 & \num{2.83e3} & 0.835 & \num{2.38e3} & 1.000* & \num{1.54e3} & 1.000* \\
        2   & \num{2.38e4} & 0.607 & \num{2.70e3} & 0.832 & \num{1.19e3} & 1.000* & \num{1.49e3} & 1.000* \\
        4   & \num{2.50e4} & 0.443 & \num{2.61e3} & 0.839 & \num{6.19e2} & 1.000* & \num{1.28e3} & 1.000* \\
        8   & \num{2.78e4} & 0.294 & \num{2.51e3} & 0.834 & \num{3.03e2} & 1.000* & \num{1.30e3} & 1.000* \\
        16  & \num{3.11e4} & 0.213 & \num{2.23e3} & 0.885 & \num{1.51e2} & 1.000* & \num{1.14e3} & 1.000* \\
        32  & \num{3.24e4} & 0.178 & \num{2.01e3} & 0.764 & \num{7.40e1} & 0.999  & \num{1.05e3} & 1.000* \\
        64  & --           & --    & \num{1.99e3} & 0.631 & \num{3.77e1} & 0.997  & \num{1.07e3} & 1.000* \\
        128 & --           & --    & --           & --    & \num{1.90e1} & 0.998  & \num{8.92e2} & 1.000* \\
        256 & --           & --    & --           & --    & \num{9.47e0} & 0.998  & \num{8.91e2} & 1.000* \\
        \bottomrule
    \end{tabular}
    \end{adjustbox}
    \end{table}


\begin{table}
    \centering
    \caption{Sift: throughput and recall. See general notes above.}
    \label{tab:results:qps-and-recall-sift}
        \small
        \setlength{\tabcolsep}{4pt}
        \begin{adjustbox}{width=\columnwidth,center}
        \begin{tabular}{@{} lcccccccc @{}}
        \toprule
        \textbf{Mult.} &
        \multicolumn{2}{c}{\textbf{HNSW}} &
        \multicolumn{2}{c}{\textbf{ANNOY}} &
        \multicolumn{2}{c}{\textbf{FAISS-IVF}} &
        \multicolumn{2}{c}{\textbf{CAKES}} \\
        \cmidrule(lr){2-3}\cmidrule(lr){4-5}\cmidrule(lr){6-7}\cmidrule(lr){8-9}
        & QPS & Recall & QPS & Recall & QPS & Recall & QPS & Recall \\
        \midrule
        1   & \num{1.93e4} & 0.782 & \num{3.98e3} & 0.686 & \num{6.98e2} & 1.000* & \num{6.20e2} & 1.000 \\
        2   & \num{2.03e4} & 0.552 & \num{3.80e3} & 0.614 & \num{3.30e2} & 1.000* & \num{2.95e2} & 1.000 \\
        4   & \num{2.18e4} & 0.394 & \num{3.69e3} & 0.637 & \num{1.65e2} & 1.000* & \num{1.76e2} & 1.000 \\
        8   & \num{2.48e4} & 0.298 & \num{3.58e3} & 0.710 & \num{7.72e1} & 1.000* & \num{1.27e2} & 1.000 \\
        16  & \num{2.68e4} & 0.210 & \num{3.50e3} & 0.690 & \num{3.98e1} & 1.000* & \num{1.47e2} & 1.000 \\
        32  & \num{2.75e4} & 0.193 & \num{3.44e3} & 0.639 & \num{2.09e1} & 0.999  & \num{1.24e2} & 1.000 \\
        64  & --           & --    & \num{3.39e3} & 0.678 & \num{8.87e0} & 0.997  & \num{1.34e2} & 1.000 \\
        128 & --           & --    & \num{3.36e3} & 0.643 & \num{4.78e0} & 0.993  & \num{1.31e2} & 1.000 \\
        \bottomrule
    \end{tabular}
    \end{adjustbox}
    \end{table}

\begin{table}
    \caption{Random dataset: throughput and recall.
    In contrast with the results on the ANN Benchmark datasets reported above, with the Random dataset, we observe that CAKES's algorithms perform quite slowly.CAKES exhibits perfect recall at all cardinalities, whereas HNSW and ANNOY exhibit \textit{much} lower recall on this random dataset than on any of the ANN benchmark datasets.
    }
    \label{tab:results:qps-and-recall-random}
    \small
    \setlength{\tabcolsep}{4pt}
    \begin{adjustbox}{width=\columnwidth,center}
    \begin{tabular}{@{} lcccccccc @{}}
    \toprule
    \textbf{Mult.} &
    \multicolumn{2}{c}{\textbf{HNSW}} &
    \multicolumn{2}{c}{\textbf{ANNOY}} &
    \multicolumn{2}{c}{\textbf{FAISS-IVF}} &
    \multicolumn{2}{c}{\textbf{CAKES}} \\
    \cmidrule(lr){2-3}\cmidrule(lr){4-5}\cmidrule(lr){6-7}\cmidrule(lr){8-9}
        & QPS & Recall & QPS & Recall & QPS & Recall & QPS & Recall \\
        \midrule
        1  & \num{1.17e4} & 0.060 & \num{4.28e3} & 0.028 & \num{7.342} & 1.000* & \num{6.06e2} & 1.000 \\
        2  & \num{1.01e4} & 0.048 & \num{4.04e3} & 0.021 & \num{3.582} & 1.000* & \num{2.75e2} & 1.000 \\
        4  & \num{9.12e3} & 0.031 & \num{3.64e3} & 0.014 & \num{1.902} & 1.000* & \num{1.35e2} & 1.000 \\
        8  & \num{8.35e3} & 0.022 & \num{3.37e3} & 0.013 & \num{8.841} & 1.000* & \num{6.13e1} & 1.000 \\
        16 & \num{8.25e3} & 0.008 & \num{3.17e3} & 0.006 & \num{4.361} & 1.000* & \num{2.82e1} & 1.000 \\
        32 & --           & --    & \num{3.01e3} & 0.007 & \num{1.721} & 1.000* & \num{1.31e1} & 1.000 \\
        \bottomrule
    \end{tabular}
    \end{adjustbox}
    \end{table}





% \begin{figure}
%     \begin{subfigure}[b]{0.5\textwidth}
%         \includegraphics[width=0.99\textwidth]{images/distance_counts/fashion-mnist_KnnRepeatedRnn_10_throughput.png}
%         \subcaption{Repeated $\rho$-NN}
%         \label{fig:results:fashion-mnist-counts-throughput}
%     \end{subfigure}%
%     \begin{subfigure}[b]{0.5\textwidth}
%         \includegraphics[width=0.99\textwidth]{images/distance_counts/fashion-mnist_KnnRepeatedRnn_10_counts.png}
%         \subcaption{Repeated $\rho$-NN}
%         \label{fig:results:glove-25-counts-counts}
%     \end{subfigure}%
%     \\
%     \begin{subfigure}[b]{0.5\textwidth}
%         \includegraphics[width=0.99\textwidth]{images/distance_counts/fashion-mnist_KnnBreadthFirst_10_throughput.png}
%         \subcaption{Breadth First Sieve}
%         \label{fig:results:sift-counts-throughput}
%     \end{subfigure}%
%     \begin{subfigure}[b]{0.5\textwidth}
%         \includegraphics[width=0.99\textwidth]{images/distance_counts/fashion-mnist_KnnBreadthFirst_10_counts.png}
%         \subcaption{Breadth First Sieve}
%         \label{fig:results:random-counts-counts}
%     \end{subfigure}%
%     \\
%     \begin{subfigure}[b]{0.5\textwidth}
%         \includegraphics[width=0.99\textwidth]{images/distance_counts/fashion-mnist_KnnDepthFirst_10_throughput.png}
%         \subcaption{Depth First Sieve}
%         \label{fig:results:silva-counts-throughput}
%     \end{subfigure}%
%     \begin{subfigure}[b]{0.5\textwidth}
%         \includegraphics[width=0.99\textwidth]{images/distance_counts/fashion-mnist_KnnDepthFirst_10_counts.png}
%         \subcaption{Depth First Sieve}
%         \label{fig:results:radioml-counts-counts}
%     \end{subfigure}%
%     \\
%     \begin{subfigure}[b]{0.94\textwidth}
%         \centering
%         \includegraphics[width=0.6\textwidth]{images/distance_counts/legend.png}
%         \label{fig:results:counts-legend}
%     \end{subfigure}%
%     \caption{Number of distance computations across four clustering strategies and three search algorithms on the Fashion-MNIST dataset.
%     Adding the instrumentation to count the number of distance computations had the side-effect of significantly slowing down the search algorithms compared to those reported in Figure~\ref{fig:results:scaling-plots}.
%     The left column shows the throughput in queries per second, while the right column shows the mean number of distance computations per query.
%     The x-axis represents increasing cardinality of the dataset.}
%     \label{fig:results:distance-counts}
% \end{figure}

    \section{Discussion and Future Work}
\label{sec:discussion-and-future-work}

We have presented CAKES, a suite of three algorithms for fast $k$-NN search across diverse distance functions. When the distance is a metric (Sec.~\ref{sec:methods}), the algorithms are exact; under cosine distance (non-metric) they achieve nearly perfect recall. CAKES is most effective when data satisfy the manifold hypothesis—lying on a low-dimensional manifold within a high-dimensional space. It builds an unbalanced-clustering binary tree that captures manifold structure and accelerates search. Conversely, performance degrades on randomly distributed data, where such structure is absent.


\subsection{Performance}

As expected, Fig.~\ref{fig:results:scaling-plots} shows CAKES scaling sublinearly with cardinality on real-world datasets with low LFD and linearly on synthetic datasets with high LFD; Sift versus Random illustrates this contrast.

On Random, CAKES throughput decreases linearly with the multiplier, whereas on Sift, Depth-First Sieve and Breadth-First Sieve are nearly constant. Repeated $\rho$-NN is not near-constant on Sift but decays much more slowly than on Random. Because Sift and Random share cardinality and embedding dimension, this isolates the effect of manifold structure on performance.

Although CAKES is slower on Random, its recall remains perfect. HNSW and ANNOY maintain near-constant throughput as the multiplier grows, but their recall declines, and their index build time and memory rise, making them unsuitable when datasets grow exponentially. CAKES’s tree builds orders of magnitude faster with far less memory; while queries are slower than HNSW/ANNOY, CAKES attains near-constant scaling with perfect recall.

The fastest CAKES variant depends on dataset and cardinality. On Fashion-MNIST, Glove-25, and Sift, Depth-First Sieve and Breadth-First Sieve show nearly constant throughput as cardinality increases and consistently outperform linear search at high cardinalities. This variability motivates auto-tuning (Sec.~\ref{sec:methods:auto-tuning}); future work will better characterize when each algorithm excels and enable more sophisticated selection from dataset properties.

We stress that CAKES targets \emph{big} data: linear search is better at small sizes, but for each ANN-Benchmarks dataset tested, CAKES overtakes linear search at sufficiently large cardinality—about $10^{5}$ for Fashion-MNIST and $10^{6}$ for Glove-25. On the high-LFD Random dataset, CAKES never beats linear search. These results support that CAKES performs well on data arising from constrained generating phenomena even as cardinality grows exponentially.



% When written with the same notation as used in Section~\ref{sec:methods:knn-search:complexity-of-sieve-methods}, we see that the time complexity of Repeated $\rho$-NN is $\mathcal{O}(\tfrac{T}{\lceil d \rceil} + L)$, where $T$ is the time complexity of tree-search and $L$ is the time complexity of leaf-search.
% Even though Repeated $\rho$-NN has the least time cost (compared to $\mathcal{O}\left((T + L)\log{(T+L)}\right)$ for Breadth-First Sieve and $\mathcal{O}(T\log{T} + L\log{k})$ for Depth-First Sieve) of the three CAKES algorithms, it is not always the fastest algorithm empirically.
% We believe that some of this discrepancy can be explained by the fact that Repeated $\rho$-NN can significantly ``overshoots'' the correct radius for $k$ hits ($\rho_k$) during tree-search.
% This causes leaf-search to exhaustively search many more clusters than necessary, rendering the true scaling factor higher than that in Equation~\ref{eq:methods:repeated-rnn-complexity}.
% This overshot can occur when the LFDs of clusters near the query are not concentrated around their expectation.
% For example, if most clusters near the query have very low LFD except for one anomaly with very high LFD,
% the harmonic mean LFD $\mu$ can still be low, so the factor of radial increase in (Equation)~\ref{eq:methods:repeated-rnn-factor} may be much larger than necessary for guaranteeing $k$ hits.
% This suggests that rather than using the reciprocal of the harmonic mean LFD in~\ref{eq:methods:repeated-rnn-factor}, we may achieve better results with a mean that is more sensitive to high outliers, such as the geometric mean.
% We leave it as an avenue for future work to characterize when Repeated $\rho$-NN significantly overestimates the correct radius $\rho_k$ and to improve upon the factor of radial increase in Equation~\ref{eq:methods:repeated-rnn-factor} so that this occurs less frequently and with less severity.


\subsection{Applicability}

Despite CAKES’s generality, effective deployment on real data can require domain knowledge. In protein sequence search, sequences vary widely in length (typically 30–1000 amino acids, with outliers) over a 20-letter alphabet. Numerous algorithms exist~\cite{kim2021entrance, daniels2013compressive, yu2015entropy, steinegger2018clustering}, and their development relied on deep domain expertise.

Simple edit distance can be poor when lengths vary substantially. CAKES can get around that issue by instead using alternative metrics or preprocessing—for example, represent each sequence as a set of $k$-mers and use Jaccard distance, as in~\cite{kim2021entrance}. If the distance relies on global alignment (e.g., Needleman–Wunsch~\cite{needleman1970general}), the dataset can be partitioned into mutually exclusive length bins with a separate CAKES tree per bin; at query time, route the sequence to its bin and a small number of adjacent bins.


\subsection{Future Work}

This study suggests several directions. First, we want to extended our evaluation to more datasets and distance functions not supported by FAISS, HNSW, or ANNOY, including Wasserstein distance~\cite{vallender1974calculation} for probability distributions (particularly high-dimensional) and Tanimoto distance~\cite{bajusz2015tanimoto} for comparing molecular structures via maximal common subgraphs. Adding a new distance to CAKES requires only a Rust implementation.

Second, we intend to improve the augmentation procedure (Sec.~\ref{sec:methods:synthetic-data}) to better preserve topological structure of datasets, e.g., by favoring perturbations along the leading principal components of the local manifold.

Third, we plan to explore CAKES algorithms' use in a streaming environment, enabling online updates to the tree as points are inserted or deleted. 

Finally, extend anomaly detection in CHAODA~\cite{ishaq2021clustered} by adding graph-based methods that use the distribution of distances among the $k$ nearest neighbors of cluster centers.

\subsection{Availability}

CLAM and CAKES are implemented in Rust and released under the MIT license at https://github.com/URI-ABD/clam.


    \section*{Acknowledgments}
    The authors thank the members of the University of Rhode Island's Algorithms for Big Data research group for their helpful comments throughout the development of this work.
    We are especially grateful to Carl Stoker and Rachel F. Daniels for their thorough reviews of the paper and valuable feedback.

    % \afterpage{\clearpage}
    \FloatBarrier
    \bibliographystyle{IEEEtran}
    \typeout{}
    \bibliography{references}
    % \newpage
    % % SIAM Supplemental File Template
\newif\ifarxiv
\arxivtrue % swap this to true to build for the arxiv
\ifarxiv
\documentclass{article}
\usepackage{arxiv}
\else
\documentclass[review,supplement,onefignum,onetabnum]{siamonline220329}
\fi

\input{cakes_shared}
\begin{document}
% Optional PDF information
\ifarxiv
\else
\ifpdf
\hypersetup{
  pdftitle={Supplementary Materials: Let them have CAKES: A Cutting-Edge Algorithm for Scalable, Efficient, and Exact Search on Big DataAn Example Article},
  pdfauthor={M. E. Prior, T. J. Howard III, O. McLaughlin, T. Ferguson, N. Ishaq, N. M. Daniels}
}
\fi
\fi

\maketitle


\section{Supplementary Methods}

\subsection{Algorithm Details}

\begin{algorithm} % enter the algorithm environment
    \caption{Partition($C$, $criteria$)} % give the algorithm a caption
    \label{alg:methods:partition} % and a label for \ref{} commands later in the document
    \begin{algorithmic} % enter the algorithmic environment
    \Require $f: X \times X \mapsto \mathbb{R}^+ \cup \{0\}$, a distance function
    \Require $C$, a cluster
    \Require $criteria$, user-specified continuation criteria

    \State $seeds \Leftarrow$ random sample of $\left\lceil \sqrt{|C|} \right\rceil$ points from $C$
    \State $c \Leftarrow$ geometric median of $seeds$
    \State $l \Leftarrow \argmax f(c, x) \ \forall \ x \in C$
    \State $r \Leftarrow \argmax f(l, x) \ \forall \ x \in C$
    \State $L \Leftarrow \{x \ | \ x \in C \land f(l, x) \le f(r, x)\}$
    \State $R \Leftarrow \{x \ | \ x \in C \land f(r, x) < f(l, x)\}$

    \If{$|L| > 1$ \textbf{and} $L$ satisfies $criteria$}
        \State Partition($L$, $criteria$)
    \EndIf

    \If{$|R| > 1$ \textbf{and} $R$ satisfies $criteria$}
        \State Partition($R$, $criteria$)
    \EndIf
\end{algorithmic}
\end{algorithm}

\begin{algorithm} % enter the algorithm environment
    \caption{Repeated $\rho$-NN($\mathcal{R}$, $q$, $k$)} % give the algorithm a caption
    \label{alg:methods:repeated-rnn} % and a label for \ref{} commands later in the document
    \begin{algorithmic} % enter the algorithmic environment
        \Require $\mathcal{R}$, the root cluster
        \Require $q$, a query
        \Require $k$, the number of neighbors to find
        \State{$H \Leftarrow [\ ]$, a max-heap by $\delta$ of size $k$}
        \State $r \Leftarrow radius$ of $\mathcal{R}$
        \State $r \Leftarrow$ $\frac{r}{|\mathcal{R}|}$
        \State $Q \Leftarrow$ tree-search($\mathcal{R}$, $q$, $r$)
        \While{$\sum_{C \in Q} |C| < k$}
            \If{$Q = \emptyset$}
                \State $r \Leftarrow 2 \cdot r$
            \ElsIf{$\sum_{C \in Q} |C| >= k$}
                \State $\mu \Leftarrow \frac{1}{|Q|} \cdot \sum_{C \in Q} \big( LFD(C)^{-1} \big)$
                \State $r \Leftarrow r \cdot \min \bigg( 2, \left( {\frac{k}{\sum_{C \in Q} |C|}} \right)^{\mu} \bigg)$
            \EndIf
            \State $Q \Leftarrow$ tree-search($\mathcal{R}$, $q$, $r$)
        \EndWhile
        \State $H \Leftarrow \text{leaf-search}(Q, q, r)$
        \State \textbf{return} $H$ as a list
    \end{algorithmic}
\end{algorithm}

\begin{algorithm}
    \caption{Breadth-First Sieve($\mathcal{R}$, $q$, $k$)} % give the algorithm a caption
    \label{alg:methods:bredth-first-sieve} % and a label for \ref{} commands later in the document
    \begin{algorithmic} % enter the algorithmic environment
        \Require $\mathcal{R}$, the root cluster
        \Require $q$, a query
        \Require $k$, the number of neighbors to find
        \State $c \Leftarrow$ \textit{center} of $\mathcal{R}$
        \State $Q \Leftarrow$ \{ ($\mathcal{R}$, $\delta^{+}_{\mathcal{R}}$, $|\mathcal{R}| - 1$), ($c$, $\delta_{\mathcal{R}}$, 1) \}
        \While{$\sum_{(\_, \_, m) \in Q} m \neq k$}
            \State $\tau \Leftarrow$ QuickSelect($Q$, $k$)
            \State $Q^{'} \Leftarrow \emptyset$
            \For{$(C, \_, \_) \in Q$}
                \If{$\delta^{-}_{C} \leq \tau$}
                    \If{$C$ is a point}
                        \State $Q^{'} \Leftarrow Q^{'} \cup \{ (C, \delta_{C}, 1) \}$
                    \ElsIf{$C$ is a leaf}
                        \State $Q^{'} \Leftarrow Q^{'} \cup \{ (p, \delta_{p}, 1)$ for $p \in C \}$
                    \Else
                        \State $[L, R] \Leftarrow$ children of $C$
                        \State $l, r \Leftarrow$ centers of $L, R$
                        \State $Q^{'} \Leftarrow Q^{'} \cup \{ (L, \delta^{+}_{L}, |L| - 1), (l, \delta_{L}, 1) \}$
                        \State $Q^{'} \Leftarrow Q^{'} \cup \{ (R, \delta^{+}_{R}, |R| - 1), (l, \delta_{R}, 1) \}$
                    \EndIf
                \EndIf
            \EndFor
            \State $Q \Leftarrow Q^{'}$
        \EndWhile
        \State QuickSelect($Q$, $k$)
        \State \textbf{return} The first $k$ points in $Q$
    \end{algorithmic}
\end{algorithm}

\begin{algorithm}
    \caption{Depth-First Sieve($\mathcal{R}$, $q$, $k$)}
    \label{alg:methods:depth-first-sieve}
    \begin{algorithmic}
        \Require $\mathcal{R}$, the root cluster
        \Require $q$, a query
        \Require $k$, the number of neighbors to find
        \State{$Q \Leftarrow [\mathcal{R}]$, a min-heap by $\delta^{-}$}
        \State{$H \Leftarrow [\ ]$, a max-heap by $\delta$ of size $k$}
        \While{$|H| < k$ \textbf{or} $H.peek.\delta \geq Q.peek.\delta^{-}$}
            \While{$Q.peek$ is not a leaf}
                \State{$C \Leftarrow Q.pop$, the closest cluster}
                \State{$[L, R] \Leftarrow$ children of $C$}
                \State{$Q.push(L)$}
                \State{$Q.push(R)$}
            \EndWhile
            \State{$leaf \Leftarrow Q.pop$}
            \For{$p \in leaf$}
                \State{$H.push(p)$}
            \EndFor
        \EndWhile
        \State \textbf{return} $H$
    \end{algorithmic}
\end{algorithm}


\subsection{Algorithm Details}
\label{sec:supplement:algorithm-details}

\subsection{\texorpdfstring{$\rho$}{p}-Nearest Neighbors Search}

We conduct $\rho$-NN search as described in~\cite{ishaq2019clustered}, but with the following improvement:
when a cluster overlaps with the query ball, instead of always proceeding to search both of its children, we proceed only with those children that might contain points in the query ball.

To determine whether both children can contain points in the query ball, we consider Figure~\ref{fig:supplement:overlapping-children}.
Here, we overload the notation for $\overline{x y}$ to refer both to the line segment joining points $x$ and $y$ as well as to the length of that line segment.

Let $q$ denote the query, $\rho$ denote the search radius, and $l$ and $r$ denote the cluster's left and right poles respectively.
Without loss of generality, we assume that $\overline{q r \vphantom{l}} \leq \overline{q l}$.
Now let $q'$ be the projection of $q$ onto $\overline{l r}$, $m$ be the midpoint of $\overline{l r}$, and $d$ be the distance from $q'$ to $m$.
As a consequence of how we assign a point in the parent cluster to the left child in the Partition algorithm, if $\rho < d$, then the left child cannot contain points inside the query ball.
In such a case we proceed to search only the right child.
Otherwise, we proceed with both children.

To check whether $d \leq \rho$, we note that $d = \overline{m q' \vphantom{l}} = \overline{m r \vphantom{l}} - \overline{q' r \vphantom{l}} = \frac{\overline{l r}}{2} - \overline{q' r \vphantom{l}}$.
Let $\theta$ denote $\angle l r q$, as shown in Figure~\ref{fig:supplement:overlapping-children}.
By the Law of Cosines on $\triangle l r q$, we have that $\text{cos}(\theta) = \tfrac{\overline{l r}^2 + \ \overline{q r \vphantom{l}}^2 - \ \overline{q l}^2}{2 \cdot \overline{l r} \cdot \overline{q r \vphantom{l}}}$.
Since $\triangle r q q'$ is a right triangle, we also have that $\text{cos}(\theta) = \tfrac{\overline{q' r \vphantom{l}}}{\overline{q r \vphantom{l}}}$.
Combining the previous two equations and solving for $\overline{q' r \vphantom{l}}$, we have that $\overline{q' r \vphantom{l}} = \tfrac{\overline{q r \vphantom{l}}^2 + \ \overline{l r}^2 - \ \overline{q l}^2}{2 \cdot \overline{l r}}$.
Substituting for $\overline{q' r \vphantom{l}}$ in the equation for $d$, we have that $d = \tfrac{\overline{l r}}{2} - \tfrac{\overline{q r \vphantom{l}}^2 + \ \overline{l r}^2 - \ \overline{q l}^2}{2 \cdot \overline{l r}} = \tfrac{\overline{q l}^2 - \overline{q r \vphantom{l}}^2}{2 \cdot \overline{l r}}$.


Thus, $d \leq \rho \iff (\overline{q l} + \overline{q r \vphantom{l}})(\overline{q l} - \overline{q r \vphantom{l}}) \leq 2 \cdot \overline{l r} \cdot \rho$. Note, in particular, that this only requires distances between actual points from the dataset, and so it can be used with any distance function, even when $q'$ and $m$ are not actual points or cannot be imputed from the data.

To perform $\rho$-NN search, we first perform a coarse \textit{tree-search}, to find the leaf clusters that overlap with the query ball or any clusters which lie entirely within the query ball.
Then, for all such clusters, we perform a finer-grained \textit{leaf-search}, to find all points that are no more than a distance $\rho$ from the query.
The asymptotic complexity of $\rho$-NN is the same as in~\cite{ishaq2019clustered} and shown in Equation~\ref{eq:supplement:rnn-search-complexity}.

\begin{gather}
    \mathcal{O}
    \Bigg(
        \underbrace{
            \log~\overbrace{\mathcal{N}_{\hat{r}}(X)}^{\textrm{metric entropy}}
        }_{\textrm{tree-search}}
        \ + \
        \underbrace{
            \overbrace{ \big| B_X(q, \rho) \big|}^{\textrm{output size}}
            \overbrace{ \left( \frac{\rho + 2 \cdot \hat{r}}{ \rho} \right) ^ d}^{\textrm{scaling factor}}
        }_{\textrm{leaf-search}}
    \Bigg)
    \label{eq:supplement:rnn-search-complexity}
\end{gather}
where $\hat{r}$ is the \textit{mean} radius of leaf clusters, $\mathcal{N}_{\hat{r}}(X)$ is the metric entropy at that radius, $B_X(q, \rho)$ is a ball of radius $\rho$ around the query $q$, and $d$ is the LFD around the query at the length scale of $\rho$ and $\rho + 2 \cdot \hat{r}$.


\begin{figure}
    \centering
    \includegraphics[scale=0.75]{images/geometry/overlapping-children-3.pdf}
    \caption{The geometry of a query ball overlapping with a cluster and either one or both of its children. Here, $l$ is the left pole, $r$ is the right pole, and $q$ is the query. Other points and distances are described in the text.}
    \label{fig:supplement:overlapping-children}
    \caption{CAKES uses geometric properties of clusters.}
\end{figure}


\section{Supplementary Results}

\subsection{Indexing and Tuning}

The plots in Figure~\ref{fig:supplement:indexing} show the results of these benchmarks.
The horizontal axis in each subplot shows the cardinality of the dataset augmented with synthetic points.
The left-most point on each line is at the cardinality of the original dataset without any synthetic augmentation.
The vertical axis denotes the sum of indexing and tuning time in seconds.
Both axes are on a logarithmic scale.
Hereafter, when we refer to the ``indexing time'' of an algorithm, we are implicitly referring to the sum of indexing and tuning time for said algorithm.

On all datasets, we observe that the indexing time for CAKES increases roughly linearly as cardinality increases.
HNSW and ANNOY have the slowest indexing times across all the algorithms we benchmarked for each of the four datasets, at each cardinality.
On some datasets, HNSW and ANNOY exhibit indexing times which are orders of magnitude slower than that of CAKES.
FAISS-Flat exhibits the fastest indexing time on each dataset.
This is not surprising, however, given that FAISS-Flat is a na\"{\i}ve linear search algorithm and is not building an index.

We also highlight some differences in indexing time between different datasets.
With Fashion-Mnist, as shown in Figure~\ref{fig:supplement:fashion-mnist-indexing}, we observe that the indexing time for CAKES is faster than that of FAISS-IVF for all cardinalities.
With Glove-25 (see Figure~\ref{fig:supplement:glove-25-indexing}), however, at cardinalities greater than $10^7$, FAISS-IVF has faster indexing time than CAKES.
With Sift, CAKES's indexing time is faster than that of FAISS-IVF until a cardinality of nearly $10^8$, and with the Random dataset, we observe that CAKES has faster indexing time than FAISS-IVF until a cardinality of nearly $10^7$.

\begin{figure}
    \captionsetup[subfigure]{aboveskip=-15pt,belowskip=-3pt}
    \begin{subfigure}[b]{0.47\textwidth}
        \includegraphics[width=0.9\textwidth]{images/indexing/fashion-mnist-indexing.png}\\
        \subcaption{Fashion-mnist}
        \label{fig:supplement:fashion-mnist-indexing}
    \end{subfigure}%
    \begin{subfigure}[b]{0.47\textwidth}
        \includegraphics[width=0.9\textwidth]{images/indexing/glove-25-indexing.png}\\
        \subcaption{Glove-25}
        \label{fig:supplement:glove-25-indexing}
    \end{subfigure}
    \\
    \begin{subfigure}[b]{0.47\textwidth}
        \includegraphics[width=0.9\textwidth]{images/indexing/sift-indexing.png}\\
        \subcaption{Sift}
        \label{fig:supplement:sift-indexing}
    \end{subfigure}%
    \begin{subfigure}[b]{0.47\textwidth}
        \includegraphics[width=0.9\textwidth]{images/indexing/random-indexing.png}\\
        \subcaption{Random}
        \label{fig:supplement:random-indexing}
    \end{subfigure}%
    \\
    \caption{Indexing and tuning time for each algorithm with each of the ANN benchmark datasets and the Random dataset.}
    \label{fig:supplement:indexing}
\end{figure}

\begin{figure}
    \begin{subfigure}[b]{0.47\textwidth}
        \includegraphics[width=1.0\textwidth]{plots/fashion-mnist_PermutedBall_100_throughput.png}
        \subcaption{Fashion-Mnist for $k=100$.}
        \label{fig:supplement:fashion-mnist-scaling}
    \end{subfigure}%
    \begin{subfigure}[b]{0.47\textwidth}
        \includegraphics[width=1.0\textwidth]{plots/glove-25_PermutedBall_100_throughput.png}
        \subcaption{Glove-25 for $k=100$.}
        \label{fig:supplement:glove-25-scaling}
    \end{subfigure}%
    \\
    \begin{subfigure}[b]{0.47\textwidth}
        \includegraphics[width=1.0\textwidth]{plots/sift_PermutedBall_100_throughput.png}
        \subcaption{Sift for $k=100$.}
        \label{fig:supplement:sift-scaling}
    \end{subfigure}%
    \begin{subfigure}[b]{0.47\textwidth}
        \includegraphics[width=1.0\textwidth]{plots/random_PermutedBall_100_throughput.png}
        \subcaption{A random dataset for $k=100$.}
        \label{fig:supplement:random-scaling}
    \end{subfigure}%
    \\
    \begin{subfigure}[b]{0.47\textwidth}
        \includegraphics[width=1.0\textwidth]{plots/silva-SSU-Ref_PermutedBall_100_throughput.png}
        \subcaption{Silva for $k=100$.}
        \label{fig:supplement:silva-scaling}
    \end{subfigure}%
    \begin{subfigure}[b]{0.47\textwidth}
        \includegraphics[width=1.0\textwidth]{plots/radio-ml_Ball_100_throughput.png}
        \subcaption{RadioML for $k=100$ at SnR = 10dB.}
        \label{fig:supplement:radioml-scaling}
    \end{subfigure}%
    \\
    \begin{subfigure}[b]{0.94\textwidth}
        \centering
        \includegraphics[width=0.7\textwidth]{plots/legend.png}
        \label{fig:supplement:scaling-legend}
    \end{subfigure}%
    \caption{Throughput (with $k$=100) across six datasets, including a randomly-generated dataset.
    In each plot, the horizontal axis represents increasing cardinality of the dataset, while the vertical axis represents the throughput in queries per second (higher is better).
    For linear search with CAKES, we only report the throughput for a few of the initial multipliers because the trend is clear.}
    \label{fig:supplement:scaling-plots}
\end{figure}


\subsection{Clustering}

\begin{figure}
    \captionsetup[subfigure]{aboveskip=-15pt,belowskip=-3pt}
    \begin{subfigure}[b]{0.47\textwidth}
        \includegraphics[width=0.9\textwidth]{images/radius/fashion-mnist.png}\\
        \subcaption{Fashion-mnist}
        \label{fig:supplement:fashion-mnist-radius}
    \end{subfigure}%
    \begin{subfigure}[b]{0.47\textwidth}
        \includegraphics[width=0.9\textwidth]{images/radius/glove-25.png}\\
        \subcaption{Glove-25}
        \label{fig:supplement:glove-25-radius}
    \end{subfigure}
    \\
    \begin{subfigure}[b]{0.47\textwidth}
        \includegraphics[width=0.9\textwidth]{images/radius/sift.png}\\
        \subcaption{Sift}
        \label{fig:supplement:sift-radius}
    \end{subfigure}%
    \begin{subfigure}[b]{0.47\textwidth}
        \includegraphics[width=0.9\textwidth]{images/radius/random.png}\\
        \subcaption{A random dataset}
        \label{fig:supplement:random-radius}
    \end{subfigure}
    \\
    \begin{subfigure}[b]{0.47\textwidth}
        \includegraphics[width=0.9\textwidth]{images/radius/silva-SSU-Ref.png}\\
        \subcaption{Silva 18S}
        \label{fig:supplement:silva-radius}
    \end{subfigure}%
    \begin{subfigure}[b]{0.47\textwidth}
        \includegraphics[width=0.9\textwidth]{images/radius/radio-ml.png}\\
        \subcaption{RadioML}
        \label{fig:supplement:radioml-radius}
    \end{subfigure}%
    \\
    \vskip 0.005in
    \begin{subfigure}[b]{0.94\textwidth}
        \centering
        \includegraphics[width=0.7\textwidth]{images/radius/legend.png}
        \label{fig:supplement:radius-legend}
    \end{subfigure}%
    \caption{Radius vs. cluster depth across six datasets, grouped by percentile of radius and weighted by the cardinalities of the clusters.
    In order to use the same y-axis for all plots, we divided the radii of all clusters by the maximum radius of any cluster in the dataset.
    Note that for the Silva 18S and the RadioML datasets, we use a logarithmic scale for the y-axis.}
    \label{fig:supplement:radius-plots}
\end{figure}

\begin{figure}
    \captionsetup[subfigure]{aboveskip=-15pt,belowskip=-3pt}
    \begin{subfigure}[b]{0.47\textwidth}
        \includegraphics[width=0.9\textwidth]{images/fractal_density/fashion-mnist.png}\\
        \subcaption{Fashion-mnist}
        \label{fig:supplement:fashion-mnist-fractal_density}
    \end{subfigure}%
    \begin{subfigure}[b]{0.47\textwidth}
            \includegraphics[width=0.9\textwidth]{images/fractal_density/glove-25.png}\\
            \subcaption{Glove-25}
            \label{fig:supplement:glove-25-fractal_density}
    \end{subfigure}
    \\
    \begin{subfigure}[b]{0.47\textwidth}
        \includegraphics[width=0.9\textwidth]{images/fractal_density/sift.png}\\
        \subcaption{Sift}
        \label{fig:supplement:sift-fractal_density}
    \end{subfigure}%
    \begin{subfigure}[b]{0.47\textwidth}
        \includegraphics[width=0.9\textwidth]{images/fractal_density/random.png}\\
        \subcaption{A random dataset}
        \label{fig:supplement:random-fractal_density}
    \end{subfigure}
    \\
    \begin{subfigure}[b]{0.47\textwidth}
        \includegraphics[width=0.9\textwidth]{images/fractal_density/silva-SSU-Ref.png}\\
        \subcaption{Silva 18S}
        \label{fig:supplement:silva-fractal_density}
    \end{subfigure}%
    \begin{subfigure}[b]{0.47\textwidth}
        \includegraphics[width=0.9\textwidth]{images/fractal_density/radio-ml.png}\\
        \subcaption{RadioML}
        \label{fig:supplement:radioml-fractal_density}
    \end{subfigure}%
    \\
    \vskip 0.005in
    \begin{subfigure}[b]{0.94\textwidth}
        \centering
        \includegraphics[width=0.7\textwidth]{images/fractal_density/legend.png}
        \label{fig:supplement:fractal_density-legend}
    \end{subfigure}%
    \caption{Fractal Density vs. cluster depth across six datasets, grouped by percentile of fractal density and weighted by the cardinalities of the clusters.
    Fractal Density is defined as $\frac{cardinality}{radius^{LFD}}$.
    As with the radius plots, we normalized the cluster radii before calculating the Fractal Density.
    We also excluded all clusters with normalized radii smaller than $10^{-6}$, as they distort the scale of the plots.}
    \label{fig:supplement:fractal_density-plots}
\end{figure}

\subsection{Clustering Strategies and Number of Distance Computations}
\label{sec:results:clustering-strategies-and-number-of-distance-computations}

In addition to the scaling experiments, we also explore how four different clustering strategies affect the performance of search.
These strategies are the Cartesian product of balanced vs. unbalanced clustering, and the presence vs. absence of depth-first reordering as described in Section~\ref{sec:methods:clustering:depth-first-reordering}.
To help with this analysis, we also added some instrumentation to our implementation of CAKES to count the number of distance computations performed during search.
We note that this instrumentation significantly slows down the wall-clock time of search, so it would not be used in a real-world application.

These four comparisons (referred to as Ball, BalancedBall, PermutedBall, and, PermutedBalancedBall, where depth-first-reordering is referred to as Permuted for brevity) for all three search algorithms on the Fashion-MNIST dataset are shown in Figure~\ref{fig:results:distance-counts}.
Notably, performance is not only strictly worse for the balanced approaches, but the asymptotic behavior is significantly worse;
the algorithms (such as Depth-First Sieve) that seemed to exhibit constant-time behavior in Figure~\ref{fig:results:scaling-plots}, which shows results using PermutedBall, no longer do so under a balanced clustering.
We note that on this dataset, depth-first reordering (the Permuted variants) seems to have little effect on throughput; depth-first reordering is primarily intended to improve space complexity.


\bibliographystyle{siamplain}
\bibliography{references}


\end{document}

\end{document}
